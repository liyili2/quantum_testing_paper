\usepackage{xcolor}
\usepackage{colortbl}

\usepackage[utf8]{inputenc}

\usepackage{amsmath}
\usepackage{stmaryrd}
\usepackage{hyperref}

\usepackage{subcaption}
%\usepackage{subfig}
\usepackage{wrapfig}

\usepackage{xspace}
\usepackage{float}
\usepackage{balance}

\usepackage{textcomp} %just for a vertical quote

\usepackage{changepage}
\usepackage{array,etoolbox}

\preto\tabular{\setcounter{magicrownumbers}{0}}
\newcounter{magicrownumbers}
\newcommand\rownumber{\stepcounter{magicrownumbers}\arabic{magicrownumbers}}


% References
%\usepackage[capitalize, noabbrev]{cleveref} % must be loaded after hyperref and amsmath
\usepackage{cleveref} % must be loaded after hyperref and amsmath

% Figures
\usepackage{wrapfig}
\usepackage{caption}

% Tables
\usepackage{longtable}
\usepackage{tabu}

% Lists
%\usepackage[inline,shortlabels]{enumitem}
\usepackage{wasysym}
% Proof trees
\usepackage{mathpartir}
%\usepackage{bussproofs}

% Symbolx
\usepackage{mathtools} %psmallmatrix
%\usepackage{stmaryrd} % more math symbols
\usepackage{xfrac} % fractions

% Enumeration
% \usepackage{enumitem} % resume option

% Quantum
\usepackage[qm,braket]{qcircuit}
%\usepackage{braket}
\newcommand{\ketbra}[2]{\ket{#1}\!\bra{#2}}

\hyphenation{Comp-Cert}

% sessions
\newcommand{\stuple}[3]{{#1}[#2, #3)}

%Language names
\newcommand{\qafnyl}{$\textsc{L}_\textsc{Qafny}$\xspace}
\newcommand{\qass}[2]{{#1}\;{\leftarrow}\;{#2}}
\newcommand{\qfor}[4]{\texttt{{for} }{#1}\texttt{ \(\in\) }[{#2},{#3})\texttt{ {\&\&} }{#4}}
\newcommand{\qif}[2]{\texttt{{if}\,(}{#1}\texttt{)\,\{}{#2}\texttt{\}}}
\newcommand{\sifq}[2]{\texttt{if}~\cn{(}{#1}\cn{)}~{#2}}
\newcommand{\modmult}[3]{(#1 * #2)\cn{\,\%} \, #3}
\newcommand{\modexp}[3]{{#1}^{#2}\,\cn{\%} \, #3}
\newcommand{\qbool}[4]{\cn{(}#1\,\cn{#2}\,#3\cn{)\,\cn{@}\,}#4}
\newcommand{\qboola}[3]{#1\,\cn{#2}\,#3}
\newcommand{\mmod}{\cn{\%}}
\newcommand{\Hassert}[1]{\{\ #1\ \}}
\colorlet{kwd}{black!80!green}
\definecolor{spec1}{RGB}{78, 131, 162}
\definecolor{spec0}{RGB}{66, 102, 136}
\definecolor{lespec}{RGB}{30, 80, 180}
\colorlet{spec}{lespec}
\colorlet{auto}{lespec!35!lightgray}
\colorlet{stack}{magenta}

\newcommand{\qafny}{\rulelab{Qafny}\xspace}
\newcommand{\sep}{\rulelab{Sep}\xspace}
%\newcommand{\qafnyl}{\textsc{QafnyL}\xspace}
\newcommand{\name}{\textsc{VQO}\xspace}
\newcommand{\qvm}{\textsc{qvm}\xspace}
\newcommand{\sourcelang}{\ensuremath{\mathcal{O}\textsc{qimp}}\xspace}
%\newcommand{\pqasm}{\textsc{oqasm}\xspace}
\newcommand{\oqasm}{\textsc{Oqasm}\xspace}
\newcommand{\preq}{\textsc{Preq}\xspace}
\newcommand{\pqasm}{\ensuremath{\textsc{Pqasm}}\xspace}
\newcommand{\intlang}{\oqasm}
\newcommand{\vqimp}{\sourcelang}
\newcommand{\vqir}{\intlang}
\newcommand{\sqir}{SQIR\xspace}
\newcommand{\coqq}{{CoqQ}\xspace}
\newcommand{\qwire}{{Qwire}\xspace}
\newcommand{\qbricks}{{QBricks}\xspace}
\newcommand{\liquids}{LIQ\emph{Ui}$\ket{}$\xspace}
\newcommand{\liquid}{Liquid\xspace}
\newcommand{\revs}{R\textsc{evs}\xspace}
\newcommand{\reverc}{R\textsc{e}V\textsc{er}C\xspace}
\newcommand{\fstar}{F${}^\star$\xspace}
\newcommand{\voqc}{\textsc{VOQC}\xspace}
\newcommand{\tket}{t$\vert$ket$\rangle$\xspace}
\newcommand{\myparagraph}[1]{\noindent\paragraph{\textbf{#1}}}

%quantum tikz macros
% These are useful if not using qcircuit (which kind of sucks, alternatives do exist
\usepackage{tikz}
\newcommand{\mycontrol}[2]{\draw[fill=black] (#1,#2) circle [radius=0.10];}
\newcommand{\qnot}[2]{\draw (#1,#2) circle [radius=0.20]; \draw (#1,#2-0.20) -- (#1,#2+0.20);}
\newcommand{\cnot}[3]{\qnot{#1}{#3}\mycontrol{#1}{#2}\draw (#1,#2) -- (#1,#3);}
\newcommand{\tof}[4]{\qnot{#1}{#4}\mycontrol{#1}{#2}\mycontrol{#1}{#3}\draw (#1,#2) -- (#1,#4);\draw (#1,#3) -- (#1,#4);}
\newcommand{\unitary}[3]{\draw[fill=white] (#2-0.4,#3-0.4) rectangle node {\texttt{#1}} (#2+0.4,#3+0.4);}
\newcommand{\had}[2]{\unitary{H}{#1}{#2}}
\newcommand{\meas}[2]{\draw[fill=white] (#1-0.8,#2-0.4) rectangle node {\texttt{meas}} (#1+0.8,#2+0.4);}
\newcommand{\discard}[2]{\draw[thick](#1,#2-0.2) -- (#1,#2+0.2);}

% From POPL2017
\newcommand{\aket}[2]{\ket{#1}_{\textcolor{spec}{#2}}}
\newcommand{\qket}[1]{\ket{\Delta(#1)}}

\newcommand{\shows}{\ensuremath{\vdash}}
\newcommand{\mprod}{\mathbin{\text{\footnotesize \ensuremath{\otimes}}}}
\newcommand{\aprod}{\mathbin{\text{\footnotesize \ensuremath{\&}}}}
\newcommand{\msum}{\mathbin{\text{\footnotesize \ensuremath{\parr}}}}
\newcommand{\asum}{\mathbin{\text{\footnotesize \ensuremath{\oplus}}}}
\newcommand{\lolto}{\ensuremath{\multimap}}
\newcommand{\bang}{\ensuremath{\oc}}
\newcommand{\whynot}{\ensuremath{\wn}}
\newcommand{\one}{\ensuremath{1}}
\newcommand{\zero}{\ensuremath{0}}
\newcommand{\interp}[1]{[ #1 ]}
\newcommand{\interpalt}[1]{\llbracket #1 \rrbracket}
\newcommand{\ecode}[1]{\emph{\texttt{#1}}}
\newcommand{\cmode}{\texttt{C}}
\newcommand{\mmode}{\texttt{M}}
\newcommand{\qmodename}{\texttt{Q}}
\newcommand{\qmode}[1]{\texttt{Q}~#1}
\newcommand{\topt}[1]{#1\;\texttt{opt}}
\newcommand{\slen}[1]{|#1|}
% From QPL2017
\newenvironment{nscenter}
 {\parskip=3pt\par\nopagebreak\centering}
 {\par\noindent\ignorespacesafterend}

% From CoqPL2018
\newcommand{\dimx}[1]{\text{dim}_x(#1)}
\newcommand{\dimy}[1]{\text{dim}_y(#1)}

% From QPL2018

% Standard mathematical definitions
% Field
\newcommand{\F}{\ensuremath{\mathbb{F}}\xspace}
% Integers
\newcommand{\Z}{\ensuremath{\mathbb{Z}}\xspace}
% Naturals
\newcommand{\N}{\ensuremath{\mathbb{N}}\xspace}
% Rationals
\newcommand{\Q}{\ensuremath{\mathbb{Q}}\xspace}
% Reals
\newcommand{\R}{\ensuremath{\mathbb{R}}\xspace}
% Complex
\newcommand{\CC}{\ensuremath{\mathbb{C}}\xspace}
% \newcommand{\C}{\ensuremath{\mathbb{C}}\xspace}

%   Tikz
\usepackage{pgf}

\usepackage{tikz} % circuit diagrams
\usetikzlibrary{%
  arrows,%
  shapes.misc,% wg. rounded rectangle
  shapes.geometric, % diamonds
  shapes.arrows,%
  shapes.callouts,
  shapes.gates.logic.US,
  chains,%
  matrix,%
  positioning,% wg. " of "
  scopes,%
  decorations.pathmorphing,% /pgf/decoration/random steps | erste Graphik
  decorations.text,
  decorations.pathreplacing, % braces
  shadows,%
  automata,
  fit, calc, arrows.meta
}

\tikzset{ machine/.style={
    % The shape:
    rectangle,
    % The size:
    minimum width=25mm,
    minimum height=18mm,
    text width=24mm,
    % The alignment
    align=center,
    % The border:
    very thick,
    draw=black,
    % The colors:
    color=black,
    fill=white,
    % Font
%    font=\ttfamily,
  }
}

% Place two figures side-by-side, possibly overlapping
\newenvironment{pairfigures}[4]{%
\newcommand{\foot}{#4} % yay, clever hackery!
  \let\and&
  \begin{center}%
  \vspace{#3\baselineskip}%
  \begin{tabular}{m{#1}m{#2}}%
}{%
  \end{tabular}
  \vspace{\foot\baselineskip}
  \end{center}
}

\newcommand\wideparen[1]{%
\tikz[baseline=(wideArcAnchor.base)]{
    \node[inner sep=0] (wideArcAnchor) {$#1$}; 
    \coordinate (wideArcAnchorA) at ($0.9*(wideArcAnchor.north west) + 0.1*(wideArcAnchor.north east)+(0.0em,0.75ex)$);
    \coordinate (wideArcAnchorB) at ($0.1*(wideArcAnchor.north west) + 0.9*(wideArcAnchor.north east)+(0.0em,0.75ex)$);
%   
    \draw[line width=0.1ex,line cap=round] 
        ($(wideArcAnchor.north west)+(0.0em,0.1ex)$) 
            .. controls (wideArcAnchorA) and (wideArcAnchorB) ..
        ($(wideArcAnchor.north east)+(0.0em,0.1ex)$)        
    ;
}}
\newcommand{\cmsg}[1]{\wideparen{#1}}
\newcommand{\function}[2]{#1~#2}
\newcommand*\rfrac[2]{{}^{#1}\!/_{#2}}
\DeclarePairedDelimiter\abs{\lvert}{\rvert}
\DeclarePairedDelimiter\norm{\lVert}{\rVert}

\makeatletter
\let\oldabs\abs
\def\abs{\@ifstar{\oldabs}{\oldabs*}}
%
\let\oldnorm\norm
\def\norm{\@ifstar{\oldnorm}{\oldnorm*}}
\makeatother

%\DeclarePairedDelimiter{\inpar}[2]{}{}{#1\;\delimsize\|\;#2}
\newcommand{\inpar}{\mathbin{\|}}

% Proper division symbol
\makeatletter
\DeclareRobustCommand{\vardivision}{%
  \mathbin{\mathpalette\@vardivision\relax}% 
}
\newcommand{\@vardivision}[2]{%
  \reflectbox{$\m@th\smallsetminus$}%
}
\makeatother

\DeclareFontFamily{U} {MnSymbolC}{}
\DeclareFontShape{U}{MnSymbolC}{m}{n}{
  <-6> MnSymbolC5
  <6-7> MnSymbolC6
  <7-8> MnSymbolC7
  <8-9> MnSymbolC8
  <9-10> MnSymbolC9
  <10-12> MnSymbolC10
  <12-> MnSymbolC12}{}
\DeclareFontShape{U}{MnSymbolC}{b}{n}{
  <-6> MnSymbolC-Bold5
  <6-7> MnSymbolC-Bold6
  <7-8> MnSymbolC-Bold7
  <8-9> MnSymbolC-Bold8
  <9-10> MnSymbolC-Bold9
  <10-12> MnSymbolC-Bold10
  <12-> MnSymbolC-Bold12}{}

\DeclareSymbolFont{MnSyC} {U} {MnSymbolC}{m}{n}

\DeclareMathSymbol{\sqcupplus}{\mathbin}{MnSyC}{70}

%\DeclareRobustCommand{\sqcupplus}{\mbox{\usefont{U}{MnSymbolC}{m}{n}\char70}}

% Latin  Abbr
\newcommand{\etal}{\emph{et al.}\xspace}
\newcommand{\eg}{\emph{e.g.,}\xspace}
\newcommand{\ie}{\emph{i.e.,}\xspace}
\newcommand{\etc}{\emph{etc.}\xspace}

% Math commands
\DeclareMathOperator{\tr}{tr}



% Table formatting
\newcommand{\ltwo}[1]{\multicolumn{2}{l}{#1}}
\newcommand{\ctwo}[1]{\multicolumn{2}{c}{#1}}
\usepackage{booktabs}

% Types
\newcommand{\One}{\code{One}}
\newcommand{\Bit}{\code{Bit}}
\newcommand{\Qubit}{\code{Qubit}}
% Qwire Syntax
\newcommand{\Gate}[2]{\code{Gate}(#1,#2)}
\newcommand{\Unitary}[2]{\code{Unitary}(#1,#2)}
\renewcommand{\Box}[2]{\code{Box}(#1,#2)}
\newcommand{\mkbox}[2]{\code{box}(#1 \Rightarrow #2)}
\newcommand{\qcontrol}[1]{\code{control}~#1}  % \control conflicts with qcircuit package
\newcommand{\bitcontrol}[1]{\code{bit-control}~#1}

% Code
\newcommand{\None}{\code{None}}
\newcommand{\Some}[1]{\code{Some}~#1}

% Listings
\usepackage{alltt}
\usepackage{listings,lstcoq}
%\usepackage{MnSymbol}
\definecolor{ltblue}{rgb}{0,0.4,0.4}
\definecolor{dkblue}{rgb}{0,0.1,0.6}
\definecolor{dkgreen}{rgb}{0,0.35,0}
\definecolor{dkviolet}{rgb}{0.3,0,0.5}
\definecolor{dkred}{rgb}{0.5,0,0}
\lstset{language=Coq}
\usepackage[export]{adjustbox}

\newcommand{\code}[1]{{\small\texttt{#1}}}
%\newcommand{\mcode}[1]{{\small\mathtt{#1}}}
%\newcommand{\code}[1]{\lstinline{#1}}

% Preventing pagebreaks
\newenvironment{absolutelynopagebreak}
  {\par\nobreak\vfil\penalty0\vfilneg
   \vtop\bgroup}
  {\par\xdef\tpd{\the\prevdepth}\egroup
   \prevdepth=\tpd}

 %% VQIMP syntax, semantic values
 \newcommand{\rulelab}[1]{{\small \textsc{#1}}}
\newcommand{\steps}{\ensuremath{\longrightarrow}}
\newcommand{\tbool}{\texttt{bool}}
\newcommand{\tsizeof}{\texttt{sizeof}}
\newcommand{\vbool}{\texttt{Bool}}
\newcommand{\verror}{\texttt{Error}}
\newcommand{\vval}{\mathpzc{val}}
\newcommand{\tfixed}{\texttt{fixedp}}
\newcommand{\tnat}{\texttt{nat}}
\newcommand{\tarr}[2]{\texttt{array}~{#1}~{#2}}
\newcommand{\econst}[2]{({#1}){#2}}
\newcommand{\eindex}[2]{{#1}\texttt{[}{#2}\texttt{]}}

\newcommand{\sexp}[3]{\texttt{let}~{#1}\,\texttt{=}\,{#2}~\texttt{in}~{#3}}
\newcommand{\sexph}[2]{\texttt{let}~{#1}\,\texttt{=}\,{#2}~\texttt{in}}
\newcommand{\sskip}{\texttt{\{\}}}

\newcommand{\sifb}[3]{\texttt{if}~{(#1)}~{#2}~\texttt{else}~{#3}}
%\newcommand{\swhile}[3]{\texttt{while}~{(#1)}~{#2}~{\{#3\}}}
\newcommand{\sqwhile}[5]{\texttt{for}~{#1 \in [#2,#3)~\texttt{\&\&}~#4}~#5}
\newcommand{\sqfora}[4]{\texttt{for}~{#1 \in [#2,#3)~\texttt{\&\&}~#4}}
\newcommand{\sqforh}[4]{\texttt{for}~{#1;~#2~\texttt{\&\&}~#3~;~#4}}
\newcommand{\sinit}[1]{\texttt{init}~{#1}}
\newcommand{\sint}[2]{\texttt{int}~{#1}\texttt{:=}\,{#2}}
\newcommand{\sincr}[1]{{#1}\texttt{++}}
\newcommand{\tnor}[1]{\texttt{Nor}({#1})}
\newcommand{\tnort}{\texttt{Nor}}
\newcommand{\trot}[1]{\texttt{Rot}({#1})}
\newcommand{\trott}{\texttt{Rot}}
%\newcommand{\thad}[3]{\texttt{Had}({#1},#2,#3)}
\newcommand{\thad}[1]{\texttt{Had}(#1)}
\newcommand{\thadt}{\texttt{Had}}
\newcommand{\tch}[3]{(#1,#2,#3)}
\newcommand{\shad}[3]{\frac{1}{\sqrt{#1}}\Motimes_{j=0}^{#2}{(\ket{0}+#3\ket{1})}}
\newcommand{\shadi}[3]{\frac{1}{\sqrt{#1}}\Motimes_{i=0}^{#2}{(\ket{0}+#3\ket{1})}}
\newcommand{\ssum}[3]{\Msum_{#1}^{#2}{#3}}
\newcommand{\sch}[3]{\Msum_{j=0}^{#1}{#2 {#3}}}
\newcommand{\scha}[4]{\Msum_{j=0}^{#1}{#2\ket{#3}#4}}
\newcommand{\schai}[4]{\Msum_{i=0}^{#1}{#2\ket{#3}#4}}
\newcommand{\schb}[2]{\Msum_{j=0}^{#1}{q(#2)}}
\newcommand{\paren}[1]{\left(#1\right)}
\newcommand{\typing}[5]{{#1},{#2}\vdash_{#3} {#4} \triangleright{#5}}
\newcommand{\smch}[3]{#1\Msum_{k=0}^{#2}{\ket{#3}}}
\newcommand{\smich}[4]{#1\Msum_{k=0}^{#2}{#3\ket{#4}}}
\newcommand{\schk}[3]{\Msum_{k=0}^{#1}{#2\ket{#3}}}
\newcommand{\schii}[3]{\Msum_{i=0}^{#1}{#2\ket{#3}}}
\newcommand{\schd}[4]{\Msum_{#1=0}^{#2}{#3\ket{#4}}}
\newcommand{\schia}[3]{\Msum_{i=0}^{#1}{#2#3}}
\newcommand{\schi}[3]{\Msum^{#1}{#2\ket{#3}}}
\newcommand{\tpower}[1]{\mathcal{P}(#1)}
\newcommand{\iev}[1]{\texttt{ev}(#1)}
\newcommand{\itau}{\mathcal{T}}
\newcommand{\snat}[1]{\texttt{nat}({#1})}
\newcommand{\sord}[1]{\texttt{ord}({#1})}
\newcommand{\srnd}[1]{\texttt{rnd}({#1})}
\DeclareMathOperator*{\Motimes}{\text{\raisebox{0.25ex}{\scalebox{0.8}{$\bigotimes$}}}}
\DeclareMathOperator*{\Msum}{\text{\raisebox{0.25ex}{\scalebox{0.8}{$\sum$}}}}
\newcommand{\frz}{\mathpzc{F}}
\newcommand{\mea}{\mathpzc{M}}
\newcommand{\ufrz}{\mathpzc{U}}
\newcommand{\ttype}[2]{\Motimes_{#1}~{#2}}
\newcommand{\cupdot}{\mathbin{\mathaccent\cdot\cup}}
\newcommand{\ias}[1]{\texttt{as}(#1)}
\newcommand{\iasa}[2]{\texttt{as}^{#1}(#2)}
\newcommand{\ibs}[1]{\texttt{bs}(#1)}
\newcommand{\ips}[1]{\texttt{ps}(#1)}
\newcommand{\fivepule}[6]{#1;#2\vdash_{#3}\big{\{}#4\big{\}}\, #5 \,\big{\{}#6\big{\}}}

\newcommand{\triple}[3]{\big{\{}#1\big{\}}\, #2 \,\big{\{}#3\big{\}}}

\newcommand{\sassign}[4]{{#1} \leftarrow {#3}~{#2}~{#4}}
\newcommand{\ssassign}[3]{{#1} \xleftarrow{#2} {#3}}
\newcommand{\sand}[2]{{#1}~{\&\&}~{#2}}
\newcommand{\samp}[1]{\texttt{amp}(#1)}
\newcommand{\sreflect}[1]{\texttt{reflect}\{#1\}}
\newcommand{\sdis}{\texttt{dis}}
\newcommand{\sreduce}[2]{\texttt{reduce}(#1,#2)}
\newcommand{\samplify}[1]{\texttt{amplify}\{#1\}}
\newcommand{\sdistr}[1]{\texttt{diffuse}(#1)}
\newcommand{\sret}[1]{\texttt{ret}(#1)}
\newcommand{\size}[1]{\texttt{size}(#1)}
\newcommand{\snext}[1]{#1\splus 1}
\newcommand{\sminus}{\texttt{-}}
\newcommand{\splus}{\texttt{+}}
\newcommand{\slt}{\texttt{<}}
\newcommand{\srange}[2]{[#1,#2)}
\newcommand{\sfor}[3]{\texttt{for}~{#1}~{#2}~{#3}}
\newcommand{\scall}[3]{{#1}\leftarrow {#2}~{#3}}
\newcommand{\sseq}[2]{{#1}\,\texttt{;}\,{#2}}
\newcommand{\sinv}[1]{\texttt{inv}~{#1}}
\newcommand{\inst}[3][ ]{\texttt{#2}^{#1}~{#3}}
\newcommand{\insttwo}[4][ ]{\texttt{#2}^{#1}~{#3}~{#4}}
\newcommand{\instthree}[5][ ]{\texttt{#2}^{#1}~{#3}~{#4}~{#5}}
\newcommand{\iu}[1]{\inst{U}{#1}}
\newcommand{\inot}[1]{\inst{X}{#1}}
\newcommand{\ictrl}[2]{\insttwo{CU}{#1}{#2}}
\newcommand{\iadd}[2]{\cn{add}(#1,#2)}
\newcommand{\irz}[3][ ]{\insttwo[#1]{RZ}{#2}{#3}}
\newcommand{\isr}[3][ ]{\insttwo[#1]{SR}{#2}{#3}}
\newcommand{\icnot}[2]{\insttwo{CNOT}{#1}{#2}}
\newcommand{\ilshift}[1]{\inst{Lshift}{#1}}
\newcommand{\irshift}[1]{\inst{Rshift}{#1}}
\newcommand{\irev}[1]{\inst{Rev}{#1}}
\newcommand{\iqft}[3][ ]{\insttwo[#1]{QFT}{#2}{#3}}
\newcommand{\iqfth}[1][ ]{\texttt{QFT}^[#1]}
\newcommand{\tphi}[1]{\texttt{Phi}~{#1}}
%\newcommand{\thad}[1]{\texttt{Had}~{#1}}
\newcommand{\ihad}[1]{\texttt{H}(#1)}
\newcommand{\inew}[1]{\texttt{new}(#1)}
\newcommand{\iry}[2]{\texttt{Ry}^{#1}{#2}}
\newcommand{\ihadh}{\texttt{H}}
\newcommand{\isp}[2]{\mathcal{#1}_{#2}}
\newcommand{\hsp}[1]{\mathcal{#1}}
\newcommand{\iseq}[2]{{#1}\,\texttt{;}\,{#2}}
\newcommand{\inval}[2]{\insttwo{Nval}{#1}{#2}}
\newcommand{\ihval}[3]{\instthree{Hval}{#1}{#2}{#3}}
\newcommand{\iqval}[2]{\insttwo{Qval}{#1}{#2}}
\newcommand{\itext}[1]{\texttt{#1}}


\newcommand{\iassign}[2]{{#1}\leftarrow{#2}}
\newcommand{\iread}[1]{\texttt{read}{(#1)}}
\newcommand{\iwrite}[1]{\texttt{write}{(#1)}}
\newcommand{\imeas}[1]{\mathpzc{M}\texttt{(}#1\texttt{)}}
\newcommand{\ilet}[3]{\texttt{let }{#1}\texttt{ = }{#2}\texttt{ in }#3}
\newcommand{\ifstmt}[3]{\texttt{if }\texttt{(}{#1}\texttt{) }{#2}\texttt{ else }{#3}}
\newcommand{\irepeat}[2]{\texttt{repeat }{#1}\texttt{ }{#2}}
\newcommand{\inits}[2]{\texttt{init }{#1}\texttt{ }{#2}}
\newcommand{\ibuild}[2]{{#1}\leftarrow \texttt{build}{(#2)}}
\newcommand{\instr}{\iota}
\newcommand{\iskip}{\texttt{\{\}}}

\newcommand{\app}[3]{#2\texttt{[}{#3}\mapsto{#1}\texttt{]}}
\newcommand{\xsem}{\texttt{xg}}
\newcommand{\hsem}{\texttt{hc}}
\newcommand{\qsem}{\texttt{qt}}
\newcommand{\psem}{\texttt{pm}}
\newcommand{\rsem}{\texttt{rz}}
\newcommand{\rrsem}{\texttt{rrz}}
\newcommand{\cnotsem}{\texttt{flip}}
\newcommand{\csem}{\texttt{cu}}
\newcommand{\srsem}{\texttt{sr}}
\newcommand{\rssem}{\texttt{rs}}
\newcommand{\cn}[1]{\texttt{#1}}
\newcommand{\Omegasz}{\Sigma}
\newcommand{\Omegaty}{\Omega}
%% Coq documentation
%\usepackage{coqdoc}

% VPHL commands
\def\keyfont#1{\texttt{#1}}
\newcommand{\constfont}[1]{\mbox{\ensuremath{\mathtt{#1}}}}
\newcommand{\SKIP}{\keyfont{skip}\xspace}
\newcommand{\ASSN}{\keyfont{:=}}
\newcommand{\assign}[2]{\ensuremath{#1\; \ASSN\; #2}}
\newcommand{\IF}{\keyfont{if}\xspace}
\newcommand{\THEN}{\keyfont{then}\xspace}
\newcommand{\ELSE}{\keyfont{else}\xspace}
\newcommand{\ift}[3]{\ensuremath{\IF\ {#1} \ \THEN\ {#2} \ \ELSE\ {#3}}\xspace}
\newcommand{\END}{\keyfont{end}\xspace} 
\newcommand{\WHILE}{\keyfont{while}\xspace}
\newcommand{\DO}{\keyfont{do}\xspace}
\newcommand{\UNTIL}{\keyfont{until}\xspace}
\newcommand{\while}[2]{\WHILE\ {#1}\ \DO\ {#2}\ \xspace}
\newcommand{\IMP}{\emph{Imp}\xspace}
\newcommand{\PRIMP}{\emph{PrImp}\xspace}
\newcommand{\VPHL}{\emph{VPHL}\xspace}
\newcommand{\TOSS}{\keyfont{toss}\xspace}
\newcommand{\toss}[2]{\ensuremath{#1\; \ASSN \; \TOSS(#2)}\xspace}
\newcommand{\cond}[2]{\ensuremath{#1 \! \mid \! #2}\xspace}
\newcommand{\bicond}[3]{\ensuremath{#1 XYZ_{#2}^{#3}}\xspace}
\newcommand{\hoare}[3]{\ensuremath{\{#1\}\; #2 \; \{#3\}}\xspace}
\newcommand{\denote}[1]{\llbracket #1 \rrbracket\xspace}
%\newcommand{\pdenote}[1]{(\!| #1 |\!)}
\newcommand{\dabs}[1]{|\!| #1 |\!|}
\newcommand{\tos}[1]{(\!| #1 |\!)}
\newcommand{\tob}[1]{[\!\!( #1 )\!\!]}
\newcommand{\qfun}[2]{#1\langle #2 \rangle}
\newcommand{\tov}[1]{\{\hspace{-0.2em}| #1 |\hspace{-0.2em}\}}
\newcommand{\pdenote}[1]{\{\hspace{-0.2em}| #1 |\hspace{-0.2em}\}}
\newcommand{\dplus}[1]{\texttt{++}#1}
\newcommand{\dminus}[1]{\texttt{-\hspace{0.1em}-}#1}
\newcommand{\lift}[1]{\ensuremath{\lceil #1 \rceil}\xspace} 
\newcommand{\opsem}[3]{\ensuremath{#1 \; / \; #2 \Downarrow#3}\xspace}
\newcommand{\unit}[1]{\ensuremath{\keyfont{Unit} \; #1}\xspace} 
\newcommand{\prb}[2]{\ensuremath{Pr_{#2}(#1)}}
\newcommand{\true}[1]{\ensuremath{\lceil #1 \rceil}}
\newcommand{\TRUE}{\texttt{true}\xspace}
\newcommand{\FALSE}{\texttt{false}\xspace} 
\newcommand{\UNIT}{\ensuremath{\keyfont{Unit}}\xspace} %Changed for VPHL

% Quantum Hoare Commands
\newcommand{\smea}[3]{\texttt{let}\;{#1}\;\texttt{=}\;{\mathpzc{M}\cn{(}#2\cn{)}}\;\texttt{in}\;#3}
\newcommand{\bb}{\textbf{b}}
\newcommand{\bn}{\textbf{n}}
\newcommand{\qb}{\textbf{q}}
\newcommand{\qn}{\textbf{qn}}
\newcommand{\rand}[3][]{\ensuremath{#2 \; \oplus_{#1} \; #3}\xspace}
\newcommand{\lpw}{\ensuremath{\mathcal{L}_{pw}}\xspace}
%\newcommand{\bra}[1]{\ensuremath{\langle #1 |}\xspace}
%\newcommand{\ket}[1]{\ensuremath{| #1 \rangle }\xspace} % bra and ket are defined in qcircuit
%\newcommand{\braket}[2]{\ensuremath{\langle #1 | #2 \rangle}\xspace} % defined as ip in qcircuit
\newcommand{\qifss}[3]{\ensuremath{\texttt{qif}\ {#1} \ \THEN\ {#2} \ \ELSE\ {#3}}\xspace}
\newcommand{\case}[3]{\ensuremath{\texttt{case}\ #1 \ \rhd \ 0: {#2}, \dots, n-1: {#3}}\xspace}
\newcommand{\qcase}[3]{\ensuremath{\texttt{qcase}\ #1 \ \rhd \ 0: {#2}, \dots, n-1: {#3}}\xspace}
\newcommand{\masgn}[2]{\ensuremath{#1 \stackrel{m}{\texttt{:=}} #2}} % measure conflicts with qcircuit package
\newcommand{\bit}{\keyfont{bit }\xspace}
\newcommand{\qbit}{\keyfont{qbit }\xspace}
%\newcommand{\discard}[1]{\keyfont{discard } #1\xspace}
\newcommand{\MEASURE}{\ensuremath{\keyfont{measure}}\xspace}
\newcommand{\mif}[3]{\ensuremath{\MEASURE \ {#1} \ \THEN\ {#2} \ \ELSE\ {#3}}\xspace}
\newcommand{\timeseq}{\mathrel{{*}{=}}} % replace with unitary
\newcommand{\ds}[3]{\ensuremath{\denote{\langle #1 \rangle \ #2 \ \langle #3 \rangle}}\xspace}
\newcommand{\mat}[1]{\mathbf{#1}} %May not use
%\newcommand{\unitary}[2]{\ensuremath{\assign{#1}{\mat{#2} #1}}\xspace}
%\newcommand{\unitary}[2]{\ensuremath{#1 \timeseq #2}\xspace}
\newcommand{\mcase}[3]{\ensuremath{\texttt{measure} \ #1[#2] \ : \ #3}\xspace}
\newcommand{\mwhile}[3]{\ensuremath{\WHILE \ #1[#2] \ \DO \ #3}\xspace}
\newcommand{\lqtd}{\ensuremath{\mathcal{L}_{qTD}}\xspace}

\newcommand{\highlight}[1]{\textcolor{red}{#1}}

%% Underscore issues
%\usepackage{relsize}
%\renewcommand{\_}{\textscale{.7}{\textunderscore}}
% LaTeX says no...

%% Unicode
\usepackage{newunicodechar}
\let\Alpha=A
\let\Beta=B
\let\Epsilon=E
\let\Zeta=Z
\let\Eta=H
\let\Iota=I
\let\Kappa=K
\let\Mu=M
\let\Nu=N
\let\Omicron=O
\let\omicron=o
\let\Rho=P
\let\Tau=T
\let\Chi=X

\newunicodechar{Α}{\ensuremath{\Alpha}}
\newunicodechar{α}{\ensuremath{\alpha}}
\newunicodechar{Β}{\ensuremath{\Beta}}
\newunicodechar{β}{\ensuremath{\beta}}
\newunicodechar{Γ}{\ensuremath{\Gamma}}
\newunicodechar{γ}{\ensuremath{\gamma}}
\newunicodechar{Δ}{\ensuremath{\Delta}}
\newunicodechar{δ}{\ensuremath{\delta}}
\newunicodechar{Ε}{\ensuremath{\Epsilon}}
\newunicodechar{ε}{\ensuremath{\epsilon}}
\newunicodechar{ϵ}{\ensuremath{\varepsilon}}
\newunicodechar{Ζ}{\ensuremath{\Zeta}}
\newunicodechar{ζ}{\ensuremath{\zeta}}
\newunicodechar{Η}{\ensuremath{\Eta}}
\newunicodechar{η}{\ensuremath{\eta}}
\newunicodechar{Θ}{\ensuremath{\Theta}}
\newunicodechar{θ}{\ensuremath{\theta}}
\newunicodechar{ϑ}{\ensuremath{\vartheta}}
\newunicodechar{Ι}{\ensuremath{\Iota}}
\newunicodechar{ι}{\ensuremath{\iota}}
\newunicodechar{Κ}{\ensuremath{\Kappa}}
\newunicodechar{κ}{\ensuremath{\kappa}}
\newunicodechar{Λ}{\ensuremath{\Lambda}}
\newunicodechar{λ}{\ensuremath{\lambda}}
\newunicodechar{Μ}{\ensuremath{\Mu}}
\newunicodechar{μ}{\ensuremath{\mu}}
\newunicodechar{Ν}{\ensuremath{\Nu}}
\newunicodechar{ν}{\ensuremath{\nu}}
\newunicodechar{Ξ}{\ensuremath{\Xi}}
\newunicodechar{ξ}{\ensuremath{\xi}}
\newunicodechar{Ο}{\ensuremath{\Omicron}}
\newunicodechar{ο}{\ensuremath{\omicron}}
\newunicodechar{Π}{\ensuremath{\Pi}}
\newunicodechar{π}{\ensuremath{\pi}}
\newunicodechar{ϖ}{\ensuremath{\varpi}}
\newunicodechar{Ρ}{\ensuremath{\Rho}}
\newunicodechar{ρ}{\ensuremath{\rho}}
\newunicodechar{ϱ}{\ensuremath{\varrho}}
\newunicodechar{Σ}{\ensuremath{\Sigma}}
\newunicodechar{σ}{\ensuremath{\sigma}}
\newunicodechar{ς}{\ensuremath{\varsigma}}
\newunicodechar{Τ}{\ensuremath{\Tau}}
\newunicodechar{τ}{\ensuremath{\tau}}
\newunicodechar{Υ}{\ensuremath{\Upsilon}}
\newunicodechar{υ}{\ensuremath{\upsilon}}
\newunicodechar{Φ}{\ensuremath{\Phi}}
\newunicodechar{φ}{\ensuremath{\phi}}
\newunicodechar{ϕ}{\ensuremath{\varphi}}
\newunicodechar{Χ}{\ensuremath{\Chi}}
\newunicodechar{χ}{\ensuremath{\chi}}
\newunicodechar{Ψ}{\ensuremath{\Psi}}
\newunicodechar{ψ}{\ensuremath{\psi}}
\newunicodechar{Ω}{\ensuremath{\Omega}}
\newunicodechar{ω}{\ensuremath{\omega}}

\newunicodechar{ℕ}{\ensuremath{\mathbb{N}}}
\newunicodechar{∅}{\ensuremath{\emptyset}}

\newunicodechar{∙}{\ensuremath{\bullet}}
\newunicodechar{≈}{\ensuremath{\approx}}
\newunicodechar{≅}{\ensuremath{\cong}}
\newunicodechar{≡}{\ensuremath{\equiv}}
\newunicodechar{≤}{\ensuremath{\le}}
\newunicodechar{≥}{\ensuremath{\ge}}
\newunicodechar{≠}{\ensuremath{\neq}}
\newunicodechar{∀}{\ensuremath{\forall}}
\newunicodechar{∃}{\ensuremath{\exists}}
\newunicodechar{±}{\ensuremath{\pm}}
\newunicodechar{∓}{\ensuremath{\pm}}
\newunicodechar{·}{\ensuremath{\cdot}}
\newunicodechar{⋯}{\ensuremath{\cdots}}
\newunicodechar{…}{\ensuremath{\ldots}}
\newunicodechar{∷}{~\mathrel{:\!\!\!:}~}
\newunicodechar{×}{\ensuremath{\times}}
\newunicodechar{∞}{\ensuremath{\infty}}
\newunicodechar{→}{\ensuremath{\to}}
\newunicodechar{←}{\ensuremath{\leftarrow}}
\newunicodechar{⇒}{\ensuremath{\Rightarrow}}
\newunicodechar{↦}{\ensuremath{\mapsto}}
\newunicodechar{↝}{\ensuremath{\leadsto}}
\newunicodechar{∨}{\ensuremath{\vee}}
\newunicodechar{∧}{\ensuremath{\wedge}}
\newunicodechar{⊢}{\ensuremath{\vdash}}
\newunicodechar{⊣}{\ensuremath{\dashv}}
\newunicodechar{∣}{\ensuremath{\mid}}
\newunicodechar{∈}{\ensuremath{\in}}
\newunicodechar{⊆}{\ensuremath{\subseteq}}
\newunicodechar{⊂}{\ensuremath{\subset}}
\newunicodechar{∪}{\ensuremath{\cup}}
\newunicodechar{⋓}{\ensuremath{\Cup}}
\newunicodechar{∉}{\ensuremath{\not\in}}
\newunicodechar{√}{\ensuremath{\sqrt}}



\newunicodechar{⊸}{\ensuremath{\multimap}}
\newunicodechar{⊗}{\ensuremath{\otimes}}
\newunicodechar{⨂}{\ensuremath{\bigotimes}}
\newunicodechar{⊕}{\ensuremath{\oplus}}
\newunicodechar{〈}{\ensuremath{\langle}}
\newunicodechar{⟨}{\ensuremath{\langle}}
\newunicodechar{⟩}{\ensuremath{\rangle}}
\newunicodechar{〉}{\ensuremath{\rangle}}
\newunicodechar{¡}{\ensuremath{\upsidedownbang}}
\newunicodechar{∘}{\ensuremath{\circ}}
\newunicodechar{†}{\ensuremath{\dagger}}
\newunicodechar{⊤}{\ensuremath{\top}}
\newunicodechar{⊥}{\ensuremath{\bot}}

\newunicodechar{〚}{\ensuremath{\llbracket}}
\newunicodechar{〛}{\ensuremath{\rrbracket}}

%% COMMENTS 
\usepackage{etoolbox} % replaces ifthen package
\newtoggle{comments}
\toggletrue{comments}
%\togglefalse{comments}
  \usepackage[normalem]{ulem}
%  \usepackage{minted}

\iftoggle{comments}{
  \newcommand{\fixme}[1]{\textbf{\textcolor{red}{[ Fixme: #1]}}}
  \newcommand{\todo}[1]{\textbf{\textcolor{green}{[ TODO: #1 ]}}}
  \newcommand{\mwh}[1]{\textbf{\textcolor{red}{[ Mike: #1 ]}}}
  \newcommand{\yxp}[1]{\textbf{\textcolor{blue}{[ Yuxiang: #1 ]}}}
  \newcommand{\khh}[1]{\textbf{\textcolor{orange}{[ Kesha: #1 ]}}}
  \newcommand{\shh}[1]{\textbf{\textcolor{purple}{[ Shih-Han: #1 ]}}}
  \newcommand{\liyi}[1]{\textbf{\textcolor{blue}{[ Liyi: #1 ]}}}
\newcommand{\anshu}[1]{\textbf{\textcolor{olive}{[ Anshu: #1 ]}}}
  \newcommand{\oth}[2]{\textbf{\textcolor{red}{[ #1: #2 ]}}}
  \newcommand{\xwu}[1]{\textbf{\textcolor{purple}{[ Xiaodi: #1 ]}}}
  \newcommand{\fsdv}[1]{\textbf{\textcolor{pink}{[ Finn: #1 ]}}}
  \newcommand{\ynote}[1]{\textbf{\textcolor{magenta}{[ Yi: #1 ]}}}

  \colorlet{MZ}{violet!80!pink}
  \newcommand{\mz}[1]{{\textcolor{MZ}{\textbf{[[}(\(\mu\)) {\small{#1}}\textbf{]]}}}}
  \newcommand{\mzs}[1]{{\color{MZ}{\sout{#1}}}}%
  \newcommand{\mzu}[1]{{\color{MZ}\uline{#1}}}
  \newcommand{\mzr}[1]{{\color{MZ}{#1}}}
  \newcommand{\was}[1]{}
  \newcommand{\mzsub}[2]{\mzs{#1}\mzr{#2}}
  % trick to get around with sout
  \NewCommandCopy{\Creff}{\Cref}
  \renewcommand{\Cref}[1]{\mbox{\Creff{#1}}}

  \usepackage[inline]{enumitem}
  \colorlet{LC}{cyan!31!teal}

  \newcommand{\lc}[1]{{\color{LC}\textbf{\textit{Le: #1}}}}
  \newcommand{\lcs}[1]{{\color{LC} \sout{#1}}}
  \newcommand{\lcu}[1]{{\color{LC}\uline{#1}}}
  \newcommand{\lcr}[1]{{\color{LC}{#1}}}
  \usepackage[inline]{enumitem}
}{
  \newcommand{\fixme}[1]{}
  \newcommand{\todo}[1]{}
  \newcommand{\rnr}[1]{}
  \newcommand{\mwh}[1]{}  
  \newcommand{\khh}[1]{}
  \newcommand{\liyi}[1]{}
  \newcommand{\shh}[1]{}
  \newcommand{\xwu}[1]{}
  \newcommand{\oth}[2]{}
  \newcommand{\mzr}[1]{}

  \newcommand{\ynote}[1]{}
}

\newtoggle{submission}
\toggletrue{submission}
%\togglefalse{submission}

\iftoggle{submission}{
  \newcommand{\aref}[1]{\Cref{#1} of the extended version of this paper}
}{
  \newcommand{\aref}[1]{\Cref{#1}}
}


%% END COMMENTS

%%% Local Variables:
%%% mode: latex
%%% TeX-master: "main"
%%% End:
