%% For double-blind review submission, w/o CCS and ACM Reference (max submission space)
\documentclass[acmsmall, screen=true, review,anonymous,nonacm,pdftex,svgnames]{acmart}%\documentclass[sigplan,review,anonymous]{acmart}\settopmatter{printfolios=true}
%% For single-blind review submission, w/o CCS and ACM Reference (max submission space)
%\documentclass[sigplan,review]{acmart}\settopmatter{printfolios=true,printccs=false,printacmref=false}
%% For single-blind review submission, w/ CCS and ACM Reference
%\documentclass[sigplan,review]{acmart}\settopmatter{printfolios=true}
%% For final camera-ready submission, w/ required CCS and ACM Reference
%\documentclass[sigplan]{acmart}\settopmatter{}
%\nolinenumbers

%% Conference information
%% Supplied to authors by publisher for camera-ready submission;
%% use defaults for review submission.
\acmJournal{PACMPL}
\acmYear{2025} 
\acmVolume{6} 
\acmNumber{OOPSLA} 
%\acmArticle{146} 
%\acmMonth{10} 
\acmPrice{}
%\acmDOI{10.1145/3563309}

%\copyrightyear{2025}
%\acmSubmissionID{oopslab22main-p263-p}
\startPage{1}

%% Copyright information
%% Supplied to authors (based on authors' rights management selection;
%% see authors.acm.org) by publisher for camera-ready submission;
%% use 'none' for review submission.
%\setcopyright{none}
%\setcopyright{acmcopyright}
%\setcopyright{acmlicensed}
%\setcopyright{rightsretained}
%\copyrightyear{2018}           %% If different from \acmYear

%% Bibliography style
%\citestyle{acmauthoryear}  %% For author/year citations
%\citestyle{acmnumeric}     %% For numeric citations
%\setcitestyle{nosort}      %% With 'acmnumeric', to disable automatic
                            %% sorting of references within a single citation;
                            %% e.g., \cite{Smith99,Carpenter05,Baker12}
                            %% rendered as [14,5,2] rather than [2,5,14].
%\setcitesyle{nocompress}   %% With 'acmnumeric', to disable automatic
                            %% compression of sequential references within a
                            %% single citation;
                            %% e.g., \cite{Baker12,Baker14,Baker16}
                            %% rendered as [2,3,4] rather than [2-4].

%\citestyle{acmnumeric}   %% For author/year citations
%%%%%%%%%%%%%%%%%%%%%%%%%%%%%%%%%%%%%%%%%%%%%%%%%%%%%%%%%%%%%%%%%%%%%%
%% Note: Authors migrating a paper from traditional SIGPLAN
%% proceedings format to PACMPL format must update the
%% '\documentclass' and topmatter commands above; see
%% 'acmart-pacmpl-template.tex'.
%%%%%%%%%%%%%%%%%%%%%%%%%%%%%%%%%%%%%%%%%%%%%%%%%%%%%%%%%%%%%%%%%%%%%%
\bibliographystyle{ACM-Reference-Format}
%% Citation style
%\citestyle{acmauthoryear}  %% For author/year citations
%\citestyle{acmnumeric}     %% For numeric citations
%\setcitestyle{nosort}      %% With 'acmnumeric', to disable automatic
                            %% sorting of references within a single citation;
                            %% e.g., \cite{Smith99,Carpenter05,Baker12}
                            %% rendered as [14,5,2] rather than [2,5,14].
%\setcitesyle{nocompress}   %% With 'acmnumeric', to disable automatic
                            %% compression of sequential references within a
                            %% single citation;
                            %% e.g., \cite{Baker12,Baker14,Baker16}
                            %% rendered as [2,3,4] rather than [2-4].

\citestyle{acmauthoryear}   %% For author/year citations

%\usepackage{MnSymbol}
%% Some recommended packages.
\usepackage{booktabs}   %% For formal tables:
                        %% http://ctan.org/pkg/booktabs
\usepackage{subcaption} %% For complex figures with subfigures/subcaptions
                        %% http://ctan.org/pkg/subcaption
\usepackage{bussproofs}
\usepackage[cal=boondoxo]{mathalfa}
\DeclareMathAlphabet{\mathpzc}{OT1}{pzc}{m}{it}
\usepackage{amsmath}
%\newtheorem{theorem}{Theorem}[section]
%\newtheorem{observation}[theorem]{Observation}
\usepackage{color}
\usepackage{xspace}
\input{macros}
%\usepackage{bm}

\usepackage{xcolor}
\usepackage{colortbl}

\usepackage[utf8]{inputenc}

\usepackage{amsmath}
\usepackage{stmaryrd}
\usepackage{hyperref}

\usepackage{subcaption}
%\usepackage{subfig}
\usepackage{wrapfig}

\usepackage{xspace}
\usepackage{float}
\usepackage{balance}

\usepackage{textcomp} %just for a vertical quote

\usepackage{changepage}
\usepackage{array,etoolbox}

\preto\tabular{\setcounter{magicrownumbers}{0}}
\newcounter{magicrownumbers}
\newcommand\rownumber{\stepcounter{magicrownumbers}\arabic{magicrownumbers}}


% References
%\usepackage[capitalize, noabbrev]{cleveref} % must be loaded after hyperref and amsmath
\usepackage{cleveref} % must be loaded after hyperref and amsmath

% Figures
\usepackage{wrapfig}
\usepackage{caption}

% Tables
\usepackage{longtable}
\usepackage{tabu}

% Lists
%\usepackage[inline,shortlabels]{enumitem}
\usepackage{wasysym}
% Proof trees
\usepackage{mathpartir}
%\usepackage{bussproofs}

% Symbolx
\usepackage{mathtools} %psmallmatrix
%\usepackage{stmaryrd} % more math symbols
\usepackage{xfrac} % fractions

% Enumeration
% \usepackage{enumitem} % resume option

% Quantum
\usepackage[qm,braket]{qcircuit}
%\usepackage{braket}
\newcommand{\ketbra}[2]{\ket{#1}\!\bra{#2}}

\hyphenation{Comp-Cert}

% sessions
\newcommand{\stuple}[3]{{#1}[#2, #3)}

%Language names
\newcommand{\qafnyl}{$\textsc{L}_\textsc{Qafny}$\xspace}
\newcommand{\qass}[2]{{#1}\;{\leftarrow}\;{#2}}
\newcommand{\qfor}[4]{\texttt{{for} }{#1}\texttt{ \(\in\) }[{#2},{#3})\texttt{ {\&\&} }{#4}}
\newcommand{\qif}[2]{\texttt{{if}\,(}{#1}\texttt{)\,\{}{#2}\texttt{\}}}
\newcommand{\sifq}[2]{\texttt{if}~\cn{(}{#1}\cn{)}~{#2}}
\newcommand{\modmult}[3]{(#1 * #2)\cn{\,\%} \, #3}
\newcommand{\modexp}[3]{{#1}^{#2}\,\cn{\%} \, #3}
\newcommand{\qbool}[4]{\cn{(}#1\,\cn{#2}\,#3\cn{)\,\cn{@}\,}#4}
\newcommand{\qboola}[3]{#1\,\cn{#2}\,#3}
\newcommand{\mmod}{\cn{\%}}
\newcommand{\Hassert}[1]{\{\ #1\ \}}
\colorlet{kwd}{black!80!green}
\definecolor{spec1}{RGB}{78, 131, 162}
\definecolor{spec0}{RGB}{66, 102, 136}
\definecolor{lespec}{RGB}{30, 80, 180}
\colorlet{spec}{lespec}
\colorlet{auto}{lespec!35!lightgray}
\colorlet{stack}{magenta}

\newcommand{\qafny}{\rulelab{Qafny}\xspace}
\newcommand{\sep}{\rulelab{Sep}\xspace}
%\newcommand{\qafnyl}{\textsc{QafnyL}\xspace}
\newcommand{\name}{\textsc{VQO}\xspace}
\newcommand{\qvm}{\textsc{qvm}\xspace}
\newcommand{\sourcelang}{\ensuremath{\mathcal{O}\textsc{qimp}}\xspace}
%\newcommand{\pqasm}{\textsc{oqasm}\xspace}
\newcommand{\oqasm}{\textsc{Oqasm}\xspace}
\newcommand{\preq}{\textsc{Preq}\xspace}
\newcommand{\pqasm}{\ensuremath{\textsc{Pqasm}}\xspace}
\newcommand{\intlang}{\oqasm}
\newcommand{\vqimp}{\sourcelang}
\newcommand{\vqir}{\intlang}
\newcommand{\sqir}{SQIR\xspace}
\newcommand{\coqq}{{CoqQ}\xspace}
\newcommand{\qwire}{{Qwire}\xspace}
\newcommand{\qbricks}{{QBricks}\xspace}
\newcommand{\liquids}{LIQ\emph{Ui}$\ket{}$\xspace}
\newcommand{\liquid}{Liquid\xspace}
\newcommand{\revs}{R\textsc{evs}\xspace}
\newcommand{\reverc}{R\textsc{e}V\textsc{er}C\xspace}
\newcommand{\fstar}{F${}^\star$\xspace}
\newcommand{\voqc}{\textsc{VOQC}\xspace}
\newcommand{\tket}{t$\vert$ket$\rangle$\xspace}
\newcommand{\myparagraph}[1]{\noindent\paragraph{\textbf{#1}}}

%quantum tikz macros
% These are useful if not using qcircuit (which kind of sucks, alternatives do exist
\usepackage{tikz}
\newcommand{\mycontrol}[2]{\draw[fill=black] (#1,#2) circle [radius=0.10];}
\newcommand{\qnot}[2]{\draw (#1,#2) circle [radius=0.20]; \draw (#1,#2-0.20) -- (#1,#2+0.20);}
\newcommand{\cnot}[3]{\qnot{#1}{#3}\mycontrol{#1}{#2}\draw (#1,#2) -- (#1,#3);}
\newcommand{\tof}[4]{\qnot{#1}{#4}\mycontrol{#1}{#2}\mycontrol{#1}{#3}\draw (#1,#2) -- (#1,#4);\draw (#1,#3) -- (#1,#4);}
\newcommand{\unitary}[3]{\draw[fill=white] (#2-0.4,#3-0.4) rectangle node {\texttt{#1}} (#2+0.4,#3+0.4);}
\newcommand{\had}[2]{\unitary{H}{#1}{#2}}
\newcommand{\meas}[2]{\draw[fill=white] (#1-0.8,#2-0.4) rectangle node {\texttt{meas}} (#1+0.8,#2+0.4);}
\newcommand{\discard}[2]{\draw[thick](#1,#2-0.2) -- (#1,#2+0.2);}

% From POPL2017
\newcommand{\aket}[2]{\ket{#1}_{\textcolor{spec}{#2}}}
\newcommand{\qket}[1]{\ket{\Delta(#1)}}

\newcommand{\shows}{\ensuremath{\vdash}}
\newcommand{\mprod}{\mathbin{\text{\footnotesize \ensuremath{\otimes}}}}
\newcommand{\aprod}{\mathbin{\text{\footnotesize \ensuremath{\&}}}}
\newcommand{\msum}{\mathbin{\text{\footnotesize \ensuremath{\parr}}}}
\newcommand{\asum}{\mathbin{\text{\footnotesize \ensuremath{\oplus}}}}
\newcommand{\lolto}{\ensuremath{\multimap}}
\newcommand{\bang}{\ensuremath{\oc}}
\newcommand{\whynot}{\ensuremath{\wn}}
\newcommand{\one}{\ensuremath{1}}
\newcommand{\zero}{\ensuremath{0}}
\newcommand{\interp}[1]{[ #1 ]}
\newcommand{\interpalt}[1]{\llbracket #1 \rrbracket}
\newcommand{\ecode}[1]{\emph{\texttt{#1}}}
\newcommand{\cmode}{\texttt{C}}
\newcommand{\mmode}{\texttt{M}}
\newcommand{\qmodename}{\texttt{Q}}
\newcommand{\qmode}[1]{\texttt{Q}~#1}
\newcommand{\topt}[1]{#1\;\texttt{opt}}
\newcommand{\slen}[1]{|#1|}
% From QPL2017
\newenvironment{nscenter}
 {\parskip=3pt\par\nopagebreak\centering}
 {\par\noindent\ignorespacesafterend}

% From CoqPL2018
\newcommand{\dimx}[1]{\text{dim}_x(#1)}
\newcommand{\dimy}[1]{\text{dim}_y(#1)}

% From QPL2018

% Standard mathematical definitions
% Field
\newcommand{\F}{\ensuremath{\mathbb{F}}\xspace}
% Integers
\newcommand{\Z}{\ensuremath{\mathbb{Z}}\xspace}
% Naturals
\newcommand{\N}{\ensuremath{\mathbb{N}}\xspace}
% Rationals
\newcommand{\Q}{\ensuremath{\mathbb{Q}}\xspace}
% Reals
\newcommand{\R}{\ensuremath{\mathbb{R}}\xspace}
% Complex
\newcommand{\CC}{\ensuremath{\mathbb{C}}\xspace}
% \newcommand{\C}{\ensuremath{\mathbb{C}}\xspace}

%   Tikz
\usepackage{pgf}

\usepackage{tikz} % circuit diagrams
\usetikzlibrary{%
  arrows,%
  shapes.misc,% wg. rounded rectangle
  shapes.geometric, % diamonds
  shapes.arrows,%
  shapes.callouts,
  shapes.gates.logic.US,
  chains,%
  matrix,%
  positioning,% wg. " of "
  scopes,%
  decorations.pathmorphing,% /pgf/decoration/random steps | erste Graphik
  decorations.text,
  decorations.pathreplacing, % braces
  shadows,%
  automata,
  fit, calc, arrows.meta
}

\tikzset{ machine/.style={
    % The shape:
    rectangle,
    % The size:
    minimum width=25mm,
    minimum height=18mm,
    text width=24mm,
    % The alignment
    align=center,
    % The border:
    very thick,
    draw=black,
    % The colors:
    color=black,
    fill=white,
    % Font
%    font=\ttfamily,
  }
}

% Place two figures side-by-side, possibly overlapping
\newenvironment{pairfigures}[4]{%
\newcommand{\foot}{#4} % yay, clever hackery!
  \let\and&
  \begin{center}%
  \vspace{#3\baselineskip}%
  \begin{tabular}{m{#1}m{#2}}%
}{%
  \end{tabular}
  \vspace{\foot\baselineskip}
  \end{center}
}

\newcommand\wideparen[1]{%
\tikz[baseline=(wideArcAnchor.base)]{
    \node[inner sep=0] (wideArcAnchor) {$#1$}; 
    \coordinate (wideArcAnchorA) at ($0.9*(wideArcAnchor.north west) + 0.1*(wideArcAnchor.north east)+(0.0em,0.75ex)$);
    \coordinate (wideArcAnchorB) at ($0.1*(wideArcAnchor.north west) + 0.9*(wideArcAnchor.north east)+(0.0em,0.75ex)$);
%   
    \draw[line width=0.1ex,line cap=round] 
        ($(wideArcAnchor.north west)+(0.0em,0.1ex)$) 
            .. controls (wideArcAnchorA) and (wideArcAnchorB) ..
        ($(wideArcAnchor.north east)+(0.0em,0.1ex)$)        
    ;
}}
\newcommand{\cmsg}[1]{\wideparen{#1}}
\newcommand{\function}[2]{#1~#2}
\newcommand*\rfrac[2]{{}^{#1}\!/_{#2}}
\DeclarePairedDelimiter\abs{\lvert}{\rvert}
\DeclarePairedDelimiter\norm{\lVert}{\rVert}

\makeatletter
\let\oldabs\abs
\def\abs{\@ifstar{\oldabs}{\oldabs*}}
%
\let\oldnorm\norm
\def\norm{\@ifstar{\oldnorm}{\oldnorm*}}
\makeatother

%\DeclarePairedDelimiter{\inpar}[2]{}{}{#1\;\delimsize\|\;#2}
\newcommand{\inpar}{\mathbin{\|}}

% Proper division symbol
\makeatletter
\DeclareRobustCommand{\vardivision}{%
  \mathbin{\mathpalette\@vardivision\relax}% 
}
\newcommand{\@vardivision}[2]{%
  \reflectbox{$\m@th\smallsetminus$}%
}
\makeatother

\DeclareFontFamily{U} {MnSymbolC}{}
\DeclareFontShape{U}{MnSymbolC}{m}{n}{
  <-6> MnSymbolC5
  <6-7> MnSymbolC6
  <7-8> MnSymbolC7
  <8-9> MnSymbolC8
  <9-10> MnSymbolC9
  <10-12> MnSymbolC10
  <12-> MnSymbolC12}{}
\DeclareFontShape{U}{MnSymbolC}{b}{n}{
  <-6> MnSymbolC-Bold5
  <6-7> MnSymbolC-Bold6
  <7-8> MnSymbolC-Bold7
  <8-9> MnSymbolC-Bold8
  <9-10> MnSymbolC-Bold9
  <10-12> MnSymbolC-Bold10
  <12-> MnSymbolC-Bold12}{}

\DeclareSymbolFont{MnSyC} {U} {MnSymbolC}{m}{n}

\DeclareMathSymbol{\sqcupplus}{\mathbin}{MnSyC}{70}

%\DeclareRobustCommand{\sqcupplus}{\mbox{\usefont{U}{MnSymbolC}{m}{n}\char70}}

% Latin  Abbr
\newcommand{\etal}{\emph{et al.}\xspace}
\newcommand{\eg}{\emph{e.g.,}\xspace}
\newcommand{\ie}{\emph{i.e.,}\xspace}
\newcommand{\etc}{\emph{etc.}\xspace}

% Math commands
\DeclareMathOperator{\tr}{tr}



% Table formatting
\newcommand{\ltwo}[1]{\multicolumn{2}{l}{#1}}
\newcommand{\ctwo}[1]{\multicolumn{2}{c}{#1}}
\usepackage{booktabs}

% Types
\newcommand{\One}{\code{One}}
\newcommand{\Bit}{\code{Bit}}
\newcommand{\Qubit}{\code{Qubit}}
% Qwire Syntax
\newcommand{\Gate}[2]{\code{Gate}(#1,#2)}
\newcommand{\Unitary}[2]{\code{Unitary}(#1,#2)}
\renewcommand{\Box}[2]{\code{Box}(#1,#2)}
\newcommand{\mkbox}[2]{\code{box}(#1 \Rightarrow #2)}
\newcommand{\qcontrol}[1]{\code{control}~#1}  % \control conflicts with qcircuit package
\newcommand{\bitcontrol}[1]{\code{bit-control}~#1}

% Code
\newcommand{\None}{\code{None}}
\newcommand{\Some}[1]{\code{Some}~#1}

% Listings
\usepackage{alltt}
\usepackage{listings,lstcoq}
%\usepackage{MnSymbol}
\definecolor{ltblue}{rgb}{0,0.4,0.4}
\definecolor{dkblue}{rgb}{0,0.1,0.6}
\definecolor{dkgreen}{rgb}{0,0.35,0}
\definecolor{dkviolet}{rgb}{0.3,0,0.5}
\definecolor{dkred}{rgb}{0.5,0,0}
\lstset{language=Coq}
\usepackage[export]{adjustbox}

\newcommand{\code}[1]{{\small\texttt{#1}}}
%\newcommand{\mcode}[1]{{\small\mathtt{#1}}}
%\newcommand{\code}[1]{\lstinline{#1}}

% Preventing pagebreaks
\newenvironment{absolutelynopagebreak}
  {\par\nobreak\vfil\penalty0\vfilneg
   \vtop\bgroup}
  {\par\xdef\tpd{\the\prevdepth}\egroup
   \prevdepth=\tpd}

 %% VQIMP syntax, semantic values
 \newcommand{\rulelab}[1]{{\small \textsc{#1}}}
\newcommand{\steps}{\ensuremath{\longrightarrow}}
\newcommand{\tbool}{\texttt{bool}}
\newcommand{\tsizeof}{\texttt{sizeof}}
\newcommand{\vbool}{\texttt{Bool}}
\newcommand{\verror}{\texttt{Error}}
\newcommand{\vval}{\mathpzc{val}}
\newcommand{\tfixed}{\texttt{fixedp}}
\newcommand{\tnat}{\texttt{nat}}
\newcommand{\tarr}[2]{\texttt{array}~{#1}~{#2}}
\newcommand{\econst}[2]{({#1}){#2}}
\newcommand{\eindex}[2]{{#1}\texttt{[}{#2}\texttt{]}}

\newcommand{\sexp}[3]{\texttt{let}~{#1}\,\texttt{=}\,{#2}~\texttt{in}~{#3}}
\newcommand{\sexph}[2]{\texttt{let}~{#1}\,\texttt{=}\,{#2}~\texttt{in}}
\newcommand{\sskip}{\texttt{\{\}}}

\newcommand{\sifb}[3]{\texttt{if}~{(#1)}~{#2}~\texttt{else}~{#3}}
%\newcommand{\swhile}[3]{\texttt{while}~{(#1)}~{#2}~{\{#3\}}}
\newcommand{\sqwhile}[5]{\texttt{for}~{#1 \in [#2,#3)~\texttt{\&\&}~#4}~#5}
\newcommand{\sqfora}[4]{\texttt{for}~{#1 \in [#2,#3)~\texttt{\&\&}~#4}}
\newcommand{\sqforh}[4]{\texttt{for}~{#1;~#2~\texttt{\&\&}~#3~;~#4}}
\newcommand{\sinit}[1]{\texttt{init}~{#1}}
\newcommand{\sint}[2]{\texttt{int}~{#1}\texttt{:=}\,{#2}}
\newcommand{\sincr}[1]{{#1}\texttt{++}}
\newcommand{\tnor}[1]{\texttt{Nor}({#1})}
\newcommand{\tnort}{\texttt{Nor}}
\newcommand{\trot}[1]{\texttt{Rot}({#1})}
\newcommand{\trott}{\texttt{Rot}}
%\newcommand{\thad}[3]{\texttt{Had}({#1},#2,#3)}
\newcommand{\thad}[1]{\texttt{Had}(#1)}
\newcommand{\thadt}{\texttt{Had}}
\newcommand{\tch}[3]{(#1,#2,#3)}
\newcommand{\shad}[3]{\frac{1}{\sqrt{#1}}\Motimes_{j=0}^{#2}{(\ket{0}+#3\ket{1})}}
\newcommand{\shadi}[3]{\frac{1}{\sqrt{#1}}\Motimes_{i=0}^{#2}{(\ket{0}+#3\ket{1})}}
\newcommand{\ssum}[3]{\Msum_{#1}^{#2}{#3}}
\newcommand{\sch}[3]{\Msum_{j=0}^{#1}{#2 {#3}}}
\newcommand{\scha}[4]{\Msum_{j=0}^{#1}{#2\ket{#3}#4}}
\newcommand{\schai}[4]{\Msum_{i=0}^{#1}{#2\ket{#3}#4}}
\newcommand{\schb}[2]{\Msum_{j=0}^{#1}{q(#2)}}
\newcommand{\paren}[1]{\left(#1\right)}
\newcommand{\typing}[5]{{#1},{#2}\vdash_{#3} {#4} \triangleright{#5}}
\newcommand{\smch}[3]{#1\Msum_{k=0}^{#2}{\ket{#3}}}
\newcommand{\smich}[4]{#1\Msum_{k=0}^{#2}{#3\ket{#4}}}
\newcommand{\schk}[3]{\Msum_{k=0}^{#1}{#2\ket{#3}}}
\newcommand{\schii}[3]{\Msum_{i=0}^{#1}{#2\ket{#3}}}
\newcommand{\schd}[4]{\Msum_{#1=0}^{#2}{#3\ket{#4}}}
\newcommand{\schia}[3]{\Msum_{i=0}^{#1}{#2#3}}
\newcommand{\schi}[3]{\Msum^{#1}{#2\ket{#3}}}
\newcommand{\tpower}[1]{\mathcal{P}(#1)}
\newcommand{\iev}[1]{\texttt{ev}(#1)}
\newcommand{\itau}{\mathcal{T}}
\newcommand{\snat}[1]{\texttt{nat}({#1})}
\newcommand{\sord}[1]{\texttt{ord}({#1})}
\newcommand{\srnd}[1]{\texttt{rnd}({#1})}
\DeclareMathOperator*{\Motimes}{\text{\raisebox{0.25ex}{\scalebox{0.8}{$\bigotimes$}}}}
\DeclareMathOperator*{\Msum}{\text{\raisebox{0.25ex}{\scalebox{0.8}{$\sum$}}}}
\newcommand{\frz}{\mathpzc{F}}
\newcommand{\mea}{\mathpzc{M}}
\newcommand{\ufrz}{\mathpzc{U}}
\newcommand{\ttype}[2]{\Motimes_{#1}~{#2}}
\newcommand{\cupdot}{\mathbin{\mathaccent\cdot\cup}}
\newcommand{\ias}[1]{\texttt{as}(#1)}
\newcommand{\iasa}[2]{\texttt{as}^{#1}(#2)}
\newcommand{\ibs}[1]{\texttt{bs}(#1)}
\newcommand{\ips}[1]{\texttt{ps}(#1)}
\newcommand{\fivepule}[6]{#1;#2\vdash_{#3}\big{\{}#4\big{\}}\, #5 \,\big{\{}#6\big{\}}}

\newcommand{\triple}[3]{\big{\{}#1\big{\}}\, #2 \,\big{\{}#3\big{\}}}

\newcommand{\sassign}[4]{{#1} \leftarrow {#3}~{#2}~{#4}}
\newcommand{\ssassign}[3]{{#1} \xleftarrow{#2} {#3}}
\newcommand{\sand}[2]{{#1}~{\&\&}~{#2}}
\newcommand{\samp}[1]{\texttt{amp}(#1)}
\newcommand{\sreflect}[1]{\texttt{reflect}\{#1\}}
\newcommand{\sdis}{\texttt{dis}}
\newcommand{\sreduce}[2]{\texttt{reduce}(#1,#2)}
\newcommand{\samplify}[1]{\texttt{amplify}\{#1\}}
\newcommand{\sdistr}[1]{\texttt{diffuse}(#1)}
\newcommand{\sret}[1]{\texttt{ret}(#1)}
\newcommand{\size}[1]{\texttt{size}(#1)}
\newcommand{\snext}[1]{#1\splus 1}
\newcommand{\sminus}{\texttt{-}}
\newcommand{\splus}{\texttt{+}}
\newcommand{\slt}{\texttt{<}}
\newcommand{\srange}[2]{[#1,#2)}
\newcommand{\sfor}[3]{\texttt{for}~{#1}~{#2}~{#3}}
\newcommand{\scall}[3]{{#1}\leftarrow {#2}~{#3}}
\newcommand{\sseq}[2]{{#1}\,\texttt{;}\,{#2}}
\newcommand{\sinv}[1]{\texttt{inv}~{#1}}
\newcommand{\inst}[3][ ]{\texttt{#2}^{#1}~{#3}}
\newcommand{\insttwo}[4][ ]{\texttt{#2}^{#1}~{#3}~{#4}}
\newcommand{\instthree}[5][ ]{\texttt{#2}^{#1}~{#3}~{#4}~{#5}}
\newcommand{\iu}[1]{\inst{U}{#1}}
\newcommand{\inot}[1]{\inst{X}{#1}}
\newcommand{\ictrl}[2]{\insttwo{CU}{#1}{#2}}
\newcommand{\iadd}[2]{\cn{add}(#1,#2)}
\newcommand{\irz}[3][ ]{\insttwo[#1]{RZ}{#2}{#3}}
\newcommand{\isr}[3][ ]{\insttwo[#1]{SR}{#2}{#3}}
\newcommand{\icnot}[2]{\insttwo{CNOT}{#1}{#2}}
\newcommand{\ilshift}[1]{\inst{Lshift}{#1}}
\newcommand{\irshift}[1]{\inst{Rshift}{#1}}
\newcommand{\irev}[1]{\inst{Rev}{#1}}
\newcommand{\iqft}[3][ ]{\insttwo[#1]{QFT}{#2}{#3}}
\newcommand{\iqfth}[1][ ]{\texttt{QFT}^[#1]}
\newcommand{\tphi}[1]{\texttt{Phi}~{#1}}
%\newcommand{\thad}[1]{\texttt{Had}~{#1}}
\newcommand{\ihad}[1]{\texttt{H}(#1)}
\newcommand{\inew}[1]{\texttt{new}(#1)}
\newcommand{\iry}[2]{\texttt{Ry}^{#1}{#2}}
\newcommand{\ihadh}{\texttt{H}}
\newcommand{\isp}[2]{\mathcal{#1}_{#2}}
\newcommand{\hsp}[1]{\mathcal{#1}}
\newcommand{\iseq}[2]{{#1}\,\texttt{;}\,{#2}}
\newcommand{\inval}[2]{\insttwo{Nval}{#1}{#2}}
\newcommand{\ihval}[3]{\instthree{Hval}{#1}{#2}{#3}}
\newcommand{\iqval}[2]{\insttwo{Qval}{#1}{#2}}
\newcommand{\itext}[1]{\texttt{#1}}


\newcommand{\iassign}[2]{{#1}\leftarrow{#2}}
\newcommand{\iread}[1]{\texttt{read}{(#1)}}
\newcommand{\iwrite}[1]{\texttt{write}{(#1)}}
\newcommand{\imeas}[1]{\mathpzc{M}\texttt{(}#1\texttt{)}}
\newcommand{\ilet}[3]{\texttt{let }{#1}\texttt{ = }{#2}\texttt{ in }#3}
\newcommand{\ifstmt}[3]{\texttt{if }\texttt{(}{#1}\texttt{) }{#2}\texttt{ else }{#3}}
\newcommand{\irepeat}[2]{\texttt{repeat }{#1}\texttt{ }{#2}}
\newcommand{\inits}[2]{\texttt{init }{#1}\texttt{ }{#2}}
\newcommand{\ibuild}[2]{{#1}\leftarrow \texttt{build}{(#2)}}
\newcommand{\instr}{\iota}
\newcommand{\iskip}{\texttt{\{\}}}

\newcommand{\app}[3]{#2\texttt{[}{#3}\mapsto{#1}\texttt{]}}
\newcommand{\xsem}{\texttt{xg}}
\newcommand{\hsem}{\texttt{hc}}
\newcommand{\qsem}{\texttt{qt}}
\newcommand{\psem}{\texttt{pm}}
\newcommand{\rsem}{\texttt{rz}}
\newcommand{\rrsem}{\texttt{rrz}}
\newcommand{\cnotsem}{\texttt{flip}}
\newcommand{\csem}{\texttt{cu}}
\newcommand{\srsem}{\texttt{sr}}
\newcommand{\rssem}{\texttt{rs}}
\newcommand{\cn}[1]{\texttt{#1}}
\newcommand{\Omegasz}{\Sigma}
\newcommand{\Omegaty}{\Omega}
%% Coq documentation
%\usepackage{coqdoc}

% VPHL commands
\def\keyfont#1{\texttt{#1}}
\newcommand{\constfont}[1]{\mbox{\ensuremath{\mathtt{#1}}}}
\newcommand{\SKIP}{\keyfont{skip}\xspace}
\newcommand{\ASSN}{\keyfont{:=}}
\newcommand{\assign}[2]{\ensuremath{#1\; \ASSN\; #2}}
\newcommand{\IF}{\keyfont{if}\xspace}
\newcommand{\THEN}{\keyfont{then}\xspace}
\newcommand{\ELSE}{\keyfont{else}\xspace}
\newcommand{\ift}[3]{\ensuremath{\IF\ {#1} \ \THEN\ {#2} \ \ELSE\ {#3}}\xspace}
\newcommand{\END}{\keyfont{end}\xspace} 
\newcommand{\WHILE}{\keyfont{while}\xspace}
\newcommand{\DO}{\keyfont{do}\xspace}
\newcommand{\UNTIL}{\keyfont{until}\xspace}
\newcommand{\while}[2]{\WHILE\ {#1}\ \DO\ {#2}\ \xspace}
\newcommand{\IMP}{\emph{Imp}\xspace}
\newcommand{\PRIMP}{\emph{PrImp}\xspace}
\newcommand{\VPHL}{\emph{VPHL}\xspace}
\newcommand{\TOSS}{\keyfont{toss}\xspace}
\newcommand{\toss}[2]{\ensuremath{#1\; \ASSN \; \TOSS(#2)}\xspace}
\newcommand{\cond}[2]{\ensuremath{#1 \! \mid \! #2}\xspace}
\newcommand{\bicond}[3]{\ensuremath{#1 XYZ_{#2}^{#3}}\xspace}
\newcommand{\hoare}[3]{\ensuremath{\{#1\}\; #2 \; \{#3\}}\xspace}
\newcommand{\denote}[1]{\llbracket #1 \rrbracket\xspace}
%\newcommand{\pdenote}[1]{(\!| #1 |\!)}
\newcommand{\dabs}[1]{|\!| #1 |\!|}
\newcommand{\tos}[1]{(\!| #1 |\!)}
\newcommand{\tob}[1]{[\!\!( #1 )\!\!]}
\newcommand{\qfun}[2]{#1\langle #2 \rangle}
\newcommand{\tov}[1]{\{\hspace{-0.2em}| #1 |\hspace{-0.2em}\}}
\newcommand{\pdenote}[1]{\{\hspace{-0.2em}| #1 |\hspace{-0.2em}\}}
\newcommand{\dplus}[1]{\texttt{++}#1}
\newcommand{\dminus}[1]{\texttt{-\hspace{0.1em}-}#1}
\newcommand{\lift}[1]{\ensuremath{\lceil #1 \rceil}\xspace} 
\newcommand{\opsem}[3]{\ensuremath{#1 \; / \; #2 \Downarrow#3}\xspace}
\newcommand{\unit}[1]{\ensuremath{\keyfont{Unit} \; #1}\xspace} 
\newcommand{\prb}[2]{\ensuremath{Pr_{#2}(#1)}}
\newcommand{\true}[1]{\ensuremath{\lceil #1 \rceil}}
\newcommand{\TRUE}{\texttt{true}\xspace}
\newcommand{\FALSE}{\texttt{false}\xspace} 
\newcommand{\UNIT}{\ensuremath{\keyfont{Unit}}\xspace} %Changed for VPHL

% Quantum Hoare Commands
\newcommand{\smea}[3]{\texttt{let}\;{#1}\;\texttt{=}\;{\mathpzc{M}\cn{(}#2\cn{)}}\;\texttt{in}\;#3}
\newcommand{\bb}{\textbf{b}}
\newcommand{\bn}{\textbf{n}}
\newcommand{\qb}{\textbf{q}}
\newcommand{\qn}{\textbf{qn}}
\newcommand{\rand}[3][]{\ensuremath{#2 \; \oplus_{#1} \; #3}\xspace}
\newcommand{\lpw}{\ensuremath{\mathcal{L}_{pw}}\xspace}
%\newcommand{\bra}[1]{\ensuremath{\langle #1 |}\xspace}
%\newcommand{\ket}[1]{\ensuremath{| #1 \rangle }\xspace} % bra and ket are defined in qcircuit
%\newcommand{\braket}[2]{\ensuremath{\langle #1 | #2 \rangle}\xspace} % defined as ip in qcircuit
\newcommand{\qifss}[3]{\ensuremath{\texttt{qif}\ {#1} \ \THEN\ {#2} \ \ELSE\ {#3}}\xspace}
\newcommand{\case}[3]{\ensuremath{\texttt{case}\ #1 \ \rhd \ 0: {#2}, \dots, n-1: {#3}}\xspace}
\newcommand{\qcase}[3]{\ensuremath{\texttt{qcase}\ #1 \ \rhd \ 0: {#2}, \dots, n-1: {#3}}\xspace}
\newcommand{\masgn}[2]{\ensuremath{#1 \stackrel{m}{\texttt{:=}} #2}} % measure conflicts with qcircuit package
\newcommand{\bit}{\keyfont{bit }\xspace}
\newcommand{\qbit}{\keyfont{qbit }\xspace}
%\newcommand{\discard}[1]{\keyfont{discard } #1\xspace}
\newcommand{\MEASURE}{\ensuremath{\keyfont{measure}}\xspace}
\newcommand{\mif}[3]{\ensuremath{\MEASURE \ {#1} \ \THEN\ {#2} \ \ELSE\ {#3}}\xspace}
\newcommand{\timeseq}{\mathrel{{*}{=}}} % replace with unitary
\newcommand{\ds}[3]{\ensuremath{\denote{\langle #1 \rangle \ #2 \ \langle #3 \rangle}}\xspace}
\newcommand{\mat}[1]{\mathbf{#1}} %May not use
%\newcommand{\unitary}[2]{\ensuremath{\assign{#1}{\mat{#2} #1}}\xspace}
%\newcommand{\unitary}[2]{\ensuremath{#1 \timeseq #2}\xspace}
\newcommand{\mcase}[3]{\ensuremath{\texttt{measure} \ #1[#2] \ : \ #3}\xspace}
\newcommand{\mwhile}[3]{\ensuremath{\WHILE \ #1[#2] \ \DO \ #3}\xspace}
\newcommand{\lqtd}{\ensuremath{\mathcal{L}_{qTD}}\xspace}

\newcommand{\highlight}[1]{\textcolor{red}{#1}}

%% Underscore issues
%\usepackage{relsize}
%\renewcommand{\_}{\textscale{.7}{\textunderscore}}
% LaTeX says no...

%% Unicode
\usepackage{newunicodechar}
\let\Alpha=A
\let\Beta=B
\let\Epsilon=E
\let\Zeta=Z
\let\Eta=H
\let\Iota=I
\let\Kappa=K
\let\Mu=M
\let\Nu=N
\let\Omicron=O
\let\omicron=o
\let\Rho=P
\let\Tau=T
\let\Chi=X

\newunicodechar{Α}{\ensuremath{\Alpha}}
\newunicodechar{α}{\ensuremath{\alpha}}
\newunicodechar{Β}{\ensuremath{\Beta}}
\newunicodechar{β}{\ensuremath{\beta}}
\newunicodechar{Γ}{\ensuremath{\Gamma}}
\newunicodechar{γ}{\ensuremath{\gamma}}
\newunicodechar{Δ}{\ensuremath{\Delta}}
\newunicodechar{δ}{\ensuremath{\delta}}
\newunicodechar{Ε}{\ensuremath{\Epsilon}}
\newunicodechar{ε}{\ensuremath{\epsilon}}
\newunicodechar{ϵ}{\ensuremath{\varepsilon}}
\newunicodechar{Ζ}{\ensuremath{\Zeta}}
\newunicodechar{ζ}{\ensuremath{\zeta}}
\newunicodechar{Η}{\ensuremath{\Eta}}
\newunicodechar{η}{\ensuremath{\eta}}
\newunicodechar{Θ}{\ensuremath{\Theta}}
\newunicodechar{θ}{\ensuremath{\theta}}
\newunicodechar{ϑ}{\ensuremath{\vartheta}}
\newunicodechar{Ι}{\ensuremath{\Iota}}
\newunicodechar{ι}{\ensuremath{\iota}}
\newunicodechar{Κ}{\ensuremath{\Kappa}}
\newunicodechar{κ}{\ensuremath{\kappa}}
\newunicodechar{Λ}{\ensuremath{\Lambda}}
\newunicodechar{λ}{\ensuremath{\lambda}}
\newunicodechar{Μ}{\ensuremath{\Mu}}
\newunicodechar{μ}{\ensuremath{\mu}}
\newunicodechar{Ν}{\ensuremath{\Nu}}
\newunicodechar{ν}{\ensuremath{\nu}}
\newunicodechar{Ξ}{\ensuremath{\Xi}}
\newunicodechar{ξ}{\ensuremath{\xi}}
\newunicodechar{Ο}{\ensuremath{\Omicron}}
\newunicodechar{ο}{\ensuremath{\omicron}}
\newunicodechar{Π}{\ensuremath{\Pi}}
\newunicodechar{π}{\ensuremath{\pi}}
\newunicodechar{ϖ}{\ensuremath{\varpi}}
\newunicodechar{Ρ}{\ensuremath{\Rho}}
\newunicodechar{ρ}{\ensuremath{\rho}}
\newunicodechar{ϱ}{\ensuremath{\varrho}}
\newunicodechar{Σ}{\ensuremath{\Sigma}}
\newunicodechar{σ}{\ensuremath{\sigma}}
\newunicodechar{ς}{\ensuremath{\varsigma}}
\newunicodechar{Τ}{\ensuremath{\Tau}}
\newunicodechar{τ}{\ensuremath{\tau}}
\newunicodechar{Υ}{\ensuremath{\Upsilon}}
\newunicodechar{υ}{\ensuremath{\upsilon}}
\newunicodechar{Φ}{\ensuremath{\Phi}}
\newunicodechar{φ}{\ensuremath{\phi}}
\newunicodechar{ϕ}{\ensuremath{\varphi}}
\newunicodechar{Χ}{\ensuremath{\Chi}}
\newunicodechar{χ}{\ensuremath{\chi}}
\newunicodechar{Ψ}{\ensuremath{\Psi}}
\newunicodechar{ψ}{\ensuremath{\psi}}
\newunicodechar{Ω}{\ensuremath{\Omega}}
\newunicodechar{ω}{\ensuremath{\omega}}

\newunicodechar{ℕ}{\ensuremath{\mathbb{N}}}
\newunicodechar{∅}{\ensuremath{\emptyset}}

\newunicodechar{∙}{\ensuremath{\bullet}}
\newunicodechar{≈}{\ensuremath{\approx}}
\newunicodechar{≅}{\ensuremath{\cong}}
\newunicodechar{≡}{\ensuremath{\equiv}}
\newunicodechar{≤}{\ensuremath{\le}}
\newunicodechar{≥}{\ensuremath{\ge}}
\newunicodechar{≠}{\ensuremath{\neq}}
\newunicodechar{∀}{\ensuremath{\forall}}
\newunicodechar{∃}{\ensuremath{\exists}}
\newunicodechar{±}{\ensuremath{\pm}}
\newunicodechar{∓}{\ensuremath{\pm}}
\newunicodechar{·}{\ensuremath{\cdot}}
\newunicodechar{⋯}{\ensuremath{\cdots}}
\newunicodechar{…}{\ensuremath{\ldots}}
\newunicodechar{∷}{~\mathrel{:\!\!\!:}~}
\newunicodechar{×}{\ensuremath{\times}}
\newunicodechar{∞}{\ensuremath{\infty}}
\newunicodechar{→}{\ensuremath{\to}}
\newunicodechar{←}{\ensuremath{\leftarrow}}
\newunicodechar{⇒}{\ensuremath{\Rightarrow}}
\newunicodechar{↦}{\ensuremath{\mapsto}}
\newunicodechar{↝}{\ensuremath{\leadsto}}
\newunicodechar{∨}{\ensuremath{\vee}}
\newunicodechar{∧}{\ensuremath{\wedge}}
\newunicodechar{⊢}{\ensuremath{\vdash}}
\newunicodechar{⊣}{\ensuremath{\dashv}}
\newunicodechar{∣}{\ensuremath{\mid}}
\newunicodechar{∈}{\ensuremath{\in}}
\newunicodechar{⊆}{\ensuremath{\subseteq}}
\newunicodechar{⊂}{\ensuremath{\subset}}
\newunicodechar{∪}{\ensuremath{\cup}}
\newunicodechar{⋓}{\ensuremath{\Cup}}
\newunicodechar{∉}{\ensuremath{\not\in}}
\newunicodechar{√}{\ensuremath{\sqrt}}



\newunicodechar{⊸}{\ensuremath{\multimap}}
\newunicodechar{⊗}{\ensuremath{\otimes}}
\newunicodechar{⨂}{\ensuremath{\bigotimes}}
\newunicodechar{⊕}{\ensuremath{\oplus}}
\newunicodechar{〈}{\ensuremath{\langle}}
\newunicodechar{⟨}{\ensuremath{\langle}}
\newunicodechar{⟩}{\ensuremath{\rangle}}
\newunicodechar{〉}{\ensuremath{\rangle}}
\newunicodechar{¡}{\ensuremath{\upsidedownbang}}
\newunicodechar{∘}{\ensuremath{\circ}}
\newunicodechar{†}{\ensuremath{\dagger}}
\newunicodechar{⊤}{\ensuremath{\top}}
\newunicodechar{⊥}{\ensuremath{\bot}}

\newunicodechar{〚}{\ensuremath{\llbracket}}
\newunicodechar{〛}{\ensuremath{\rrbracket}}

%% COMMENTS 
\usepackage{etoolbox} % replaces ifthen package
\newtoggle{comments}
\toggletrue{comments}
%\togglefalse{comments}
  \usepackage[normalem]{ulem}
%  \usepackage{minted}

\iftoggle{comments}{
  \newcommand{\fixme}[1]{\textbf{\textcolor{red}{[ Fixme: #1]}}}
  \newcommand{\todo}[1]{\textbf{\textcolor{green}{[ TODO: #1 ]}}}
  \newcommand{\mwh}[1]{\textbf{\textcolor{red}{[ Mike: #1 ]}}}
  \newcommand{\yxp}[1]{\textbf{\textcolor{blue}{[ Yuxiang: #1 ]}}}
  \newcommand{\khh}[1]{\textbf{\textcolor{orange}{[ Kesha: #1 ]}}}
  \newcommand{\shh}[1]{\textbf{\textcolor{purple}{[ Shih-Han: #1 ]}}}
  \newcommand{\liyi}[1]{\textbf{\textcolor{blue}{[ Liyi: #1 ]}}}
\newcommand{\anshu}[1]{\textbf{\textcolor{olive}{[ Anshu: #1 ]}}}
  \newcommand{\oth}[2]{\textbf{\textcolor{red}{[ #1: #2 ]}}}
  \newcommand{\xwu}[1]{\textbf{\textcolor{purple}{[ Xiaodi: #1 ]}}}
  \newcommand{\fsdv}[1]{\textbf{\textcolor{pink}{[ Finn: #1 ]}}}
  \newcommand{\ynote}[1]{\textbf{\textcolor{magenta}{[ Yi: #1 ]}}}

  \colorlet{MZ}{violet!80!pink}
  \newcommand{\mz}[1]{{\textcolor{MZ}{\textbf{[[}(\(\mu\)) {\small{#1}}\textbf{]]}}}}
  \newcommand{\mzs}[1]{{\color{MZ}{\sout{#1}}}}%
  \newcommand{\mzu}[1]{{\color{MZ}\uline{#1}}}
  \newcommand{\mzr}[1]{{\color{MZ}{#1}}}
  \newcommand{\was}[1]{}
  \newcommand{\mzsub}[2]{\mzs{#1}\mzr{#2}}
  % trick to get around with sout
  \NewCommandCopy{\Creff}{\Cref}
  \renewcommand{\Cref}[1]{\mbox{\Creff{#1}}}

  \usepackage[inline]{enumitem}
  \colorlet{LC}{cyan!31!teal}

  \newcommand{\lc}[1]{{\color{LC}\textbf{\textit{Le: #1}}}}
  \newcommand{\lcs}[1]{{\color{LC} \sout{#1}}}
  \newcommand{\lcu}[1]{{\color{LC}\uline{#1}}}
  \newcommand{\lcr}[1]{{\color{LC}{#1}}}
  \usepackage[inline]{enumitem}
}{
  \newcommand{\fixme}[1]{}
  \newcommand{\todo}[1]{}
  \newcommand{\rnr}[1]{}
  \newcommand{\mwh}[1]{}  
  \newcommand{\khh}[1]{}
  \newcommand{\liyi}[1]{}
  \newcommand{\shh}[1]{}
  \newcommand{\xwu}[1]{}
  \newcommand{\oth}[2]{}
  \newcommand{\mzr}[1]{}

  \newcommand{\ynote}[1]{}
}

\newtoggle{submission}
\toggletrue{submission}
%\togglefalse{submission}

\iftoggle{submission}{
  \newcommand{\aref}[1]{\Cref{#1} of the extended version of this paper}
}{
  \newcommand{\aref}[1]{\Cref{#1}}
}


%% END COMMENTS

%%% Local Variables:
%%% mode: latex
%%% TeX-master: "main"
%%% End:

\usepackage{pifont}% http://ctan.org/pkg/pifont
\usepackage{multicol,tabularx,capt-of}
\usepackage{multirow}
\newcommand{\cmark}{\text{\ding{51}}}
\newcommand{\xmark}{\text{\ding{55}}}
%\usepackage{txfonts}
%\usepackage{unicode-math}

\begin{document}
%\bibliographystyle{plainurl}%% Citation style


\title{Validating Quantum State Preparation Programs}                      %% \subtitle is optional

\def\titlerunning{Validating Quantum State Preparation Programs}
\def\authorrunning{L. Li, A. Sharma, Z. Tagba, S. Frett, A. Potanin}

\author{Liyi Li}
\affiliation{
  \institution{Iowa State University}
  \country{USA}
}
\email{liyili2@iastate.edu}

\author{Anshu Sharma}
\affiliation{
  \institution{The College of William and Mary}
  \country{USA}
}
\email{agsharma@wm.edu}

\author{Zoukarneini Difaizi Tagba}
\affiliation{
  \institution{Iowa State University}
  \country{USA}
}
\email{difaizi@iastate.edu}

\author{Sean Frett}
\affiliation{
  \institution{Iowa State University}
  \country{USA}
}
\email{fretts@iastate.edu}

\author{Alex Potanin}
\affiliation{
  \institution{Australian National University}
  \country{Australia}
}
\email{alex.potanin@anu.edu.au}

%\author{Rance Cleaveland}{University of Maryland}{rance@cs.umd.edu}{}{}

%\author{Alexander Nicolellis}{Iowa State University}{akn5@iastate.edu}{}{}

%\author{Yi Lee}{University of Maryland}{ylee1228@umd.edu}{}{}

%\author{Le Chang}{University of Maryland}{lchang21@umd.edu}{}{}

%\author{Xiaodi Wu}{University of Maryland}{xwu@cs.umd.edu}{https://orcid.org/0000-0001-8877-9802}{}

%\authorrunning{L. Li, A. Sharma, Z. Tagba, A. Potanin} %TODO mandatory. First: Use abbreviated first/middle names. Second (only in severe cases): Use first author plus 'et al.'

%\Copyright{Jane Open Access and Joan R. Public} %TODO mandatory, please use full first names. LIPIcs license is "CC-BY";  http://creativecommons.org/licenses/by/3.0/

%\Copyright{Li, Sharma, Tagba, Potanin.} %TODO mandatory, please use full first names. LIPIcs license is "CC-BY";  http://creativecommons.org/licenses/by/3.0/

%\ccsdesc[100]{} %TODO mandatory: Please choose ACM 2012 classifications from https://dl.acm.org/ccs/ccs_flat.cfm 

%\keywords{Quantum Computing, Compiler Validation, Property-based Testing}

%\relatedversion{} %optional, e.g. full version hosted on arXiv, HAL, or other respository/website

%\acknowledgements{We thank Finn Voichick for his helpful comments and contributions during the work's development. This paper is dedicated to the memory of our dear co-author Rance Cleaveland.}%optional
%\includeonly{appendix}


%% Title information
%\title{The Quantum Superposition Virtual Machine}         %% [Short Title] is optional;
                                        %% when present, will be used in
                                        %% header instead of Full Title.
%\titlenote{with title note}             %% \titlenote is optional;
                                        %% can be repeated if necessary;
                                        %% contents suppressed with 'anonymous'
%\subtitlenote{Extended Version}       %% \subtitlenote is optional;
                                        %% can be repeated if necessary;
                                        %% contents suppressed with 'anonymous'


%% Author information
%% Contents and number of authors suppressed with 'anonymous'.
%% Each author should be introduced by \author, followed by
%% \authornote (optional), \orcid (optional), \affiliation, and
%% \email.
%% An author may have multiple affiliations and/or emails; repeat the
%% appropriate command.
%% Many elements are not rendered, but should be provided for metadata
%% extraction tools.

%% Author with single affiliation.
%% Abstract
%% Note: \begin{abstract}...\end{abstract} environment must come
%% before \maketitle command
\begin{abstract}
 One of the key steps in quantum algorithms is to prepare an initial quantum superposition state with different kinds of features. These so-called \emph{state preparation} algorithms are essential to the behavior of quantum algorithms, and complicated state preparation algorithms are difficult to program correctly and effectively.
This paper presents QSV: a high-assurance framework implemented with the Rocq proof assistant, permitting the development of quantum state preparation programs and validating them to correctly reflect quantum program behaviors.
The key in the framework is to reduce the program correctness assurance of a program containing a quantum superposition state to the program correctness assurance for the program state without superposition.
The reduction allows the development of \textit{an effective framework for validating quantum state preparation algorithm implementations on a classical computer} --- considered a hard problem with no clear solution until this point.
We utilize the QuickChick property-based testing framework to validate state preparation programs.
We evaluated the effectiveness of our approach over 5 case studies implemented using QSV; such cases are not even simulatable in the current quantum simulators.

\end{abstract}

\begin{CCSXML}
<ccs2012>
<concept>
<concept_id>10011007.10011006.10011008</concept_id>
<concept_desc>Software and its engineering~General programming languages</concept_desc>
<concept_significance>500</concept_significance>
</concept>
<concept>
<concept_id>10003456.10003457.10003521.10003525</concept_id>
<concept_desc>Social and professional topics~History of programming languages</concept_desc>
<concept_significance>300</concept_significance>
</concept>
</ccs2012>
\end{CCSXML}

\maketitle

%% 2012 ACM Computing Classification System (CSS) concepts
%% Generate at 'http://dl.acm.org/ccs/ccs.cfm'.
% \begin{CCSXML}
% <ccs2012>
% <concept>
% <concept_id>10011007.10011006.10011008</concept_id>
% <concept_desc>Software and its engineering~General programming languages</concept_desc>
% <concept_significance>500</concept_significance>
% </concept>
% <concept>
% <concept_id>10003456.10003457.10003521.10003525</concept_id>
% <concept_desc>Social and professional topics~History of programming languages</concept_desc>
% <concept_significance>300</concept_significance>
% </concept>
% </ccs2012>
% \end{CCSXML}

% \ccsdesc[500]{Software and its engineering~General programming languages}
% \ccsdesc[300]{Social and professional topics~History of programming languages}
%% End of generated code


%% Keywords
%% comma separated list
%\keywords{keyword1, keyword2, keyword3}  %% \keywords are mandatory in final camera-ready submission


%% \maketitle
%% Note: \maketitle command must come after title commands, author
%% commands, abstract environment, Computing Classification System
%% environment and commands, and keywords command.


%\textcolor{red}{Pointing out quantum conditional proofs somewhere. The essense of qafny is to permit inter-op automated proof.}


\section{Introduction}
\label{sec:intro}

% QC offers speedups
% - Source of speedup: Oracle, run coherently
Despite recent advances ~\cite{morphq_bugs,fuzz4all,10.1109/ASE51524.2021.9678798,fortunato,long:24,QDiff}, quantum program developers still lack tools to quickly validate program correctness---testing a program with many test inputs in a short time---as well as other properties when writing comprehensive quantum programs ~\cite{gill2024quantumcomputingvisionchallenges,MattSwayne}.
%By validating the initial quantum state preparation procedure, we ensure that the quantum program will produce the right state when it is executed on a quantum computer.
Testing quantum programs directly on quantum hardware is problematic because running actual quantum computers is expensive, and the probabilistic nature of quantum computing means repeated trials may be necessary to validate correctness, driving up costs further.
Ideally, we should ensure that a program satisfies user specifications before running it on quantum hardware.
Unfortunately, such a framework might not exist for validating (testing) a quantum program for arbitrary properties since quantum programs are not classically simulatable without an exponential number of classical bits relative to qubits.
%, and because of the difficulty of simulating true quantum randomness.

%Quantum computers can be used to program substantially faster algorithms compared to those written for classical computers.
%For example, Shor's algorithm \cite{shors} can factor a number in polynomial time, which is not known to be possible on classical computers.
%Developing more comprehensive quantum programs and algorithms is essential for the continued practical development of Quantum Computing (QC)
% Alex's sentence rewrite (please check):
%Since executing quantum programs on quantum hardware is expensive and probabilistic, one of the key challenges of QC is enabling reasoning about quantum program correctness properties and effective validation using a deterministic classical computer.
%
% A key quantum programming facility is an effective validation framework that can assert program properties, at least the correctness properties, on a classical computer.
%
%Since quantum programs are not classically simulatable without an exponential number of classical bits relative to qubits (and because of the difficulty of simulating true quantum randomness), such a framework might not exist for testing a quantum program for arbitrary properties.
%Many verification approaches have been proposed for quantum algorithms~\cite{qhoreusage,qhoare,qbricks,qsepa,qseplocal,VOQC,li2024}.
%These include using interactive theorem provers, such as Isabelle, Coq, and Why3, and building quantum semantic interpretations and libraries --- however, developing any programs via verification frameworks can be time-consuming.


A quantum validation framework needs to satisfy three key design goals.

\begin{itemize}
\item Programmers can develop a quantum program based on a proper abstraction, with respect to high-level program properties, without worrying too much about low-level gates.

\item The framework contains a scalable and effective validator to quickly judge the correctness of a user-defined program, as well as other properties, based on certain types of quantum program patterns.

\item The validated program can be compiled into the quantum circuit for execution.
\end{itemize}

\begin{figure}[h]
\vspace*{-1em}
  \includegraphics[width=0.7\textwidth]{pflow}
   \caption{The QSV Flow}
   \vspace*{-0.7em}
\label{fig:qsv}
\end{figure}

We propose the Quantum State preparation program Validation framework (QSV), which has the flow in \Cref{fig:qsv}, permitting effective validation of state preparation programs (The program limitation is discussed in \Cref{sec:conclusion}).
It includes three components. Users can develop their programs in a PQASM language, specifically to allow them to write state preparation programs in a high-level abstraction. Such programs can be validated by our validator, based on QuickChick \cite{quickchick}  (a Rocq-based property-based testing facility),
and we show several quantum program patterns that can be effectively validated via our framework.
After a program is adequately developed in our framework, users can use our certified circuit compiler to compile the program to a quantum circuit, executable in quantum hardware.

\subsection{Motivating Examples}\label{sec:motivation}

We begin by discussing a simple state preparation subroutine, preparing a superposition of $n$ distinct basis-ket states, appearing in many algorithms \cite{Gorjan2007,mike-and-ike}, having the following program transition property predicate (a pre-state is transitioned to a post-state connected by $\to$) with program input of a length $m$ qubit array, initialized as $\aket{0}{m}$, and output a superposition of $n$ different basis-ket states, each with basis-vector $\aket{k}{m}$.

{\small
\begin{center}
$
\aket{0}{m}\to\sum_{j=0}^{n-1}\frac{1}{\sqrt{n}}\aket{j}{m}
$
\end{center}
}

%Previously, VQO \cite{oracleoopsla} developed a toolchain that is capable of effectively testing (on a classical computer) quantum arithmetic oracle programs, which is a subset of quantum programs and is a key component in many quantum algorithms.
%This indicated that we could identify a subset of quantum programs that might be effectively testable for some key properties.
%
%This paper proposes a new system, \pqasm, which intends to be a framework for effectively testing quantum superposition state preparation programs, at least for the correctness properties.
%A quantum state preparation program can be generalized as the starting component of a quantum algorithm.

Generally speaking, \emph{a state preparation program can be defined as the starting component of a quantum algorithm.}
A quantum algorithm typically starts with a length $m$ qubit array, each qubit initialized as zero ($\aket{0}{m}$) state, and prepares a superposition state $\sum_j \alpha_j \aket{c_j}{m}$ --- a linear sum of pairs (basis-kets) of complex amplitude $\alpha_j$ and bitstring (basis-vector) $c_j$ such that $\sum_j \slen{\alpha_j}^{2} =1 $ --- via a series of quantum operations.

Superposition is a key feature of quantum states, and quantum computers can execute programs with superposition states to query all possible inputs simultaneously, as discussed in \Cref{sec:background}.
Many quantum algorithms require a comprehensive design of state preparation components with different superposition structures.
For example, in Shor's algorithm, we need to prepare a superposition of pairs of a number ($x$) and the modular multiplication result of the number ($\modexp{c}{x}{n}$) (\Cref{fig:mod-mult}).

A difficulty in developing state preparation programs is that quantum program operations might affect every basis-ket in a superposition that might contain exponentially many basis-ket states, unlike the classical programs, where only a single basis-ket might be affected.
For example, in \Cref{fig:intros2}, a function $f$ is applied to every basis-ket in the quantum superposition state after all Hadamard operations were applied.
Moreover, many quantum languages are circuit gate-based \cite{Qiskit2019,tket,Cross2017,10.1145/3505636}.
These hinder the quantum program development as writing programs becomes unintuitive.
For example, given the above simple $n$ basis-ket program property, it is not straightforward to develop the program.

\begin{figure}[t]
\vspace*{-0.5em}
{\footnotesize
\begin{minipage}[t]{0.46\textwidth}
\subcaption{State Preparation Circuit}
\label{fig:intros2}
\vspace*{0.5em}
{\scriptsize
  \Qcircuit @C=0.5em @R=0.5em {
   & & \qw    & \gate{H} & \qw & \multigate{3}{{ f(\ket{j})=(-i)^{j}\aket{j}{m}}}   & \qw &\qw & \qw \\
  \push{x:\aket{0}{m}\quad} & &  \vdots &          &     &                                          &    \rstick{{ {\Msum_{j=0}^{2^m\sminus 1}(-i)^{j}\aket{j}{m}}}} & &\\
   & & \vdots  &          &     &                                          &     &       &  \\
   & &  \qw   & \gate{H} & \qw &  \ghost{{ f(\ket{j})=(-i)^{j}\aket{j}{m}}}         &\qw  &\qw    & \qw
      \gategroup{1}{2}{4}{2}{1em}{\{}
    }
}
\end{minipage}
\hfill
\begin{minipage}[t]{0.46\textwidth}
\subcaption{Preparing Superposition of $n$ Basis-kets}
\label{fig:intros-example}
\vspace*{0.5em}
 { \scriptsize
  \Qcircuit @C=0.5em @R=0.5em {
    &                     & & \qw & \gate{H} & \qw & \qw & \multigate{6}{\texttt{$\qbool{x}{<}{n}{y}$}} & \qw & \qw & \qw & \\
    & \push{x:\aket{0}{m}\quad} & & \qw & \gate{H} & \qw & \qw &  \ghost{\texttt{$\qbool{x}{<}{n}{y}$}}       &\qw & \qw & \qw &  \\
    & & &  &       & & &  & & & & \push{\varphi} \\
    & & &  & \dots & & & & & & & \\
    & & & & & & & & & & & \\
    &                     & & \qw & \gate{H} & \qw & \qw &   \ghost{\texttt{$\qbool{x}{<}{n}{y}$}}      & \qw & \qw & \qw & \\
    & \push{y:\aket{0}{1}\quad} & & \qw & \qw      & \qw & \qw & \ghost{\texttt{$\qbool{x}{<}{n}{y}$}}        & \qw & \qw & \meter & \push{v} 
    \gategroup{1}{3}{6}{3}{1em}{\{}
    \gategroup{1}{11}{6}{11}{1em}{\}}
    }
  }
\end{minipage}
}
\vspace*{-1em}
\caption{$x$ has $m$ qubits, and $y$ register has $1$ qubit. The right figure is one step in the repeat-until-success program to prepare the superposition state. $\varphi=\frac{1}{\sqrt{n}}\sum_{j=0}^n\aket{j}{m}$ if $v=1$; otherwise, $\varphi=\frac{1}{\sqrt{2^m-n}}\sum_{j=n}^{2^m}\aket{j}{m}$.}
\label{fig:circuits}
\vspace*{-1em}
\end{figure}

\Cref{fig:intros-example} shows a one-step procedure of the repeat-until-success program implementation of the $n$ basis-ket program.
The procedure starts with a series of Hadamard operations to prepare a uniform superposition of $2^m$ basis-kets as $\frac{1}{\sqrt{2^m}}\sum_{j=0}^{2^m}\aket{j}{m}$,
and then compares each basis-ket with the natural number $n$ and stores the comparison result in the extra $y$ qubit.
After measuring the $y$ qubit and if the result is $1$, all the basis-kets ($\alpha_j\aket{j}{m}$), having bitstring numbers $j \ge n$, disappear, while those having bitstrings $j<n$ will stay in $x$'s quantum state; thus, the correct state is prepared.
Since the measurement result $1$ happens probabilistically, the repeat-until-success program requires repeating the one-step procedure many times to prepare the target state probabilistically.
In writing the program, a key component is the comparator $\qbool{x}{<}{n}{y}$, comparing every basis-vector of a quantum array $x$ with number $n$ and storing the result in qubit $y$.
Such arithmetic operations have effective implementations \cite{oracleoopsla} and many works discuss circuit-level optimizations \cite{VOQC,Xu2022}.
Therefore, QSV abstracts all these quantum arithmetic operations and relieves the pain of writing quantum programs.
The QSV compiler compiles and optimizes the arithmetic operations to quantum circuits. 
Moreover, we also provide types to classify different program patterns for users to write state preparation programs.

Even if we permit high-level abstractions in QSV, validating a program might still be challenging because of the exponential basis-kets in a quantum state,
e.g., the output state $\sum_{j=0}^{n-1}\frac{1}{\sqrt{n}}\aket{j}{m}$ might contain exponentially many basis-kets, checking them individually might be challenging.
In developing our validator, we have two observations on quantum algorithms.
First, almost all quantum algorithms start with $m$ different Hadamard operations to prepare a uniform superposition having $2^m$ basis-kets. These beginning Hadamard operations are simple enough and do not need to be validated, but they provide the source of superposition state for later program operations to carve on. 
Second, even though quantum operations are probabilistic, one can "determinize" their behaviors.
If we consider the superposition of basis-kets as an array of basis-kets, quantum operations, except measurement, behave similarly to higher-order map functions applying to the quantum state. Measurement behaves similarly to a set selection, selecting a basis-ket element in the array, and the probability of such selection can be computed based on the amplitude value associated with the basis-ket.

In QSV, we classify beginning Hadamard operations as a special \cn{Had} type, to indicate that they are the source of the superposition state. When performing testing, instead of faithfully representing their operational behavior, QSV adopts a random pick of an individual basis-ket state as a representative, and validates programs based on transitions of the basis-ket. A measurement operation is then determinized and its probability is simply calculated via the amplitude value in the basis-ket.

For example, in dealing with the $n$ basis-ket program above, after applying the Hadamard operations, the program property is turned as the one on the left below, where $\sum_{j=0}^{2^m\sminus 1}\frac{1}{\sqrt{2^m}}\aket{j}{m}$ being the result of applying $m$ Hadamard operations. The QSV process of determinizing the basis-kets is to select a particular $j$, which turns the program property to be the right one.

{\footnotesize
\begin{mathpar}
 \inferrule[]{}{ \sum_{j=0}^{2^m\sminus 1}\frac{1}{\sqrt{2^m}}\aket{j}{m}\to\sum_{j=0}^{n\sminus 1}\frac{1}{\sqrt{n}}\aket{j}{m}}
  
   \inferrule[]{}{\forall j \in [0,2^m) \,,\, \frac{1}{\sqrt{2^m}}\aket{j}{m} \to(\frac{1}{\sqrt{n}}\aket{j}{m} \wedge j \in [0, n))}
    \end{mathpar}
}

Essentially, the validation process based on the right property is to assume a single basis-ket $\frac{1}{\sqrt{2^m}}\aket{j}{m}$ with a bitstring $j \in [0,2^m)$, then to validate to see if the output is the bitstring $j$ within range of $[0,n)$ and the associated amplitude being $\frac{1}{\sqrt{n}}$.
Via a property-based testing facility, such as QuickChick, by testing the program with enough candidate basis-kets, it will be highly likely that our validator can capture a bug if there is any. 

After validation, we compile the program to a quantum circuit via our certified compiler.

%In preparing a quantum state, we utilize a generalized summation formula on the right above,
%where the correctness of a function $f$, applying to a pair, is defined as $(x_j,y) \longrightarrow f(x_j,y)$, for all $j$;
%and the semantics of applying the function to a superposition state is inferred as the bottom transition.
%In many state preparation programs, a standard program structure first applies different Hadamard operations and then applies a function $f$ having a similar structure above.

%In analyzing many state preparation programs, we found that a typical starting pattern is that one begins with a step of initializing $m$ different $\ket{0}$ qubits followed by many Hadamard operations to prepare a simple uniform superposition state ($\frac{1}{\sqrt{2^m}}\sum_{j=0}^{2^m}\ket{j}$).


%Using the above correctness property for a program $e$, the testing can be defined as giving a superposition input state $\varphi$. We pick a candidate basis-ket $\ket{j}$, such that $j\in[0,2^m)$, and then the program outputs another superposition state $\varphi'$, with a candidate basis-ket $\ket{j}$, with the restriction $j\in [0,n)$.
%The testing still utilizes the summation formula above, but in the opposite direction, i.e., the correctness property is a transition between two superposition states. We test the property by picking up candidate basis-kets and ensure that the output candidate basis-ket satisfies certain restrictions, such as the above restriction $j\in [0,n)$.

\ignore{
State preparation programs typically contain a quantum oracle subcomponent, effectively definable in VQO \cite{oracleoopsla}.
For example, the comparison program ($x<n @ y$) in \Cref{fig:intros-example} has the functional behavior $\ket{x}\ket{0}\to \ket{x}\ket{x<n}$, where it takes in length $m$ qubit array $x$ and a single qubit ($y=\ket{0}$), and stores the comparison result of the number represented as a bitstring ($x$) with $n$ in the single qubit
i.e., each basis-state of a length $m$ qubit array $x$ can be viewed as a length $m$ bitstring, so the quantum comparison essentially compares each basis-state in the length $m$ qubit array with a number $n$.
%In \Cref{fig:mod-mult}, for each basis-state $\ket{u}$ in the qubit array $u$, each modulo-multiplication operation applies 

In preparing a quantum state, we utilize a generalized summation formula on the right above,
where the correctness of a function $f$, applying to a pair, is defined as $(x_j,y) \longrightarrow f(x_j,y)$, for all $j$;
and the semantics of applying the function to a superposition state is inferred as the bottom transition.
In many state preparation programs, a standard program structure first applies different Hadamard operations and then applies a function $f$ having a similar structure above.

A key difference between the correctness property of a quantum oracle operation and a state preparation is that the correctness of a quantum oracle operation is defined as the correctness of $f$ above, while the correctness of a state preparation is defined as the inferred linear sum state,
i.e., in \Cref{fig:circuits}, we actually prepare a superposition state, similar to the inferred linear sum state on the left summation formula above.
To utilize the summation formula for testing the correctness properties, we adopt the symbolic abstraction concept.
Here, we accept that a linear sum state is hard to test and might contain exponentially many basis-ket states.
On the other hand, we also notice that the components in a state preparation program might be similar to the function $f$ above,
where each basis-state is applied the same $f$, even though basis-ket values might differ.
The general testing procedure in \pqasm can be a two-step process.
First, we can view the linear sum state as a symbolic representation $\sum_j \eta(j)$, with $\eta(j)$ being an expression capturing the state property of a specific basis-ket.
Second, testing $f$ on $\eta(j)$ can be effectively performed for a single basis-state $\eta(j)$; we construct many different $\eta(j_v)$ for different $j_v$ and testing $f$ on different $\eta(j_v)$.
The testing result assures us that the resulting linear sum state $\sum_j f(\eta(j))$ is indeed what we are looking for, a.k.a., satisfying the correctness property.
}

%1. Quantum programs are not classically simulatable. 

%2. We want testing facilities for quantum programs. Previously, we have VQO for oracle programs. Does it exist other kinds of subsets.

%3. to have a testing facilities, need some abstract interpretation on the properties for program testing.

%4. Symbolic abstraction to perform testing on targeted properties, i.e., program correctness.

%5. Show two examples. CU operations in Shor's algorithm, and repeat-until-success for uniformed superposition.


\subsection{Contributions and Roadmap} 

We present QSV, a framework that enables programmers to develop state preparation programs.
Our contributions are as follows, with all Rocq proofs and experiment results available. We discuss QSV and \pqasm limitations in \Cref{sec:conclusion}.

%\item We present a validated circuit compiler from \qafny to \sqir, showing that the \qafny proof system correctly reflects quantum program semantics due to its sound and completeness.

\begin{itemize}
\item We present the syntax, semantics, and type system of \pqasm, allowing users to define state preparation programs with the proof of type soundness in Rocq (\Cref{sec:pqasm}).

\item We develop a property-based testing (PBT) framework for validating programs written in \pqasm (\Cref{sec:implementation}) by showing a general flow of constructing such PBT frameworks for validating quantum programs.

\item We certify a compiler from \pqasm to \sqir \cite{VOQC} (\Cref{sec:vqir-compilation}) to ensure that our \pqasm tool correctly reflect quantum program behaviors.

%\item \qafny is based on \emph{classical} separation logic, and we show how to compile from \qafny to classical separation logic. 

\item We evaluate \pqasm via a selection of state preparation programs and demonstrate that QSV is capable of validating the programs (\Cref{sec:evaluation,sec:eval}). 
These programs were previously considered to be hard or impossible to simulate on classical machines, and some of them (amplitude amplification, hamming weight, and element distinctness state preparation programs) were never verified or validated. We attempted to run these programs in an industrial quantum simulator, Qiskit and DDSim \cite{ddsim} (a key in the Munich Quantum Toolkit \cite{mqt}). None of the state preparation programs are simulatable in the simulators (\Cref{sec:eval}) if the input qubit length is a normal one (60 qubits per register and up to 361 qubits as the total input qubit size). In contrast, QSV can effectively run 10,000 test cases to validate these large qubit length programs via QuickChick. 
\end{itemize}
                   % intro
\vspace*{-0.5em}
\section{Background}
\label{sec:background}

Here, we provide background information on Quantum Computing. 

\noindent\textbf{\textit{Quantum Data}.} A quantum datum
%\footnote{Most literature describes quantum data as \emph{quantum states}. Here, we refer to them as quantum data to avoid confusion between program and quantum states.  } %
consists of one or more quantum bits (\emph{qubits}), which can be expressed as a two-dimensional vector $\begin{psmallmatrix} \alpha \\ \beta \end{psmallmatrix}$ where the \emph{amplitudes} $\alpha$ and $\beta$ are complex numbers and $|\alpha|^2 + |\beta|^2 = 1$.
%
We frequently write the qubit vector as $\alpha\aket{0}{1} + \beta\aket{1}{1}$ (the Dirac notation \cite{Dirac1939ANN}), where $\aket{0}{1} = \begin{psmallmatrix} 1 \\ 0 \end{psmallmatrix}$ and $\aket{1}{1} = \begin{psmallmatrix} 0 \\ 1 \end{psmallmatrix}$ are \emph{computational basis-vectors} and $\alpha\aket{0}{1}$ and $\beta\aket{1}{1}$ are basis-kets. The subscripts indicate the number of qubits for the basis-ket. When no necessity of mentioning qubit numbers, we denote the basis-ket as $\ket{0}$ or $\ket{1}$ by omitting them.
When both $\alpha$ and $\beta$ are non-zero, we can think of the qubit being ``both 0 and 1 at once,'' a.k.a. in a \emph{superposition} \cite{mike-and-ike}, e.g., $\frac{1}{\sqrt{2}}(\aket{0}{1} + \aket{1}{1})$ represents a superposition of $\aket{0}{1}$ and $\aket{1}{1}$.
Larger quantum data can be formed by composing smaller ones with the \emph{tensor product} ($\otimes$) from linear algebra, e.g., the two-qubit datum $\aket{0}{1} \otimes \aket{1}{1}$ (also written as $\aket{01}{2}$) corresponds to vector $[~0~1~0~0~]^T$.
% 
However, many multi-qubit data cannot be \emph{separated} and expressed as the tensor product of smaller data; such inseparable datum states are called \emph{entangled}, e.g.\/ $\frac{1}{\sqrt{2}}(\aket{00}{2} + \aket{11}{2})$, known as a \emph{Bell pair}, which can be rewritten to $\Msum_{d=0}^{1}{\frac{1}{\sqrt{2}}}{\aket{dd}{2}}$, where $dd$ is a bit string consisting of two bits, both being the same (i.e., $d=0$ or $d=1$). Each term $\frac{1}{\sqrt{2}}\aket{dd}{2}$ is named a \emph{basis-ket}, consisting an amplitude $\frac{1}{\sqrt{2}}$ and a basis vector $\aket{dd}{2}$.

%
%It is computationally hard to determine if an arbitrary quantum value is
%separable, therefore we say a general quantum value is \emph{possibly
%  entangled}.

%   
% An $n$-qubit quantum value state is typically represented as a $2^n$ dimensional
% vector. Alternatively, the values can be represented in different forms. For
% example, an initialized qubit typically has a value $\ket{0}$ or $\ket{1}$,
% called a \textit{normal typed value} ($\tnort$) in \qafny. A collection of $n$
% qubits that are in superposition but not entangled, i.e.,
% $\frac{1}{\sqrt{2}}(\ket{0} + \alpha(r_0)\ket{1})\otimes ... \otimes
% \frac{1}{\sqrt{2}}(\ket{0} + \alpha(r_{n-1})\ket{1})$, can be encoded as
% $\shad{2^n}{n\sminus 1}{\alpha(r_j)}$, where $\alpha(r_j)=e^{2\pi i r_j}$ and
% $r_j \in \mathbb{R}$, which is called a \textit{Hadamard typed value} ($\thadt$)
% in \qafny.

%Value $\alpha(r_j)$ is the \emph{local phase} of the state, which is a unique quantum amplitude whose norm is $1$, i.e., $\slen{\alpha(r_j)}=1$. 
%In the state $\frac{1}{\sqrt{2}}(\ket{0} + \ket{1})$, we can view the local phase $1$ as $e^{0}$, and $\frac{1}{\sqrt{2^n}}e^{0}$ is the amplitude for every basis-ket.
%This is not a standard form for all unentangled multi-qubit value states but rather a convenient way of representing a particular class of states common in many quantum algorithms, which can be utilized for proof automation.

% The most general representation for $n$-qubit values is as a linear combination of basis-kets \cite{mike-and-ike}, or in
% Dirac notation in \cite{Dirac1939ANN}, as $\sch{m}{z_j}{\ket{c_j}}$, where $z_j\in \mathbb{C}$ is an amplitude, $c_j$ is an $n$-length bitstring called a \emph{basis} (computational basis only), and $m < 2^n$. The above notation is the same as ${z_0}\ket{c_0}+...+{z_{m}}\ket{c_{m}}$; each $j$-th element ${z_j}\ket{c_j}$ represents a \emph{basis-ket} in the superposition value state. 
% This is called an \textit{entanglement typed value} ($\tcht$) in \qafny.
% For example, the bell pair can be represented as $\sch{1}{\frac{1}{\sqrt{2}}}{\ket{c_j}}$ with $c_0=00$ and $c_1=11$.
% Notice that the basis-kets' bases ($c_j$) are all distinct in a value state. 
% \qafny identifies these three different representations and uses a type system to transform quantum-state representations properly.

%%Notice that the amplitudes satisfy the relation $\sum_{0}^{m}\slen{z_j}^2 = 1$. However, in some intermediate program evaluation in QNP, we lose the restriction to be $\sum_{0}^{m}\slen{z_j}^2 \le 1$, because a state $\sch{m}{z_j}{c_j}$ can be split into two parts as $\sch{m}{z_j}{c_j}=\schii{m_1}{z_i}{c_i}+\schk{m_2}{z_k}{c_k}$, and we might only want to reason about a portion of the state $\schii{m_1}{z_i}{c_i}$ locally so that $\sum_{0}^{m_1}\slen{z_i}^2 < 1$. 
%
%%In QNP, each quantum state is associated with a \emph{session}, referring to a cluster of quantum array pieces possibly entangled. We can view a session as a virtual quantum array that manages quantum physical qubit array pieces living in different locations but is locally connected through entanglement. See \Cref{sec:quantum-state}.

          \begin{wrapfigure}{r}{3cm}
          %  \vspace*{-0.2em}
            {\qquad
              \footnotesize
              \Qcircuit @C=0.5em @R=0.5em {
                \lstick{\ket{0}} & \gate{H} & \ctrl{1} & \qw &\qw & & \dots & \\
                \lstick{\ket{0}} & \qw & \targ & \ctrl{1} & \qw & &  \dots &  \\
                \lstick{\ket{0}} & \qw & \qw   & \targ & \qw & &  \dots &  \\
                & \vdots &   &  &  & & & \\
                & \vdots &  & \dots & & & \ctrl{1} & \qw  \\
                \lstick{\ket{0}} & \qw & \qw & \qw &\qw &\qw & \targ & \qw
              }
            }
            \caption{GHZ Circuit}
            \label{fig:background-circuit-examplea}
          \end{wrapfigure}
          
\noindent\textbf{\textit{Quantum Computation and Measurement.}} 
%Quantum programming languages are essentially hybrid, containing both quantum and classical components, so that they can collaboratively finish a task (the \emph{QRAM model}~\cite{Knill1996}).
%
Computation on a quantum datum consists of a series of \emph{quantum operations}, each acting on a subset of qubits in the quantum datum. In the standard form, quantum computations are expressed as \emph{circuits}, as in \Cref{fig:background-circuit-examplea}, which depicts a circuit that prepares the Greenberger-Horne-Zeilinger (GHZ) state \cite{Greenberger1989} --- an $n$-qubit entangled datum of the form: $\ket{\text{GHZ}^n} = \frac{1}{\sqrt{2}}(\bigotimes^n \ket{0}+\bigotimes^n\ket{1})$.
In these circuits, each horizontal wire represents a qubit, and boxes on these wires indicate quantum operations, or \emph{gates}. The circuit in \Cref{fig:background-circuit-examplea} uses $n$ qubits and applies $n$ gates: a \emph{Hadamard} (\coqe{H}) gate and $n-1$ \emph{controlled-not} (\coqe{CNOT}) gates. Applying a gate to a quantum datum \emph{evolves} it.
% 
Its traditional semantics is expressed by multiplying the datum's vector form by the gate's corresponding matrix representation: $n$-qubit gates are $2^n$-by-$2^n$ matrices.
% 
Except for measurement gates, a gate's matrix must be \emph{unitary} and thus preserve appropriate invariants of quantum data's amplitudes. 
%
A \emph{measurement} operation extracts classical information from a quantum datum. It collapses the datum to a basis-ket with a probability related to the datum's amplitudes (\emph{measurement probability}), e.g., measuring $\frac{1}{\sqrt{2}}(\aket{0}{1} + \aket{1}{1})$ collapses the datum to $\aket{0}{1}$ or $\aket{1}{1}$, each with probability $\frac{1}{2}$. The ket values correspond to classical values $0$ or $1$, respectively. 
A more general form of quantum measurement is \emph{partial measurement}, which measures a subset of qubits in a qubit array;
% 
such operations often have simultaneity effects due to entanglement, \ie{} in a Bell pair $\frac{1}{\sqrt{2}}(\aket{00}{2} + \aket{11}{2})$, measuring one qubit guarantees the same outcome for the other --- if the first bit is measured as $0$, the second bit will be measured as $0$.

\noindent\textbf{\textit{Quantum Oracles.}} Quantum algorithms manipulate input information encoded in ``oracles'', which are callable black-box circuits. 
%For example, Grover's algorithm for unstructured quantum search \cite{grover1996,grover1997} is a general approach for searching a quantum ``database,'' which is encoded in an oracle for a function $f : \{0, 1\}^n \to \{0, 1\}$. Grover's algorithm finds an element $x \in \{0, 1\}^n$ such that $f(x) = 1$ using $O(2^{n/2})$ queries, a quadratic speedup over the best possible classical algorithm, which requires $\Omega(2^n)$ queries. 
Quantum oracles are usually quantum-reversible implementations of classical operations, especially arithmetic operations. Their behavior is defined in terms of transitions between single basis-kets.
We can infer the global state behavior based on the single basis-ket behavior through the quantum summation formula below. This resembles an array map operation in \Cref{fig:intros2}.
\oqasm in VQO~\cite{oracleoopsla} is a language that permits the definitions of quantum oracles with efficient verification and testing facilities using the following summation formula:
%The main approach in VQO is to utilize the summation formula to disallow the appearance of quantum entanglement state in analyzing programs,
%whereas \pqasm disregards if a state is entangled or a simple superposition without entanglement and utilizes the summation formula to reduce a quantum superposition state to a singleton basis-ket state to perform analysis.

{\footnotesize
  \begin{mathpar}
 \inferrule[]{ \forall j\,.\, x_j \longrightarrow f(x_j)  }{ \Sigma_j \alpha_j \ket{x_j} \longrightarrow \Sigma_j \alpha_j f(\ket{x_j})}
 % \inferrule[]{ \forall j\,.\, (x_j,y) \longrightarrow f(x_j,y) }{ \Sigma_j \alpha_j \ket{x_j}\ket{y} \longrightarrow \Sigma_j \alpha_j f(\ket{x_j}\ket{y})}
    \end{mathpar}
}

\noindent\textbf{\textit{Repeat-Until-Success Quantum Programs.}} A repeat-until-success program utilizes the probabilistic feature of partial measurement operations. We first set up a one-step repeat-until-success by linking the desired quantum state with the success measurement of a certain classical value. If such a value is observed after measurement, we know that the desired state is successfully prepared; otherwise, we repeat the one-step procedure.
One example of a one-step repeat-until-success procedure is in \Cref{fig:intros-example} to repeat the $n$ basis-ket superposition state.
If we measure out $v=1$, the desired state $\varphi$ is prepared; otherwise, we repeat the procedure.

\noindent\textbf{\textit{No Cloning.}} 
The \emph{no-cloning theorem} indicates no general way of exactly copying a quantum datum. In quantum circuits, this is related to ensuring the reversible property of unitary gate applications.
% 
For example, the controlled node and controlled body of a quantum control gate cannot refer to the same qubits, e.g., $\ictrl{q}{\iota}$ violates the property if $q$ is mentioned in $\iota$.
% 
\pqasm{} enforces no cloning through our type system.
              % background
%\input{overview}                % overview
%\vspace*{-0.5em}
\section{PQASM: An Assembly Language for Quantum State Preparations}
\label{sec:pqasm}
%\vspace*{-0.5em}

We designed \pqasm to express quantum state preparation programs in a high-level abstraction, facilitating the validation of these programs.
\pqasm operations leverage a quantum state design, with the type system to track the types of different qubits.
Such types restrict the kinds of quantum states, facilitating effective validation and analysis of the \pqasm program utilizing our quantum state representations.
%being effectively testable and simpler to analyze.
%\pqasm's type system tracks the bases of variables in \pqasm programs, forbidding operations that would introduce entanglement. 
%\pqasm states are therefore efficiently represented, so programs can be effectively tested and are simpler to analyze. 
%In addition, \pqasm uses \emph{virtual qubits} to support \emph{position shifting operations},
%which support arithmetic operations without introducing extra gates during translation. All of these features are novel to quantum assembly languages. 
This section presents \pqasm states and the language's syntax, semantics, typing, and soundness results.

As a running example, we program the $n$ basis-ket state preparation in \Cref{fig:intros-example} in \pqasm below.
The repeat-until-success program creates an qubit array $\overline{q}$ consisting of all zeroes and a new qubit $q'$, applies a Hadamard gate to each qubit in $\overline{q}$, uses a comparison operator $\qbool{\overline{q}}{<}{n}{q'}$ comparing $\overline{q}$ with $n$ and storing the result in $q'$, and measures the qubit $q'$.
If the measurement result is $1$, we stop the process, and $\overline{q}$ prepares the correct superposition state; otherwise, we repeat the process.
$\ihad{\overline{q}}$ is a syntactic sugar of sequence of Hadamard operations as $\iseq{\ihad{\overline{q}[0]}}{\iseq{...}{\ihad{\overline{q}[m\sminus 1]}}}$.

\begin{definition}[$n$ Basis-ket State Preparation Program]\label{def:circuit-example}\rm 
Example \pqasm program $P$ to prepare $n$ superposition state in $\overline{q}$, whose qubit array length is at least $\cn{log}(n)+1$; $q'$ is a single qubit.

{
$
P\triangleq\iseq{\inew{\overline{q}}}{\iseq{\iseq{\inew{q'}}{\ihad{\overline{q}}}}{\iseq{\qbool{\overline{q}}{<}{n}{q'}}{\smea{x}{q'}{\sifb{x=1}{\sskip}{P}}}}}
$
}
\end{definition}

\subsection{\pqasm States} \label{sec:pqasm-states}

\begin{figure}[t]
\vspace*{-0.8em}
{\small
  \[\hspace*{-0.5em}
  \begin{array}{l}
  \begin{array}{l@{\;\;}c@{\;}c@{\;}l@{\qquad}l@{\;\;}c@{\;}c@{\;}l@{\;}@{\qquad}l@{\;\;}c@{\;}c@{\;}l@{\;}}
        \text{Qubit Name}  & q && & \text{Nat} & n,m & \in & \mathbb{N} &      \text{Real} & r & \in & \mathbb{R}\\[0.2em]
        \text{Complex} & z & \in & \mathbb{C} & \text{Bit} & b & \in & \{0,1\} & \text{Bitstring} & c & ::= & \overline{b}
  \end{array}\\[1em]
\begin{array}{l c c l@{\;}c@{\;}l@{\;}c@{\;}l}
      \text{Qubit Basis State} & \nu & ::= & \aket{b}{1} & & &\mid &\qket{r} \\
       \text{Qubit Records} & \theta & ::= & (\overline{q} &, & \overline{q} &, & \overline{q}) \\
      \text{Type} & \tau & ::= & \thadt & \mid& \tnort & \mid& \trott\\
      \text{Basis Vector} & \eta & ::= &  \Motimes_j\nu_j  \\
      \text{Basis-Ket} & \rho & ::= &  z\cdot \eta  \\
      \text{Quantum Data} & \varphi & ::= & \rho &\mid& \sum_{b=0}^1 \varphi \\
      \text{Quantum State} & \Phi & ::= & \theta \to \varphi
    \end{array}
    \end{array}
  \]
  \vspace*{-1em}
  \caption{\pqasm state syntax. $\overline{S}$ denotes a sequence of $S$. $\aket{c}{n\splus 1} \equiv \aket{c[0]}{1}\otimes...\otimes\aket{c[n]}{1}$, with $\slen{c}=n\splus 1$.}
  \label{fig:pqasm-state}
}
\vspace*{-1em}
\end{figure}

A \pqasm program state is represented according to the grammar in \Cref{fig:pqasm-state}.
A quantum state is managed in terms of qubit records, each of which is a collection of qubits possibly being entangled, while qubits in different records are guaranteed to have no entanglement. A state $\Phi$ maps from qubit records $\theta$ to a quantum datum $\varphi$.
Our quantum data consists of a quantum entanglement state that can be analyzed as two portions: 1) a sequence of sum operators $\sum_{b_1=0}^1...\sum_{b_n=0}^1$, and 2) a basis-ket $\rho$, a pair of a complex amplitude $z$ and a tensor product of basis vector $\eta$, which is a tensor of single qubit basis states $\nu$.
Each sum operator represents the creation of a superposition state via a Hadamard operation $\cn{H}$, i.e., the number of sum operators in a state for a qubit record $\theta$ represents the number of Hadamard operations applied to qubits in the state so far.
The variable $\rho=z\cdot \eta$ represents a basis-ket of a quantum state.
To understand the relation between a basis-ket and a whole quantum superposition state connected with a sequence of sum operators,
one can think of a superposition state as a collection of "quantum choices", and a basis-ket represents a possible choice, i.e., a measurement of a qubit record produces one possible choice, with the amplitude $z$ related to the probability of the choice.
%We utilize the two-portion view to perform symbolic execution and abstract interpretation in \Cref{sec:implementation}.

A qubit basis state $\nu$ has one of two forms, $\aket{b}{1}$ and $\qket{r}$.
The former corresponds to the two types $\thadt$ and $\tnort$, and the latter corresponds to the $\trott$ type.
The three types of qubit basis states are represented as the three fields in a qubit record,
i.e., $(\overline{q}_1, \overline{q}_2, \overline{q}_3)$ has three disjoint qubit sequences.
$\overline{q}_1$ is always typed as $\thadt$, $\overline{q}_2$ has type $\tnort$, and $\overline{q}_3$ has type $\trott$.
The $\thadt$ and $\tnort$ typed qubits are in the computational basis.
The $\trott$ typed basis state is different from the other types in terms of \emph{bases}, and it has the form $\qket{r} = \cn{cos}(r)\aket{0}{1}+\cn{sin}(r)\aket{1}{1}$, which is a basis state in the Hadamard basis with $Y$-axis rotations.
Applying a $\cn{Ry}$ with the $Y$-axis angle $r$ to a $\aket{0}{1}$ qubit results in $\qket{r}=\cn{cos}(r)\ket{0}+\cn{sin}(r)\ket{1}$.
 
\subsection{PQASM Syntax}\label{sec:pqasm-syn}

\begin{figure}[h]
\vspace*{-0.5em}
{\small 
\begin{center}
  $ \hspace*{-0.8em}
\begin{array}{l}
      \;\;\text{Classical Variable}~x,y \qquad\qquad \text{Boolean Expressions}~B\\
\begin{array}{llcl}
      \text{Parameters} & \alpha & ::= & \overline{q} \mid n\\
      \text{OQASM Arithemtic Ops} & \mu & ::= & \iadd{\alpha}{\alpha} \mid \modmult{n}{\alpha}{m} \mid \qbool{\alpha}{=}{\alpha}{q} \mid \qbool{\alpha}{<}{\alpha}{q}\mid ...\\
      \text{Instruction} & \iota & ::= & \mu \mid \iry{r}{q} \mid \ictrl{q}{\iota} \mid \iseq{\iota}{\iota}\\
      \text{Program} & e & ::= & \iota \mid \iseq{e}{e} \mid \ihad{q} \mid \inew{{q}} \mid \smea{x}{\overline{q}}{e} \mid \sifb{B}{e}{e}
\end{array}
    \end{array}
  $
  \end{center}
}
\vspace*{-0.5em}
  \caption{\pqasm syntax.}
  \label{fig:pqasm}
  \vspace*{-0.5em}
\end{figure}

\Cref{fig:pqasm} presents \pqasm's syntax.  A \pqasm program $e$ is either an instruction $\iota$, a sequence operation $\iseq{e}{e}$, applying a Hadamard operation $\ihad{q}$ to a qubit $q$ to create a superposition, creating ($\inew{q}$) a new blank qubit $q$, a \cn{let} binding that measures a sequence of qubits $\overline{q}$ and uses the result $x$ in $e$, or classical conditional $\sifb{B}{e}{e}$ with classical Boolean guard $B$.
Each \cn{let} binding assigns the measurement result of a quantum function expression to a variable $x$, representing a binary sequence.
In measurement operations ($\mathpzc{M}$), we apply an operator to a \emph{qubit} $q$ or a sequence of quantum qubits $\overline{q}$.
We assume $\ihad{\overline{q}}$ and $\inew{\overline{q}}$ as syntactic sugars of applying a sequence of Hadamard and new-qubit operations.

The instructions $\iota$ correspond to unitary quantum circuit operations, including unitary oracle arithemtic operations ($\mu$) implementable through \oqasm operations \cite{oracleoopsla} on a qubit sequence $\overline{q}$ (detailed in Appendix, and it permits $Z$-axis rotation gates), a $Y$-axis rotation gate $\iry{r}{q}$ that rotates an angle $r$, a quantum control instruction ($\ictrl{q}{\instr}$), and a sequence operation ($\iseq{\iota}{\iota}$). Operation $\ictrl{q}{\instr}$ applies instruction $\instr$ \emph{controlled} on qubit $q$. 
%
In this paper, we provide several sample arithmetic oracle operations $\mu$ in \Cref{fig:pqasm}, such as addition ($\iadd{\alpha}{\alpha}$, adding the first to the second), modular multiplication ($\modmult{n}{\alpha}{m}$), quantum equality ($\qbool{\alpha}{=}{\alpha}{q}$), quantum comparison ($\qbool{\alpha}{<}{\alpha}{q}$), etc.
Each parameter $\alpha$ is either a group of qubits $\overline{q}$ or a number $n$.
Recall that a basis-ket state of a qubit array $\overline{q}$ is essentially a bitstring with a complex amplitude.
As in the summation formula in \Cref{sec:background}, a quantum arithmetic operation applies the classical version of the operation to each basis-ket in a quantum superposition state, e.g., $\qbool{\overline{q}}{=}{n}{q}$ compares the bitstring representation of each basis-ket in $\overline{q}$ with the number $n$ and stores the result in $q$.

In a \pqasm program containing qubit array $\overline{q}$, $x$ in a \cn{let} binding  binds a local classical value ---
we bind $x$'s value with the computational basis measurement result ($\mathpzc{M}$) on qubits $\overline{q}$. While the classical variable scope is local, the quantum qubits are immutable and globally scoped, i.e., quantum operations are applied to a global quantum state; each qubit in the state is referred to by quantum qubit names ($q$) in the program.
In \pqasm, we express a SKIP operation ($\iskip$) via a $\mu$ operation having empty qubits, as $
\mu \; \equiv\; \iskip\;\;\cn{when}\;\;FV(\mu)=\emptyset
$ ($FV$ collects free variables).

%In \pqasm, we assume that an instruction is equivalent to a SKIP operation ($\iskip$), if its qubit number is $0$, $FV(\mu)=\emptyset$.
%In the \pqasm instruction level, one can utilize \oqasm oracle function $\mu$ to express different kinds of unitary gates, such as SKIP ($\iskip$), $\cn{X}$, and $Z$-axis rotation gates. Here, $FV(\mu)=\emptyset$ (free variables in $\mu$) makes $\mu$ become a SKIP operation $\iskip$.

\subsection{Semantics}\label{sec:pqasm-dsem}

\begin{figure}[t]
\vspace*{-0.5em}
{\footnotesize
\[
\begin{array}{lll}

\llbracket \mu \rrbracket\eta &= \app{\llbracket \mu \rrbracket\eta(\overline{q})}{\eta}{\overline{q}}
&
\texttt{where  }
FV(\mu) = \overline{q}
\\[0.5em]

\llbracket \iry{r}{q} \rrbracket\eta &=  \app{\qket{r}}{\eta}{q}
&
\texttt{where  }
\eta(q) = \ket{0}{1}
\\[0.4em]

\llbracket \iry{r}{q} \rrbracket\eta &=  \app{\qket{\frac{3\pi}{2}- r}}{\eta}{q}
&
\texttt{where  }
\eta(q) = \ket{1}{1}
\\[0.4em]

\llbracket \iry{r}{q} \rrbracket\eta &=  \app{\qket{r+r'}}{\eta}{q}
&
\texttt{where  }
\eta(q) = \qket{r'}
\\[0.4em]

\llbracket \ictrl{q}{\instr} \rrbracket\eta &=  \csem(\eta(q),\instr,\eta)
&
\texttt{where  }
\csem({\ket{0}{1}},{\instr},\eta)=\eta\quad\;\,
\csem({\ket{1}{1}},{\instr},\eta)=\llbracket \instr \rrbracket\eta
\\[0.4em]

\llbracket \iota_1; \iota_2 \rrbracket\eta &= \llbracket \iota_2 \rrbracket (\llbracket \iota_1 \rrbracket\eta)
\end{array}
\]
}
{\footnotesize
\begin{center}
$
\app{\eta'}{\eta}{\overline{q}}=\app{\eta'(q)}{\eta}{\forall q\in \overline{q}.\;q}
$
\end{center}
}
\vspace*{-0.5em}
\caption{Instruction level \pqasm semantics; $\eta(\overline{q})$ restricts the qubit states $\overline{q}$ in $\eta$.}
  \label{fig:deno-sem}
  \vspace*{-1em}
\end{figure}

\pqasm has two levels of semantics: instruction and program levels.
The \textit{instruction} level semantics is a partial function $\llbracket - \rrbracket$ from an instruction $\instr$ and input basis vector state $\rho$ to an output state $\eta'$, written 
$\llbracket \instr \rrbracket\eta=\eta'$, shown in \Cref{fig:deno-sem}.
The \textit{program} level semantics is a labelled transition system $(\Phi,e) \xrightarrow{r} (\Phi',e')$ in \Cref{fig:exp-semantics}, stating that the input configuration $(\Phi,e)$ is possibly evaluated to an output configuration $(\Phi',e')$ with the probability $r$.
It essentially represents a Markov chain, where a program evaluation path represents a chain of probabilities, showing the probability path leads to a particular configuration from the initial configuration.

In the instruction level semantics, the semantic rule description assumes that one can locate a qubit state $q$ in $\eta$ as $\eta(q)$, where we can refer to $\app{\nu}{\eta}{q}$ as updating the qubit state $\nu$ for the qubit $q$ in $\eta$.
Recall that a length $n$ basis vector state $\eta$ is a tuple of $n$ qubit values, modeling the tensor product $\nu_1\otimes \cdots \otimes \nu_n$. 
The rules implicitly map each qubit variable $q$ to a qubit value position in the state, e.g., 
$\eta(q)$ corresponds to some sub-state $\nu_q$, where $\nu_q$ locates at the $q$'s position in $\eta$.
%
Many of the rules in \Cref{fig:deno-sem} update a \emph{portion} of a state. We write $\app{\nu_{q}}{\eta}{q}$ to update the qubit value of $q$ in $\eta$ with $\nu_q$, and
$\app{\eta'}{\eta}{\overline{q}}$ to update a range of qubits $\overline{q}$ according to the vector state $\eta'$, i.e., we update each $q\in\overline{q}$ with the qubit value $\eta'(q)$ and $\slen{\eta'}=\slen{\overline{q}}$.
The function \texttt{cu} is a conditional operation depending on the $\tnort$/$\thadt$ typed qubit $q$. 

\begin{figure*}[t]
\vspace*{-0.5em}
{\footnotesize
  \begin{mathpar}
      \inferrule[S-Ins]{b = b_1,...,b_n\\ \varphi = \sum_{b_1=0}^1...\sum_{b_n=0}^1 z_b\cdot\eta_b \\\llbracket \iota \rrbracket (\eta_b)=\eta'_b}{ (\Phi[\theta\mapsto \varphi],\iota) \xrightarrow{1}  (\Phi[\theta\mapsto \sum_{b_1=0}^1...\sum_{b_n=0}^1 z_b\cdot\eta'_b], \sskip)}

        \inferrule[S-SeqC]{(\Phi,e_1) \xrightarrow{r} (\Phi', e'_1) }
        {(\Phi,\sseq{e_1}{e_2}) \xrightarrow{r} (\Phi',\sseq{e'_1}{e_2}) }

        \inferrule[S-SeqT]{}
        {(\Phi,\sseq{\sskip}{e_2}) \xrightarrow{1} (\Phi,e_2) }
        
        \inferrule[S-New]{}
        {(\Phi,\inew{q}) \xrightarrow{1} (\app{\aket{0}{1}}{\Phi}{(\emptyset,q,\emptyset)},\sskip)}

      \inferrule[S-IfT]{}{ (\Phi,\sifb{\cn{true}}{e_1}{e_2}) \xrightarrow{1} (\Phi,e_1)}
 
       \inferrule[S-IfF]{}{ (\Phi,\sifb{\cn{false}}{e_1}{e_2}) \xrightarrow{1} (\Phi,e_2)}
        
        \inferrule[S-Had]{}{ (\Phi[(\emptyset,q,\emptyset)\mapsto \aket{b}{1}],\ihad{q}) \xrightarrow{1} (\app{\sum_{j=0}^1 (\sminus 1)^{j\cdot b}\aket{j}{m}}{\Phi}{(q,\emptyset,\emptyset)}, \sskip) }
             
           \inferrule[S-Mea]{\Phi = \Phi' \uplus \{\uparrow\overline{q} \mapsto \Msum_j z_j\aket{c}{m}{\aket{c_j}{n}}+\qfun{\phi}{\overline{q}',c \neq \overline{q}'}\}  \\ r= \Msum_j \slen{z_j}^2 }
  {(\Phi,\smea{x}{\overline{q}}{e}) \xrightarrow{r} (\Phi'\uplus \{ (\uparrow\overline{q}) \textbackslash \overline{q} :  \Msum_j {\frac{z_j}{\sqrt{r}}}{\aket{c_j}{m}}\},e[c/x]) }
  \end{mathpar}
}
{\footnotesize
\begin{center}
$
\begin{array}{lcl}
\uparrow\overline{q}&\triangleq& \exists \overline{q}_1, \overline{q}_2,\overline{q}_3 \,.\,\uparrow\overline{q}=(\overline{q}_1, \overline{q}_2,\overline{q}_3)\wedge \overline{q} \subseteq \overline{q}_1 \uplus \overline{q}_2 \uplus \overline{q}_3
\\[0.2em]
\qfun{(\Msum_{i}{z_i}{\aket{c_{i}}{m}}{\aket{c'_i}{n}}+\varphi)}{\overline{q},b} &\triangleq& \Msum_{i}{z_i}{\aket{c_{i}}{m}}{\aket{c'_i}{n}}
      \qquad\qquad\texttt{where}\quad\forall i.\,\slen{c_{i}}=\slen{\overline{q}'}=m\wedge \denote{b[c_{i}/\overline{q}']}=\texttt{true}
\end{array}
$
\end{center}
}
\vspace*{-0.5em}
\caption{Program Level \pqasm semantics.}
\label{fig:exp-semantics}
\vspace*{-1em}
\end{figure*}


\Cref{fig:exp-semantics} shows the program level semantics.
$\Phi$ is the quantum state mapping from qubit records to superposition state values.
Since qubit records in $\Phi$ partition the qubit domain, we can think of $\Phi$ as a multiset of pairs of qubit records and state values, as in \rulelab{S-Mea},
i.e., $\Phi[\theta \mapsto \varphi] \equiv \Phi \uplus \{\theta\mapsto \varphi\}$.
Rule \rulelab{S-Ins} connects the instruction level semantics with the program level by evaluating each basis vector state $\eta$ through the instruction $\iota$.
Rule \rulelab{S-New} creates a new blank ($\aket{0}{1}$) qubit, which is stored in the record $(\emptyset,q, \emptyset)$, a qubit ($q$) being created are $\tnort$ typed.
For a $\tnort$ typed qubit $(\emptyset, q, \emptyset)$, rule \rulelab{S-Had} turns the qubit to be $\thadt$ typed superposition, as $(q,\emptyset, \emptyset)$.

%These rules implicitly map each qubit variable $q$ to a qubit value position in the state, e.g., 
%$\eta(q)$ corresponds to some sub-state $\nu_q$, where $\nu_q$ locates at the $q$'s position in $\eta$.
%
%Many of the rules in \Cref{fig:deno-sem} update a \emph{portion} of a
%state. We write $\app{\nu_{q}}{\eta}{q}$ to update the qubit value of $q$ in $\eta$ with $\nu_q$, and
%$\app{\nu_{\overline{q}}}{\eta}{\overline{q}}$ to update a range of qubits $\overline{q}$ according to the qubit \emph{tuple} $\nu_{\overline{q}}$.
Rules \rulelab{S-IfT} and \rulelab{S-IfF} perform classical conditional evaluation.
In \pqasm, the classical variables are evaluated via a substitution-based approach, as in \rulelab{S-Mea}.
The measurement rule (\rulelab{S-Mea}) produces a probability $r$ label, and the value comes from the measurement result.
We first rewrite the quantum state to be a linear sum of computational basis-kets $\Msum_j r_j\aket{c}{m}{\aket{c_j}{n}}+\qfun{\varphi}{\overline{q}',c \neq \overline{q}'}$,
where every basis-vector $\aket{c}{m}$ (or $\aket{c_j}{n}$) is a bitstring, and all the sum operators are resolved as a single sum.

Any \pqasm state can be written as a sum of computational basis-kets.
As the equations shown below, the basis-ket state $\aket{c}{n}\qket{r}$ can be rewritten to be a sum of two computational basis-kets as $\cn{cos}(r)\aket{c}{n}\aket{0}{1}+\cn{sin}(r)\aket{c}{n}\aket{1}{1}$, while the two sum operators can be replaced as a single sum operator over length-$2$ bitstring $c$, where we replace $b_j$ with $c[0]$ (indexing $0$ of $c$) and $b_k$ with $c[1]$.

{\small
\begin{center}
$
\aket{c}{n}\qket{r} \equiv \cn{cos}(r)\aket{c}{n}\aket{0}{1}+\cn{sin}(r)\aket{c}{n}\aket{1}{1}
\qquad
\sum_{b_j=0}^1 \sum_{b_k=0}^1 \eta \equiv \sum_{c\in\{0,1\}^2} \eta[c[0]/b_j][c[1]/b_k]
$
\end{center}
}

% Several takeaways about \pqasm denotational semantics.
% For any operation application within the space domain $\hsp{S}^d$, the semantic application $U$ only has effect on the specific qubit ($\varphi_{(x,n)}$) / qubit array ($\varphi_{x}$) that it targets at, which does not create entanglement with other subsystems.
% This clear separation only works for the domain $\hsp{S}^d$.
% When we compile these operations to \sqir and see their effects on a general Hilbert space $\hsp{H}$, they might have entanglement effects.
% \yxp{Even if we turn it into unitary over the Hilbert space, it still does not generate entanglement with other subsystems.}
% \liyi{Can you have CNOT x y when you have x is Had and y is in Nor, then you will definitely have entanglement. }
% However, the clear separation in $\hsp{S}^d$ provides us a decompositional and analytical way of verifying and validating quantum oracles; thus, each sub-oracle-component can be analyzed individually. The potential entanglements in a general Hilbert space becomes the naturally extended (additive) superposition effects.
% In addition, all semantic functions in Fig.~\ref{fig:deno-sem} are carefully engineered to only target qubits in a register $\varphi$, and does not target on invidual vectors in the vector space $\varphi$ represents.
% For example, $\xsem$ is defined for a basis phase space case $\ket{c}$, and we also define the case for superposition $\frac{1}{\sqrt{2}}(\ket{0}+(-1)^c\ket{1})$. We do not assume the the semantics of the basis phase space is automatically extended to dealing with individual elements in the superposition case.
% By using the semantics to prove quantum oracle properties, we only need to consider $O(n)$ qubits instead of the possible $2^n$ expanded vector elements.
% The semantics of a universal quantum assembly language like \sqir, by contrast, represents a quantum state as a unitary matrix whose size is \emph{exponential} in the number of vectors by expanding qubits to vectors in a register. \sqir's semantics also relies on the use of concrete qubits; using a unitary matrix and virtual positions would inject a virtual-to-physical mapping into the semantic definition, which can severely complicate proofs~\cite{PQPC}. This leads to the successful correctness proof of the QFT-adder for the first time (Sec.~\ref{sec:op-verification}).
% We only define semantic functions for qubit forms when it is possible to apply. For example, we do not define $\xsem$ for the form $\frac{1}{\sqrt{2}}(\ket{0}+e^{2\pi{i} b}\ket{1})$, because the \pqasm type system does not allow it. 

\subsection{Typing}
\label{sec:pqasm-typing}

In \pqasm, typing is with respect to a \emph{type environment} $\Omega$, a set of qubit records partitioning qubits into different disjoint union regions, and a \emph{kind environment} $\Sigma$, a set tracking local variable scopes.
Typing judgments are two leveled, and are written as $\Omega\vdash \instr \triangleright_g \Omega'$ and $\Sigma; \Omega\vdash e\triangleright \Omega'$,
 which state that instruction $\instr$ and program expression $e$ are well-typed under environments $\Omega$ and $\Sigma$, and
transforms variables' bases as characterized by type environment $\Omega'$.
$\Omega$ is populated via qubit creation operations (\cn{new}), while $\Sigma$ is populated via \cn{let} binding.
Typing rules are in \Cref{fig:exp-well-typed}.


\begin{figure}[t]
\vspace*{-0.5em}
{\footnotesize
  \begin{mathpar}
 
      \inferrule[Eqv]{\Omega\equiv \Omega'\\\Omega' \vdash_g e\triangleright \Omega''}{ \Omega\vdash_g e\triangleright\Omega''}

     \inferrule[RyN]{}{\{(\emptyset,\{q\},\emptyset)\} \uplus \Omega\vdash_{\cn{C}} \iry{r}{q}\triangleright \{(\emptyset,\emptyset,\{q\})\}\uplus \Omega}
     
     \inferrule[RyH]{\trot{\theta}=\{q\} \uplus \overline{q}}{\{\theta\}\uplus \Omega \vdash_{g}  \iry{r}{q}\triangleright \{\theta\}\uplus \Omega}

     \inferrule[MuT]{\overline{q} \subseteq \overline{q_1}\cup \overline{q_2}}{\{(\overline{q_1},\overline{q_2},\overline{q_3})\}\uplus \Omega \vdash_g \mu(\overline{q})\triangleright \{(\overline{q_1},\overline{q_2},\overline{q_3})\}\uplus \Omega}

    \inferrule[CuN]{\{\tnor{\theta}\downarrow \overline{q}\} \uplus \Omega \vdash_{\cn{M}} \iota \triangleright \{\tnor{\theta}\downarrow \overline{q}\}\uplus \Omega}{\{\tnor{\theta}\downarrow \{q\}\uplus\overline{q}\} \uplus \Omega \vdash_g \ictrl{q}{\iota} \triangleright \{\tnor{\theta}\downarrow \{q\}\uplus\overline{q}\} \uplus \Omega} 

    \inferrule[CuH]{\{\thad{\theta}\downarrow \overline{q}\} \uplus \Omega \vdash_{\cn{M}} \iota \triangleright \{\thad{\theta}\downarrow \overline{q}\} \uplus \Omega}{\{\thad{\theta}\downarrow \{q\}\uplus\overline{q}\} \uplus \Omega \vdash_g \ictrl{q}{\iota} \triangleright \{\thad{\theta}\downarrow \{q\}\uplus\overline{q}\} \uplus \Omega} 

     \inferrule[Seq]{\Omega\vdash_g \instr_1\triangleright \Omega' \\ \Omega'\vdash_g \instr_2\triangleright \Omega''}{\Omega \vdash_g \iseq{\instr_1}{\instr_2}\triangleright \Omega''} 

       \inferrule[ESeq]{\Sigma;\Omega\vdash e_1\triangleright \Omega' \\ \Sigma;\Omega'\vdash e_2\triangleright \Omega''}{\Sigma;\Omega \vdash \iseq{e_1}{e_2}\triangleright \Omega''} 
      
      \inferrule[New]{q\not\in \Omega}{\Sigma;\Omega \vdash \inew{{q}}\triangleright \Omega \uplus \{(\emptyset,{q},\emptyset)\}} 

    \inferrule[HT]{}{\Sigma;\Omega\uplus\{(\emptyset,q,\emptyset)\} \vdash \ihad{q}\triangleright \Omega\uplus\{(q,\emptyset,\emptyset)\}} 
      
      \inferrule[Tup]{\Omega\vdash_{\cn{C}} \instr\triangleright \Omega'}{\Sigma;\Omega \vdash \iota\triangleright \Omega'} 
               
      \inferrule[Mea]{\overline{q}\subseteq \theta \\ \Sigma\cup\{x\};\Omega \uplus\{\theta\textbackslash \overline{q}\} \vdash e \triangleright \Omega'}{\Sigma;\Omega \uplus\{\theta\} \vdash \smea{x}{\overline{q}}{e}\triangleright \Omega'} 
      
      \inferrule[TIf]{\Sigma \vdash B \\ \Sigma;\Omega\vdash e_1 \triangleright \Omega'\\\Sigma;\Omega\vdash e_2 \triangleright \Omega'}{\Sigma;\Omega\vdash \sifb{B}{e_1}{e_2} \triangleright \Omega'} 
  \end{mathpar}
}
\vspace*{-0.8em}
  \caption{\pqasm typing rules. $(\overline{q}_1,\overline{q}_2,\overline{q}_3) \textbackslash \overline{q} \triangleq (\overline{q}_1 \textbackslash\, \overline{q},\overline{q}_2 \textbackslash\, \overline{q},\overline{q}_3 \textbackslash\, \overline{q})$, where $\overline{q}_1 \textbackslash\, \overline{q}$ is set subtraction. }
  \label{fig:exp-well-typed}
  \vspace*{-1em}
\end{figure}

The instruction level type system is flow-sensitive, where $g$ is the context flag and can be either $\cn{M}$ or $\cn{C}$, indicating whether the current instruction is inside a controlled operation. The program level type system communicates with the instruction level by assuming a \cn{C} mode context flag, shown in rule \rulelab{Tup}.
We explain the necessity of the context flag below.
Each qubit record $\theta$ represents an entanglement group, i.e., qubits in the same record might or might not be entangled, while qubits in different records are ensured not to be entangled.

{\footnotesize
\begin{center}
$
\begin{array}{l@{\;}c@{\;}l}
(\overline{q_1},\overline{q_2},\overline{q_3})\uplus(\overline{q_4},\overline{q_5},\overline{q_6}) &\equiv& 
(\overline{q_1}\uplus\overline{q_4},\overline{q_2}\uplus \overline{q_5},\overline{q_3}\uplus \overline{q_6})
\\[0.2em]
(\emptyset,\overline{q_1}\uplus \overline{q_2},\overline{q_3}\uplus \overline{q_4}) &\equiv& (\emptyset,\overline{q_1},\overline{q_3})\uplus (\emptyset,\overline{q_2},\overline{q_4})
\end{array}
$
\end{center}
}

In our type system, we permit ordered equational rewrites among quantum qubit states.
Each type environment, mainly the operation $\uplus$, admits associativity, commutativity, and identity equational properties.
The $\uplus$ operations in the three fields in a qubit record also admit the three properties.
Other than the three equational properties, we admit the above partial order relations, where we permit the rewrites from left to right in our type system to permit qubit records merging and splitting.
Record merging can always happen, i.e., two qubit entanglement groups can be merged into one.
Qubit splitting cannot occur in $\thadt$ typed qubits.
A qubit record, including a $\thadt$ typed qubit, represents a quantum entanglement with qubits not separable,
while qubits being $\tnort$ and $\trott$ typed can be split into different records.
Rule \rulelab{Eqv} imposes the equivalence relation to permit the rewrites, only allowing rewrites from left to right, of equivalent qubit records.
Note that a type environment determines qubit record scopes in a quantum state $\varphi$ (\Cref{fig:pqasm-state}),
i.e., a quantum state should have the same qubit record domain as the type environment at a program point,
so the equational rewrite of a type environment might affect the qubit state representation.

Other than the equational rewrites, the type system enforces three properties. 
First, it enforces that classical and quantum variables are properly scoped.
%expressions and instructions are well-formed,
%meaning that classical and quantum variables are locally scoped.
Rule \rulelab{Mea} includes the local variable $x$ in $\Sigma$, while rule \rulelab{TIf} ensures that any variables mentioned in $B$ are properly scoped by the constraint $FV(B)\subseteq \Sigma$ \footnote{The rule is parameterized by different $B$ formalism.}, i.e., all free variables in $B$ are bounded by $\Sigma$.
Note that the two branches of \rulelab{TIf} have the same output $\Sigma'$, i.e., the qubit manipulations in the two branches need to have the same effect.
If one branch has a measurement on a qubit, the other branch must have the same qubit measurement.

Rule \rulelab{New} creates a new record $(\emptyset,q,\emptyset)$ in the post-environment, provided that $q$ does not appear in any record in $\Omega$.
In rule \rulelab{RyH}, the premise $\trot{\theta}=\{q\} \uplus \overline{q}$ utilizes $\trott$ to finds the $\trott$ filed in the record $\theta$ and ensures that $q$ is in the field. In rule \rulelab{CuH}, the premise $\thad{\theta}\downarrow \{q\}\uplus\overline{q}$ ensures that the controlled position $q$ is in the $\thadt$ field in $\theta$.
In rule \rulelab{MuT}, we ensure that the qubits $\overline{q}$ being applied by the $\mu$ operation are $\tnort$ and $\thadt$ typed, through the premise $\overline{q} \subseteq \overline{q_1}\cup \overline{q_2}$.

Second, we ensure that expressions and instructions are well-formed, i.e., any control qubit is distinct from the target(s), 
which enforces the quantum \emph{no-cloning rule}.
In rules \rulelab{CuH} and \rulelab{CuN} for control operations, when typing the target instruction $\iota$ (the upper level), we remove the control qubit $q$ in the records to ensure that $q$ cannot be mentioned in $\iota$.
In rule \rulelab{Mea}, we also remove the measured qubits $\overline{q}$ from the record $\theta$.

Third, the type system enforces that expressions and instructions leave affected qubits in a proper type ($\tnort$, $\thadt$, and $\trott$), representing certain forms of qubit states, mentioned in \Cref{fig:pqasm-state}; therefore, one can utilize the procedure mentioned in \Cref{sec:intro} to analyze \pqasm programs effectively.
The key is to utilize the summation formula to reduce the analysis of a general quantum state to that of a quantum state without entanglement.
Specifically, the $\iry{r}{q}$ opeartion is permitted only if $q$ is of $\tnort$ type, which is turned to $\trott$ type and stays there;
$\mu$ can be applied to $\tnort$ typed qubits $\overline{q}$ where $FV(\mu)=\overline{q}$; and a control qubit $q$ in $\ictrl{q}{\iota}$ can be applied to a $\tnort$ and $\thadt$ typed qubit.

We ensure type restrictions for qubits via pre- and post-type environments.
Rule \rulelab{HT} permits the generation of $\thadt$ type qubits, a.k.a. superposition qubits $q$, provided that $q$ is of $\tnort$ type and not entangled with other qubits. Once a Hadamard operation is applied, we turn the qubit types to $\thadt$ in the post-type environment, so one cannot apply Hadamard operations again to the qubits.
This does not mean that users can only apply Hadamard operations once in \pqasm, because combining \cn{X} (our oracle operation) and \cn{Ry} gates can produce a Hadamard gate. We utilize the type information to locate the first appearance of Hadamard operations, identify them as the source of superposition, and apply treatments for them in our validation testing framework; see \Cref{sec:rand-testing}.

\begin{wrapfigure}{r}{3cm}
{\small
$\begin{array}{c}
  \Qcircuit @C=0.5em @R=0.5em {
    &                     \qw & \ctrl{1} &  \qw \\
    &                     \gate{Ry(0)}  & \gate{Ry(r)}  & \qw    
    }
    \end{array}
$
}
\caption{Ensuring qubits inside a controlled \cn{Ry} have the same type.}
%\vspace*{-1em}
\label{fig:rygate}
\end{wrapfigure}

In controlled operations (rules \rulelab{CuN} and \rulelab{CuH}), we ensure that the pre- and post-type environments are the same.
In $\ictrl{q}{\iota}$, if $\iota$ contains a \cn{Ry} operation, applying to a qubit $q$, $q$ must already be $\trott$ type.
\Cref{fig:rygate} provides a programming prototype satisfying this type requirement,
where programmers explicitly add a \cn{Ry} gate before the controlled \cn{Ry} operation to ensure the second qubit is in $\trott$ type.
The extra \cn{Ry} operation can be a $0$ rotation, equivalent to a SKIP operation, and can be removed by an optimizer when compiling to quantum circuits.
%
The above qubit type restriction does not depend on the applications of controlled operations.
We ensure this by associating the context flags with the instruction level type system.
When applying a $\ictrl{q}{\iota}$ operation, rules \rulelab{CuN} and \rulelab{CuH} turn the context flag to $\cn{M}$, indicating that $\iota$ lives inside a controlled operation.
Rule \rulelab{RyN} requires a context flag \cn{C}, meaning that the rule is valid only if the \cn{Ry} operation lives outside any controlled operation.
In contrast, rule \rulelab{RyH} does not require a specific context flag, which indicates that a \cn{Ry} operation inside a controlled node must apply to a qubit already in $\trott$ type.


% \begin{figure}[t]
% {\footnotesize
% \begin{center}
% \begin{tikzpicture}[->,>=stealth',shorten >=1pt,auto,node distance=3.2cm,
%                     semithick]
%   \tikzstyle{every state}=[fill=white,draw=black,text=black]
% 
%   \node[initial,accepting,state] (A)              {$\texttt{OK}$};
%   \node[state]         (B) [right of=A] {$ $};
% 
%   \path (A) edge [loop above]            node {$b,\epsilon / \epsilon$} (A)
%             edge  [above] node {$a,\emptyset / a$} (B);
%   \path (B) edge [loop right]            node [right] {$\begin{array}{l}b,\epsilon / \epsilon\\
%                                                                 a,a' / a a'\\
%                                                                 a,\overline{a} / \epsilon\\
%                                                  \end{array}$} (B)
%             edge  [bend left]             node [above] {$\epsilon,\emptyset / \emptyset$} (A);
% \end{tikzpicture}
% \end{center}
% }
% {
% \footnotesize
% $
% \begin{array}{l}
% a,a'\in \{\ilshift{x},\irshift{x},\irev{x} \} \wedge a' \neq \overline{a}
% \\
% \overline{\ilshift{x}}=\irshift{x}
% \quad
% \overline{\irshift{x}}=\ilshift{x}
% \quad
% \overline{\irev{x}}=\irev{x}
% \\
% b\not\in\{\ilshift{x},\irshift{x},\irev{x}, \instr;\instr \}
% \\
% \emptyset=\text{ no element in stack}
% \end{array}
% $
% }
% 
% \caption{Pushdown automata for \texttt{neutral}}
% \label{fig:pushdown-neu}
% \end{figure}
% \texttt{neutral}'s definition in \Cref{fig:pushdown-neu}
% views $\instr$ as a string concatenated
% by the sequence operation ($;$) and requires $\instr$ to be
% accepted according to a family of pushdown automatas $\{G\}_{x}$ for every $x$ presented in $\instr$. 
% A program $\instr$ is \texttt{neutral}, iff, $\instr$ as a string is
% accepted by all the automatas in $\{G\}_{x}$.

\noindent\textbf{\textit{Soundness.}}
We prove that well-typed \pqasm programs are well defined; i.e., the
type system is sound with respect to the semantics. 
The type soundness theorem relies on a well-formed definition of a program $e$, $FV(e)\subseteq \Sigma$, meaning that all free variables in $e$ are bounded by $\Sigma$.
We also need the definition of the well-formedness of an \pqasm state as follows.

\begin{definition}[Well-formed \pqasm state]\label{def:well-formed}\rm 
  A state $\Phi$ is \emph{well-formed}, written
  $\Omega \vdash \Phi$, iff:
\begin{itemize}
\item For every $q$ such that $\Omegaty(q) = \tnort$ or $\Omegaty(q) = \thadt$ ,
  $\Phi(q)$ has the form $\aket{b}{1}$.

\item For every $q$ such that $\Omegaty(q) = \trott$, $\Phi(q)$ has the form $\qket{r}$.
\end{itemize}
\end{definition}

\noindent
Type soundness is stated as two theorems: type progress and preservation; the proof is by induction on $\instr$ and is mechanized in Rocq.

\begin{theorem}\label{thm:type-progress}\rm[\pqasm Type Progress]
If $\emptyset; \Omega \vdash e \triangleright \Omega'$, $FV(e)\subseteq \emptyset$, and $\Omega \vdash \Phi$, then either $e = \sskip$ or there exists $r$, $e'$, and $\Phi'$, such that $(\Phi,e)\xrightarrow{r}(\Phi',e')$.
\end{theorem}

\begin{proof}
Fully mechanized proofs were done by induction on type rules using Rocq.
\end{proof}

\begin{theorem}\label{thm:type-sound-pqasm}\rm[\pqasm Type Preservation]
If $\Sigma; \Omega \vdash e \triangleright \Omega'$, $FV(e)\subseteq \Sigma$, $\Omega \vdash \Phi$ and $(\Phi,e)\xrightarrow{r} (\Phi',e')$, then there exists $\Omega_a$, such that $\Sigma;\Omega_a\vdash e' \triangleright \Omega'$ and $\Omega_a \vdash \Phi'$.
\end{theorem}

\begin{proof}
Fully mechanized proofs were done by induction on type rules using Rocq.
\end{proof}
                   % formal developement
\section{QSV Applications}
\label{sec:implementation}

%\liyi{Symbolic execution, and compiler.}

This section presents QSV applications, based on \pqasm programs, to validate and compile the programs.
%where we view \pqasm as a framework for specifying, compiling, and testing quantum state preparation programs.
%whose architecture was given in \Cref{fig:arch}.
We start by discussing QSV's PBT framework for \pqasm programs. 
Next, we consider translation from \pqasm to \sqir and proof of its correctness. 
%Next, we discuss \pqasm's property-based random testing framework for \pqasm programs. 
%Finally, we discuss \vqimp, a simple imperative language for writing oracles, which compiles to \pqasm. We also present its proved-correct compiler and means to test the correctness of \vqimp oracles.

%the qubit state side-effects of adding a number to qubit state phases caused by an $\texttt{SR}$ gate
%We utilize the lemma below to show the qubit state side-effects of the phase addition caused by an $\texttt{SR}$ gate when we transform the state from \texttt{Phi} to \texttt{Nor}-basis by applying a  $\texttt{QFT}^{-1}$ gate.
%After such transformation, phase changes are transformed into changes in qubit state bases.
%\mwh{Don't understand the previous statement. The rzadder program never references SR-inverse. It does reference \texttt{Rev} but you never mention that.} \liyi{modified. }
%\mwh{The modification didn't help me. You still mention $\texttt{SR}^{\lbrack -1 \rbrack}$ but I don't know why; it appears nowhere in the code for this circuit.}
%The side-effect depends on the $m$ value of the state basis type $\texttt{Phi}\;m$.
%For the example in \Cref{fig:circuit-example}, the \texttt{QFT} gate 
%turns \code{b}'s basis to $\texttt{Phi}\;(\Sigma($\code{b}$))$, which is an exact QFT instead of an AQFT gate.
%In this case, the effect of applying an  $\texttt{SR}^{\lbrack -1 \rbrack}$ gate onto the \texttt{Nor}-basis state is as follows:
%\begin{lemma}\label{thm:true-meaning}\rm
%For a variable $x$ having type $\texttt{Phi}\;(\Sigma(x))$, let $y$ being the result \texttt{Nor}-basis state by applying $\texttt{QFT}^{-1}$ to $x$; thus, applying $\isr{n}{x}$ following by a $\texttt{QFT}^{-1}$ gate means to add $2^{\Sigma(x)-n-1}$ to $y$;
%and applying $\isr[-1]{n}{x}$ following by a $\texttt{QFT}^{-1}$ gate means to subtract $2^{\Sigma(x)-n-1}$ from $y$; .
%\end{lemma}
%\liyi{might not need the following.}
%In short, with the lemmas above, we first inductively show the effect of applying a series of $\texttt{SR}$ gates to variable %\texttt{b}'s qubit state phases as shown in the QFT-adder circuit \Cref{fig:circuit-example}.
%Then, we utilize \Cref{thm:true-meaning} to show the computation result of the phase manipulation transformation to \texttt{b}'s qubit state bases after applying a $\texttt{QFT}^{-1}$ gate.

\subsection{PQASM Typing for Effectively Validating State Preparation Programs}\label{sec:rand-testing}

%Full formal proof is the gold standard for correctness, but it is also laborious. It is especially deflating to be well through a proof only to discover that the intended property does not hold and, worse, that nontrivial changes to the program are necessary. 
The QSV validator is built on PBT to give assurance that a \pqasm program property is correct by attempting to falsify it using thousands of randomly generated instances with good coverage of the program's input space. We have used PBT to validate the correctness of a variety of state preparation programs, presented in \Cref{sec:evaluation}.
Below, we show our validator construction.

% Proofs of operator correctness can be time-consuming and repetitive.
\pqasm's state representation and type system ensure that states can be represented effectively.
We leverage this fact to implement a validation framework for \pqasm programs using QuickChick \cite{quickchick}, a property-based testing (PBT) framework for Rocq in the style of Haskell's QuickCheck~\cite{10.1145/351240.351266}. We use this framework for two purposes: to validate the correctness properties of \pqasm programs and to experiment with effective implementations of correct state preparation programs.
\pqasm contains measurement operations, which, due to the randomness inherent in quantum measurement, are difficult to test effectively, even with the assistance of program abstractions.
To have an effective validation framework, we restrict the properties that can be questioned to solely focus on the properties related to program correctness.

\noindent\textbf{\textit{Implementation.}}
PBT randomly generates inputs using a hand-crafted \emph{generator} and confirms that a property holds for these inputs. 
We develop a validation toolchain, based on the methodology in \Cref{sec:motivation}, using the symbolic state representation concept and carefully selecting the properties to validate for a program. 

\begin{figure}[h]
\vspace*{-1em}
  \includegraphics[width=0.8\textwidth]{pqasm}
   \caption{The Flow of PBT for QSV}
   \vspace*{-0.7em}
\label{fig:testing}
\end{figure}

The flow of the QSV PBT framework is given in \Cref{fig:testing}.
To validate a \pqasm program, we first utilize our \pqasm type system to generate a type environment $\Omega$ for qubits in a program $e$,
i.e., $\emptyset;\emptyset\vdash e \triangleright \Omega$, typing with an empty kind and type environment.
Essentially, $\Omega$ partitions all qubits used in $e$ into three sets $(\overline{q_1}, \overline{q_2}, \overline{q_3})$, with $\overline{q_1}$ containing all qubits in $\thadt$ type.
In \pqasm, once a qubit is turned into $\thadt$, it stays in the type.
We utilize the property to locate all the $\thadt$ typed qubits in a program $e$ and generate random testing data based on these qubits,
i.e., we identify the set $\overline{q_1}$ in $\Omega$ as the set of $\thadt$ typed qubits and generate random boolean values in $\{0,1\}$ for variables in the set.

Second, we assume that a program is in the form of $e=\iseq{\inew{\overline{q}}}{\iseq{\ihad{\overline{q}}}{e'}}$ \footnote{$\inew{\overline{q}}$ and $\ihad{\overline{q}}$ are syntactic sugar for multiple $\inew{q}$ and $\ihad{q}$ operations.},
i.e., all the $\cn{new}$ and $\cn{H}$ operations appear in the front of the program and $e'$ does not contain any such operations.
We then split the program by taking the $e'$ part and removing the $\cn{new}$ and $\cn{H}$ operations, and assuming that these operations have been applied.
For some program patterns, such as repeat-until-success programs, we replace recursive process variables in $e'$ with $\sskip$ operations so that we only validate one step of a repeat-until-success, because every recursive step in these programs is interdependent. 
One example of programming splitting for \Cref{def:circuit-example} is given below; it removes the part $\iseq{\inew{\overline{q}}}{\iseq{\inew{q'}}{\ihad{\overline{q}}}}$ and replacing $P$ with $\sskip$.

{\small
\begin{center}
$
P'={\iseq{\qbool{\overline{q}}{<}{n}{q'}}{\smea{x}{q'}{\sifb{x=1}{\sskip}{\sskip}}}}
$
\end{center}
}

The ``Test Gen'' step in our PBT framework (\Cref{fig:testing}) generates test cases to validate the key component $P'$ above.
In \Cref{def:circuit-example}, after applying the \cn{new} and \cn{H} operations, the post-state is $(\Phi,P')$, with $\Phi$ mapping $\theta$, entanglement groups, to superposition states $\varphi$. Each \cn{H} operation generates a single qubit uniformly distributed superposition state.
However, transitions over a superposition state make it hard to perform effective validation.
To resolve this, we treat quantum program operations $P'$ as higher-order map operations and validate the transition correctness based on basis-kets, instead of validating over the whole superposition state.

Generally, $\varphi$ can be written in the Dirac notation of $\sum_j \rho_j$ with $\rho_j = z_j \cdot \eta_j$,
i.e., given $\theta =(b_0,b_1,...,b_m,\overline{q_2},\overline{q_3})$, the superposition state $\varphi$ could also be written as the follows.

{\small
\begin{center}
$
\sum_{b_0=0}^1 \sum_{b_1=0}^1 ... \sum_{b_m=0}^1 z(b_0,b_1,...,b_m)\cdot \eta({b_0,b_1,...,b_m})
$
\end{center}
}

%the post-state after applying \cn{new} and \cn{H} ops is $(\theta \to z\cdot \eta,P')$, with $z\cdot \eta$ being a representative basis-ket state based on $\varphi$ above. 
%Clearly, $\varphi$ can be written in the Dirac notation of $\sum_j \rho_j$ with $\rho_j = z_j \cdot \eta_j$,
%so picking a representative basis-ket state could be an arbitrary $j$-th basis-ket, i.e., $z_j \cdot \eta_j$.
%Given $\theta =(b_0,b_1,...,b_m,\overline{q_2},\overline{q_3})$, the superposition state $\varphi$ could also be written as the follows.

Here, $b_0$, $b_1$, ..., $b_m$ are qubit variables assumed to be already manipulated by the initial \cn{H} operations, e.g., $P'$ above assumes that $\overline{q}$ were manipulated by \cn{H} operations.
Applying a \cn{H} operation to a $\tnort$ typed qubit $\aket{b_a}{1}$ creates a uniformly distributed superposition $\sum_{b=0}^1 \frac{1}{\sqrt{2}} (-1)^{b\cdot b_a} \aket{b}{1}$, so it results in the state form above, with $z(b_0,b_1,...,b_m)$ being an amplitude formula and $\eta({b_0,b_1,...,b_m})$ being a basis-vector formula.
We can then select the symbolic basis-ket state $z(b_0,b_1,...,b_m)\cdot \eta({b_0,b_1,...,b_m})$ as the representative basis-ket and rewrite the $\varphi$ state to be in the form $(\theta \to z\cdot \eta,P')$ for validation, with $b_0$, $b_1$, ..., $b_m$ acting as random variables.
For each random variable $b_j$, we can randomly choose the value $b_j\in\{0,1\}$ for a particular test instance.
Thus, for $m$ qubits, we have $2^m$ different test instances depending on the different value selection for the random variables.

For the $P'$ program above, we view each element in the $ m$-qubit array $\overline{q}$ as a random variable.
Then, we generate an initial state $(\theta \to z(\overline{q}[0],...,\overline{q}[m\sminus 1])\cdot \eta(\overline{q}[0],...,\overline{q}[m\sminus 1]),P')$
with $\overline{q}[0],...,\overline{q}[m\sminus 1]$ being random variables and $\theta=(\overline{q},q,\emptyset)$.
We can then generate test instances for choosing different variables for $\overline{q}[j]$ with $j\in[0,m)$.

Once we randomly generate test instances, we can then validate the program by running each test instance in our \pqasm interpreter, developed based on our program semantics.
The result of the interpreter is provided as input to a specification checker to validate if the specification is satisfied.
If the checker makes all test instances answer \cn{true}, we validate the program; otherwise, we report a fault in the program.
Below, we show the construction of the specification checker to validate the program's correctness and other properties.

\noindent\textbf{\textit{Validating Correctness.}}
We first run a test instance in our interpreter with the initial state, as $(\theta \mapsto z\cdot \eta, P)\longrightarrow^*(\theta' \mapsto z'\cdot \eta',\sskip)$, where $\theta \equiv \theta'$. For a user-specified property $\psi$, a satisfiability check is applied to $\psi(z \cdot \eta, z' \cdot \eta')$ by replacing variables with $z \cdot \eta$ and  $z' \cdot \eta'$.
%In testing the correctness, a general pattern for the property $\psi$ is listed as follows.
Recall that we demonstrate a transformation of correctness property in \Cref{sec:intro} for the program in \Cref{def:circuit-example},
to conduct validation on individual basis-kets rather than the whole quantum state.
Such property transformations can be summarized as the transformation from (1) to (3) as below:

{\footnotesize
\begin{center}
$
\text{(1) }
\sum_j z_j \cdot \eta_j \to f(\sum_j z_j \cdot \eta_j)
\qquad
\text{(2) }
\sum_j z_j \cdot \eta_j \to \sum_k g(z_k \cdot \eta_k) \wedge \phi(k)
\qquad
\text{(3) }
\forall j. z_j \cdot \eta_j \to g'(z_j \cdot \eta_j) \wedge \phi(j)
$
\end{center}
}

The correctness property should be written in the format of (3).
The property (1) describes the program semantics, i.e., given a superposition state $\sum_j r_j \cdot \eta_j$, $f$ represents the semantic function for a program $e$.
The effects of such a function can always be in the form of a linear sum by moving the sum operator to the front, as in (2): one can always find $g$ such that $f(\sum_j r_j \cdot \eta_j) = \sum_k g(r_k \cdot \eta_k)$.
In some cases, we might need to insert the $\phi(k)$ predicate above, constraining index $k$ in the sum operator.
Here, both $j$ and $k$ are indices for two different sum operators.
Often, the function $g$ can be turned into an equivalent form $g'$ based on the index $j$.
Once we equate the two indices, we can then transform the formula to (3) without any sum operators, via the axiom of extensionality.
Such transformation might come with the index restriction $\phi$ based on index $j$, as shown in (3),
which refers to the fact that we start with a superposition state with a representative basis-ket state $r_j \cdot \eta_j$ and output a basis-ket state $g'(r_j \cdot \eta_j)$, with the post-state index restriction $\phi(j)$.

To validate $P'$ above, we transform the correctness property from the left to the right one below ($\Rightarrow$ is logic implication).

{\small
\begin{center}
$
x = 1 \Rightarrow \sum_j^{2^m}\frac{1}{\sqrt{2^m}}\aket{j}{m}\aket{0}{1} \to \sum_j^{n}\frac{1}{\sqrt{n}}\aket{j}{m}
\qquad
\forall j\in[0,2^m)\,.\, x = 1 \Rightarrow \aket{j}{m}\aket{0}{1} \to \aket{j}{m} \wedge j < n
$
\end{center}
}

The right property states that, if the measurement results in $1$ ($x = 1$), each basis-ket in the pre-state for $\overline{q}$ and $q$ respectively have the basis vector forms $\aket{j}{m}$ and $\aket{0}{1}$, and results in $\overline{q}$ being the same $\aket{j}{m}$ with the restriction $j < n$.
The ${\otimes m}$ and ${\otimes 1}$ flags refer to the number of qubits in quantum array variables $\overline{q}$ and $q$.
%Essentially, each basis vector state is a bitstring, and a flag ${\otimes m}$ identifies a bitstring with length $m$.
In implementing a validation property, the flag essentially indicates the length of a bitstring piece, which is cast into a natural number for comparison.
%
To validate the correctness property against $P'$, we create a length $m$ bitstring for $\aket{j}{m}$ and view $j[k]$, for $k\in[0,m)$, being the $k$-th bit in the bitstring.
Recall that $P'$ has an initial basis vector state pattern as $\aket{\overline{q}[0]}{1}...\aket{\overline{q}[m\sminus 1]}{1}\aket{0}{1}$.
Here, $\aket{0}{1}$ is the state for qubit $q$ and $\overline{q}[0],...,\overline{q}[m\sminus 1]$ are random variables for qubit array $\overline{q}$.
To check the property, we bind each $j[k]$ with $\overline{q}[k]$ for $k\in[0,m)$, and see if the output basis-ket state results in the same $\overline{q}[0],...,\overline{q}[m\sminus 1]$. We also check if $\overline{q}$'s natural number representation is less than $n$,
i.e., we turn $\overline{q}[0],...,\overline{q}[m\sminus 1]$ to a number and compare it with $n$.

%\noindent\textbf{\textit{Testing Correctness.}}
%
% We have used PBT to test the correctness of a
%variety of operators useful in oracle programs, as presented in
%\Cref{sec:arith-oqasm}. When implementing a QFT-adder circuit, using
%PBT revealed that we had encoded the wrong endianness. 
%We have also used PBT with \vqimp programs by first
%compiling them to \pqasm and then testing their correctness at that
%level.

\noindent\textbf{\textit{Validating Other Properties.}}
The above procedure is only useful in validating correctness.
There might be other interesting properties, such as probability and effectiveness properties.
For example, in validating \Cref{def:circuit-example}, we might want to ask how likely the qualified state can be prepared,
which is hard to validate in general, but it can be effectively sampled out in some cases.


{\small
\begin{center}
$
x = 1 \Rightarrow \sum_j^{2^m}\frac{1}{\sqrt{2^m}}\aket{j}{m}\aket{0}{1} \to \sum_j^{n}\frac{1}{\sqrt{n}}\aket{j}{m}
$
\end{center}
}

In the superposition state-based property for $P'$ above, the number of qubits in $\overline{q}$ is $m$ and $n\in[0,2^m)$.
Here, let's see how to validate the effectiveness of the program, i.e., the probability that the repeat-until-success program produces the correct state.
Note that superposition states are always uniformly distributed without any \cn{Ry} operations.
The success rate of preparing a superposition state in the repeat-until-success scheme is the ratio between the number of possible basis-vector values less than $n$ and the total number of possible basis-vector values, i.e., $\overline{q}[0],...,\overline{q}[m\sminus 1]$. We can validate the effectiveness by calculating the number of possible values $\overline{q}$ less than $n$, by interpreting $\overline{q}$ as a natural number, dividing the number of possible values in $\overline{q}$; that is, $2^m$.

In general, assume that we have a basis-vector expression $e(\overline{q})$ for $m$ qubits $\overline{q}$, a measurement statement $\mathpzc{M}\cn{(}e(\overline{q})\cn{)}$ storing the result in $v$, and have a boolean check on $v$ as $B(v)$ defining the good states.
By assigning $\{0,1\}^m$ for $\overline{q}$, the probability of having the good states is the division of number of possible basis-vector values with $B(e(\overline{q}))=\cn{true}$ and the number of possible basis-vector values $e(\overline{q})$ for all possible assignments.
In \Cref{fig:intros-example}, the effectiveness can be described as $\frac{n}{2^m}$; such a property can be validated by sampling.
In some complicated cases, the right property might be hard to validate, but one can always use Rocq to verify the effectiveness via the above scheme.

\noindent\textbf{\textit{Performance Optimizations.}}
%
We took several steps to improve validation performance, e.g., we streamlined the representation of states: per the semantics in \Cref{fig:deno-sem}, in a state with $n$ qubits, the amplitude associated with each qubit can be written as $\Delta(\frac{\upsilon}{2^n})$ for some natural number $\upsilon$. 
Qubit values in both bases are thus pairs of natural numbers: the global phase $\upsilon$ (in range $[0,2^n)$) and $b$ (for $\aket{b}{1}$) or $y$ (for $\qket{\frac{y}{2^n}}$). 
An \pqasm state $\varphi$ is a map from qubit positions $p$ to qubit values $q$; in our proofs, this map is implemented as a partial function, but for validation, we use an AVL tree implementation (proved equivalent to the functional map). 
To avoid excessive stack use, we implemented the \pqasm semantics function tail-recursively. 
To run the tests, QuickChick runs OCaml code that it \emph{extracts} from the Rocq definitions; during extraction, we replace natural numbers and operations thereon with machine integers and operations. Performance results are in \Cref{sec:evaluation}.



% There are two modes of the comparison framework. First, we randomly generate same inputs for the two circuits, and answer the question how many percentages of outputs are exactly the same. If the difference rate is small, in some cases, the approximate circuit could be useful. Second, we also generate same inputs for the two circuits and answer the question that what parts in the output bitstrings are exactly the same. In analyzing circuits and their approximate implementations, most likely, a circuit might produce results that are almost always different from the results from its approximate implementation. However, some parts of their outputs might be the same. For example, in comparing the QFT-based addition circuit and the AQFT-based one, we find that the result high bits that are within the AQFT precision number range are the same as long as the low bits do not produce extra carry bit.
% Then, we can utilize the similarity being discovered in our framework to construct other useful oracles. See \Cref{sec:qft-case}.

% For PBT to be efficient, careful consideration must be given to the datatypes used to represent program states.
% As shown in \Cref{def:well-formed}, all possible phase values have the form $e^{2 \pi i b}$ for some real value $b$.In our implementation, we consider a finite approximation of $b$, such that there is a bitstring $[\upsilon]_n$ with $b\approx\frac{1}{2^{1\cdot [\upsilon]_n(k)}}+...+\frac{1}{2^{n-k\cdot [\upsilon]_n(n-1)}}$.
% This is why we have a smallest phase precision number $r_0$ (represented by the $n$ here) in translation from \pqasm to \sqir.
% Obviously, $[\upsilon]_n$ can be represented by a natural number $\upsilon$.
% The remaining issue is that we need to find a good number representation of $b$, so that phase rotation gate applications ($\texttt{RZ}^{[-1]}$ and $\texttt{SR}^{[-1]}$) can be implemented by natural number operations.
% We choose to record $b$ as a number $\upsilon'$ whose bitstring representation is $\texttt{rev}([\upsilon]_n)$ \footnote{$\texttt{rev}([c_1,...,c_n])=[c_n,...,c_1]$}, i.e, $[\upsilon']_n=\texttt{rev}([\upsilon]_n)$, where $[\upsilon]_n$ is the bitstring mentioned above.
% Then, applying the operation $(\upsilon'+2^{n - j})\%2^{n}$ is exactly the same as applying a phase rotation $e^{\frac{2 \pi i}{2^j}}$ to $b$ as $e^{2\pi i * (b+\frac{1}{2^j})}$, and applying $(\upsilon'+2^{n}-2^{n - j})\%2^{n}$ is the same as applying a phase rotation $e^{-\frac{2 \pi i}{2^j}}$ to $b$ as $e^{2\pi i * (b-\frac{1}{2^j})}$.
% Since $\texttt{SR}^{[-1]}$ is a series of $\texttt{RZ}^{[-1]}$, the $\upsilon'$ representation allows us to represent phases (and phase rotations) as natural numbers (and arithmetic operations)

\subsection{Translation from \pqasm to \sqir}\label{sec:vqir-compilation}

\newcommand{\tget}{\texttt{get}}
\newcommand{\tstart}{\texttt{start}}
\newcommand{\tfst}{\texttt{fst}}
\newcommand{\tsnd}{\texttt{snd}}
\newcommand{\tucom}[1]{\texttt{ucom}~{#1}}
\newcommand{\tif}{\texttt{if}}
\newcommand{\tthen}{\texttt{then}}
\newcommand{\telse}{\texttt{else}}
\newcommand{\tlet}{\texttt{let}}
\newcommand{\tin}{\texttt{in}}

We translate \pqasm to \sqir by mapping \pqasm virtual qubits to \sqir concrete qubit indices and expanding \pqasm instructions to sequences of \sqir gates.
\sqir \cite{VOQC} is a quantum circuit language based on Rocq, containing standard quantum gates, such as Hadamard, controlled, and \cn{Ry} gates, sequential operations, and quantum measurement operations.
To express the classical components of quantum algorithms, \sqir typically utilizes Rocq program constructs.
To define our compiler, we utilize the \sqir one-step non-deterministic semantics, containing one-step operational semantics for simple Rocq constructs, such as conditionals and classical sequential operations.


% Most \pqasm instructions are easily mapped to operations in \sqir, with the exception of the position shifting instructions.  
% The difficulty there is the virtual-to-physical qubit compilation.
% In \pqasm, a position $p$ is a pair of a qubit variable and offset, not
% a physical qubit location in a quantum circuit. We keep track of a map
% from each \pqasm position to a concrete \sqir qubit index.
%
The \pqasm to \sqir translation is expressed as the two-level judgments
$\Xi\vdash \iota \gg \epsilon$ and $(n,\Xi, e) \gg (n',\Xi',\chi)$, where $\epsilon$ is the output \sqir circuit, and $\Xi$ and $\Xi'$ map an \pqasm qubit $q$ to a \sqir concrete qubit index (i.e., offset into a  global qubit register), $\chi$ is a hybrid program including \sqir quantum circuits and Rocq classical programs,
and $n$ and $n'$ are the qubit sizes in the whole system.

\begin{figure}[t]
\vspace*{-0.5em}
{\small
  \begin{mathpar}
    \inferrule[CRy]{}{\Xi \vdash \iry{r}{q} \gg (\gamma,\textcolor{blue}{\iry{r}{(\Xi(q))}})}
    
    \inferrule[CCU]{\Xi\vdash \instr \gg \textcolor{blue}{\epsilon}\\
      \textcolor{blue}{\epsilon' = \texttt{ctrl}(\gamma(q),\epsilon)}}{\Xi\vdash\ictrl{p}{\instr} \gg \textcolor{blue}{\epsilon'}}    
   
       \inferrule[CNext]{ \Xi \vdash \iota \gg \textcolor{blue}{\epsilon}}{(n,\Xi,\iota) \gg (n,\Xi,\textcolor{blue}{\epsilon})}             
    
     \inferrule[CHad]{}{(n,\Xi, \ihad{q}) \gg (n,\Xi, \ihad{\Xi(q)})}             
               
    \inferrule[CSeq]{ (n,\Xi, e_1) \gg (n',\Xi',\textcolor{blue}{\chi_1}) \\ (n',\Xi',e_2) \gg (n'',\Xi'',\textcolor{blue}{\chi_2})}{(n,\Xi,\sseq{e_1}{e_2}) \to (n'',\Xi'',\textcolor{blue}{\sseq{\chi_1}{\chi_2}})}             
  
    \inferrule[CNew]{\Xi'=\Xi[\forall q\in\overline{q}\,.\,q \mapsto \slen{\Xi}+\cn{ind}(\overline{q},q)]}{(n,\Xi, \inew{\overline{q}}) \gg (n+\slen{\overline{q}},\Xi', \sskip)}             
  
      \inferrule[CMea]{(n,\Xi \backslash \overline{q},e) \gg (n',\Xi',\textcolor{blue}{\chi})}{(n,\Xi,\smea{x}{\overline{q}}{e}) \to (n',\Xi',\textcolor{blue}{{\mathpzc{M}(\overline{q})};\chi})} 
  
  
  \end{mathpar}
}
\vspace*{-1em}
\caption{Select \pqasm to \sqir translation rules (\sqir circuits are marked blue). $\slen{\Xi}$: the length of $\Xi$; $\cn{ind}(\overline{q},q)$: the index of $q$ in array $\overline{q}$. $\textcolor{blue}{{\mathpzc{M}(\overline{q})}}$ repeats $\slen{\overline{q}}$ times on measuring qubit array $\overline{q}$.}
\label{fig:compile-vqir}
\vspace*{-1em}
\end{figure}

\Cref{fig:compile-vqir} depicts a selection of translation rules.
Rules \rulelab{CRy} and \rulelab{CCU} are the instruction level translation rules, which translate a \pqasm instruction to a \sqir unitary operation.
$\iry{r}{q}$ has a directly corresponding gate in \sqir.
In the \texttt{CU} translation, the rule assumes that $\instr$'s translation does not affect the $\Xi$ position map. This requirement is assured for well-typed programs per rule \rulelab{CU} in \Cref{fig:exp-well-typed}. 
 \texttt{ctrl} generates the controlled version of an arbitrary \sqir program using standard decompositions \cite[Chapter 4.3]{mike-and-ike}.

The other rules in \Cref{fig:compile-vqir} are the program level rules, which translate a \pqasm program to a hybrid Rocq program including \sqir circuits with possible measurement operations.
Rule \rulelab{CNext} connects the instruction and the program-level translations.
Rule \rulelab{CHad} translates a Hadamard operation to a \sqir Hadamard gate, while rule \rulelab{CSeq} translates a sequencing operator.

Rule \rulelab{CNew} translates a qubit creation operation in \pqasm.
In \sqir, there is no qubit creation, in the sense that every qubit is assumed to exist in the first place.
The translation essentially translates the operation to a SKIP operation in \sqir and increments the qubit heap size in the generated \sqir program.
Note that the qubit size in the translation is always incrementing, i.e., a quantum measurement does not remove qubits but just makes some qubits inaccessible.
Rule \rulelab{CMea} translates a \pqasm measurement operation to \sqir measurements by repeatedly measuring out qubits in $\overline{q}$.
The translation removes measured qubits from $\Xi$, but it does not modify the qubit size.

\newcommand{\transs}[3]{[\!|{#1}|\!]^{#2}_{#3}}

We have proved the \pqasm-to-\sqir translation correct.
The proof utilizes the \sqir nondeterministic semantics, where a qubit measurement produces two possible outcomes with different probabilities associated with the outcomes, 
i.e., this nondeterministic semantics is essentially \sqir's way of describing a Markov-chain procedure.
To formally state the correctness property, we relate \pqasm superposition states $\Phi$ to \sqir states, written as $\denote{\Phi}^{n'}$, which are vectors of $2^{n'}$ complex numbers.
We can utilize $\Xi$ to relate qubits in \pqasm with qubit positions in \sqir.
%
%For example, say that our program uses two variables, $x$ and $y$, and both have two qubits.
%The qubit states are $\ket{0}$ and $\ket{1}$ (meaning that $x$ has type \texttt{Nor}), and $\qket{r_1}$ and $\qket{r_2}$ (meaning that $y$ has type \texttt{Phi}).
%Furthermore, say that $\gamma = \{(x,0)\mapsto 0,(x,1)\mapsto 1, (y,0)\mapsto 2, (y,1)\mapsto 3\}$. 
%This \pqasm program state will be mapped to the $2^4$-element vector $\ket{0}\otimes \ket{1}\otimes (\ket{0}+e^{2\pi i r_1}\ket{1})\otimes (\ket{0}+e^{2\pi i r_2}\ket{1})$.

\begin{theorem}\label{thm:vqir-compile}\rm[Translation Correctness]
  Suppose $\Sigma; \Omega \vdash e \triangleright \Omega'$ and
  $(n,\Xi,e) \gg (n',\Xi',\chi)$.
Then for $\Omega \vdash \Phi$ and $(\Phi,e)\xrightarrow{r}(\Phi',e')$, we have $(\denote{\Phi}^{n'},\chi)\xrightarrow{r}(\denote{\Phi'}^{n'},\chi')$ and $(n',\Xi',e')\gg (n'',\Xi'',\chi')$.
\end{theorem}

\begin{proof}
The proof of translation correctness is by induction on the \pqasm program $e$. 
Most of the proof simply shows the correspondence of operations in $e$ to their translated-to gates $\epsilon$ in \sqir, except for \cn{new} and measurement operations, which update the $\Xi$ map.
\end{proof}
% Notice that a \pqasm shifting operation on variable $x$ changes the virtual to physical map from $\gamma$ to $\gamma'$ while generating only \texttt{ID} gates. The map shifting changes the ``world view'' of later operations on $x$, because the qubit physical positions are different between $\gamma$ and $\gamma'$.
% To prove the correctness, for physical positions in $x$, we compare their virtual positions before and after the shifting by using the inverse maps of $\gamma$ and $\gamma'$. Then, we show that difference implements the shifting operation semantics.

%Note that to link a complete, translated oracle $\instr$ into a larger \sqir program may require that $\gamma = \gamma'$, i.e., $\texttt{neutral}(\instr)$, so that logical inputs match logical outputs. This requirement is naturally met for programs written to be reversible, as is the case for all arithmetic circuits in this paper, e.g., \coqe{rz_adder} from \Cref{fig:circuit-example}. % If necessary, the programmer could add dynamic swap instructions manually (encodable in \pqasm).

\ignore{
\begin{lemma}\label{thm:subgroupoid-lemma}\rm
   For all $\epislon \in \{\epsilon^{(\Sigma,\Omega)}\}$, if $\epislon$ is valid operation in $\hsp{S}_n$, $n \le m$, and $\hsp{S}_m$ and it every qubit in $\hsp{S}_m$ satisfies $\Omega$'s restriction, then 
\end{lemma}


We view $(\mathcal{H}, \instr )$ as a groupoid over Hilbert space $\mathcal{H}$, we can then defined a subset of $\mathcal{H}$ as $\mathcal{H}^n_s$, where it has the following conditions:

\begin{itemize}
\item Each element in $\mathcal{H}^n_s$ has the form: $\aket{q_1}{1}\otimes ... \otimes \aket{q_n}{1}$, where $\aket{q_1}{1}$,...,$\aket{q_n}{1}$ are 1-dimensional qubit. 
\item For any element $\aket{q_1}{1}$,...,$\aket{q_n}{1}$ in $\mathcal{H}^n_s$, $\aket{q_i}{1}$ has three possible forms:  $\alpha\aket{c}{n}$, $\frac{1}{\sqrt{2}} \alpha( s_1 \aket{0}{1}+ s_2 \aket{1}{1})$, or $\frac{1}{\sqrt{2}}\alpha~(\aket{0}{1}+\beta\aket{1}{1})$.
\end{itemize}

We view $\Sigma;\Omega\vdash \iota \triangleright \Omega'$ as a predicate for each \pqasm operation $\iota$ on where a program $\iota$ is defined given a subspace $\mathcal{H}_{(x, p)}$, then $(\mathcal{H}^n_s, \instr )$ is a sub-groupoid of $(\mathcal{H}, \instr )$ for all $\instr$ that is type-checked in $\mathcal{H}^n_s$.

We then define a superoperator over $\instr$ as $\instr^*(\rho)= \llbracket \instr \rrbracket \rho \llbracket \instr \rrbracket^{\dag}$ where $\rho \in (\mathcal{H}^n_s)^*$. $(\mathcal{H}^n_s)^*$ is the collection of density matrices seen as linear transformations from $\mathcal{H}^n_s$ to $\mathcal{H}^n_s$.
The superoperator gives the density matrix interpretation of the \pqasm semantics. We define a $2^m$ dimensional database $D$ as $\mathcal{H}^n_s \otimes D$, and $D$ has the format $\aket{q_1}{1}$,...,$\aket{q_{2^n}}{1}$ where $q_i$ is a $k$ array bitstring, each of the bitstring position is either $0$ or $1$.
We define a new operation in $\instr$ as $\texttt{read}\;y\;x$, such that $y$ is a $k$-length qubit, and $x$ is an $m$-length qubit representing the position. The desity matrix semantics of the $\texttt{read}$ operation is given as:

{
\[\Sigma^{2^m-1}_{0}\ket{i}\bra{i}\otimes D_{i}\]
}

With finite bijection mapping $\tget(\rho)$ and $\varrho$, we develop the translation process as the function $(d * \Sigma * \rho * \instr * \varrho) \to (\tucom{d}* \rho * \varrho)$, where $d$ is the dimension number indicating the number of qubits in the system, $\Sigma$ maps variables to qubit numbers in \pqasm, $\rho$ is the position mapping database, $\varrho$ is the inverse function of $\tget(\rho)$, $\instr$ is an \pqasm program, and $\delta\in\tucom{d}$ is a \sqir circuit. 


create a mapping database $\rho$ that maps positions $p$ to a data structure $\coqe{nat} * (\coqe{nat} \to \coqe{nat}) * (\coqe{nat} \to \coqe{nat})$. We assume that all qubit locations in \sqir are managed as a number line counting from $0$. The first \coqe{nat} value is the starting position for an \pqasm variable $x$ on the number line. We assume that $\texttt{start}(\rho,x)$ is a function to get the start position of $x$ in the map $\rho$. The second function ($\mu$, $\coqe{nat} \to \coqe{nat}$) is a mapping from position offset to the offset number of the physical position in \sqir. 
For example, a position $(x,i)$ is mapped to $\tstart(\rho,x)+\mu(i)$ in the number line. The third function ($\nu$, $\coqe{nat} \to \coqe{nat}$) is the inverse function mapping from an offset in \sqir back to the offset in \pqasm. For every offset $i$ for $x$ in \pqasm, if $\mu$ and $\nu$ are the two maps in $\rho(x)$, then $\mu(\nu(i)) = i$, and vice versa. We assume that the actual virtual to physical position mapping is $ \tget(\rho)$, which gets the physical position in \sqir for $p$.
$\tget(\rho,p)$ gives us the \sqir position for $p$ and its definition is $\texttt{start}(\rho,\texttt{fst}(p))+\texttt{get\_}\mu(\rho,\texttt{fst}(p))(\texttt{snd}(p))$.
On the other hand, since different virtual positions map to different physical positions, the function $\tget(\rho)$ is bijective; there is an inverse function $\varrho$ for $\tget(\rho)$, such that $\tget(\rho,p)=i \Rightarrow \varrho(i) = p$. The functions $\rho$, $\tget(\rho)$, and its inverse function $\varrho$ are also useful in the translation process, and we assume that they satisfy \textbf{finite bijection}, where for a set of positions $\overline{p}$, there exists a mapping database $\rho$, a dimension number $d$ and an inverse function $\varrho$, such that for all $p$ in $\overline{p}$, $\tget(\rho,p)=i$, $i<d$, $\varrho(\tget(\rho,p))=p$, and $\tget(\rho,\varrho(i))=i$.}
\ignore{
\begin{definition}\label{def:vars-def}\rm
(\textbf{finite bijection})
Given a virtual to physical position mapping database $\rho$, and the mapping function $\tget(\rho)$, its inverse function $\varrho$, a map from \pqasm variables to its qubit size $\Sigma$, and $d$ is the dimension of the qubits in \sqir and it is a maximum number that is larger than all physical position number in the image of $\tget{\rho}$, we say that $\rho$ and $\varrho$ is finitely bijective iff:

\begin{itemize}
  \item For all $p$, if $\tfst(p))$ is in the domain of $\rho$ and $\tsnd(p)< \Sigma(\tfst(p))$, then $\tget(\rho,p)<d$.
  \item For all $i$, if $i < d$, then $\tfst(\varrho(i))$ is in the domain of $\rho$ and $\tsnd(\varrho(i))< \Sigma(\tfst(\varrho(i)))$
\item For all $p$, if $\tfst(p))$ is in the domain of $\rho$ and $\tsnd(p)< \Sigma(\tfst(p))$, then $\varrho(\tget(\rho,p)) = p$.
\item For all $i$, if $i < d$, then $\tget(\rho, \varrho(i))=i$.
\item For all $p_1$ $p_2$, if $p_1 \neq p_2$, then $\tstart(\rho,p_1) \neq \tstart(\rho,p_1)$.
\item For all $x$ $y$, if $x \neq y$, then for all $i$ $j$, such that $i < \Sigma(x)$ and $j < \Sigma(y)$, $\tget(\rho,(x,i)) \neq \tget(\rho,(y, j))$.
\item For all $p$, if $\tsnd(p) < \Sigma(\tfst(p))$, then $\tget\_\mu(\rho,\tfst(p))(\tsnd(p))<\Sigma(\tfst(p))$.
\item For all $p$, if $i < \Sigma(x)$, then $\tget\_\nu(\rho,x)(i)<\Sigma(x)$.
\end{itemize}

\end{definition}
}

\ignore{
\subsection{\vqimp: A High-Level Oracle Language}\label{sec:qimp}

\begin{figure}[t]
{\footnotesize
\centering
\[\hspace*{-1em}
\begin{array}{c}
\begin{array}{l}
\texttt{fixedp sin}(\textcolor{red}{Q~\texttt{fixedp }x_{/8}},\;\textcolor{red}{Q~\texttt{fixedp }x_r},\;C~\texttt{nat }n)\{
\\[0.2em]
\;\;\textcolor{red}{x_r\texttt{ = }x_{/8};} \;\;C~\texttt{fixedp }n_y;\;\;\textcolor{red}{Q~\texttt{fixedp }x_z;}\;\;
\textcolor{red}{Q~\texttt{fixedp }x_1;}
\\[0.2em]
\;\;
C~\texttt{nat }n_1;\;\;
C~\texttt{nat }n_2;\;\;
C~\texttt{nat }n_3;\;\;
C~\texttt{nat }n_4;\;\;
C~\texttt{nat }n_5;
\\[0.2em]
\;\;\texttt{for }(C~\texttt{nat }i\texttt{ = }0;\; i\texttt{ < }n;\;i\texttt{++})\{\\
\;\;\quad n_1\texttt{ = }i+1;\;\;n_2\texttt{ = }2*n_1;\;\;n_3\texttt{ = }\texttt{pow}(8,n_2);\;\;
          n_4\texttt{ = }n_2+1;
\\[0.2em]
\;\;\quad n_5\texttt{ = }n_4!;\;\;n_y\texttt{ = }n_3 / n_5;\;\;
\textcolor{red}{x_z\texttt{ = }\texttt{pow}(x_{/8},n_4);}\\[0.2em]
\;\;\quad \texttt{if }(\texttt{even}(n_1))\;\;{\{
\textcolor{red}{{x_1}\texttt{ = }{n_y}*{x_z};\;\;
x_r\texttt{ += }{x_1};}

\}}\\[0.2em]
\;\;\quad\texttt{else } {\{\textcolor{red}{{x_1}\texttt{ = }{n_y}*{x_z};\;\;{x_r}\texttt{ -= }{x_1};}\};}\\[0.2em]
\;\;\quad\textcolor{red}{\texttt{inv}(x_1);\;\;\texttt{inv}(x_z);}

\}\\
\;\;\texttt{return }\textcolor{red}{(8*x_r)};\\
\}
\end{array}\\[8em]
\sin{x}\approx 8*(\frac{x}{8}-\frac{8^2}{3!}(\frac{x}{8})^3+\frac{8^4}{5!}(\frac{x}{8})^5-\frac{8^6}{7!}(\frac{x}{8})^7+...+(-1)^{n-1}\frac{8^{2n-2}}{(2n-1)!}(\frac{x}{8})^{2n-1})
\end{array}
\]
}
\vspace*{-1em}
\caption{Implementing sine in \vqimp}
\label{fig:sine-impl}
\end{figure}

It is not uncommon for programmers to write oracles as metaprograms in
a quantum assembly's host language, e.g., as we did for \coqe{rz_adder} in
\Cref{fig:circuit-example}. But this process can be tedious and error-prone,
especially when trying to write optimized code.
%
To make writing efficient arithmetic-based quantum oracles easier,
we developed \vqimp, a higher-level imperative language that compiles
to \pqasm. Here we discuss \vqimp's basic features, describe how we 
optimize \vqimp programs during compilation using partial
evaluation, and provide correctness guarantees for \vqimp programs. 
Using \vqimp, we have defined operations for the ChaCha20 hash-function \cite{chacha}, exponentiation, sine, arcsine, and cosine, and tested program correctness by running inputs through \vqimp's semantics. 
%
More details about \vqimp are available in \cite{vqoex} Appendix B.

\myparagraph{Language Features}

An \vqimp program is a sequence of function definitions, with the last
acting as the ``main'' function. Each function definition is a series
of statements that concludes by returning a value $v$.  \vqimp statements contain
variable declarations, assignments (e.g., $x_r\texttt{ = }x_{/8}$ in \Cref{fig:sine-impl}),
arithmetic computations ($n_1\texttt{ = }i+1$), loops, conditionals,
and function calls.
%In declarations, all variables are initialized as $0$.
Variables $x$ have types $\tau$, which are either primitive types
$\omega^m$ or arrays thereof, of size $n$. A primitive type pairs a
base type $\omega$ with a \emph{quantum mode} $m$. There are three base
types: type $\tnat$ indicates non-negative (natural) numbers; type
$\tfixed$ indicates fixed-precision real numbers in the range $(-1,1)$;
and type $\tbool$ represents booleans. The programmer specifies the
number of qubits to use to represent $\tnat$ and $\tfixed$ numbers
when invoking the \vqimp compiler.  
%
The mode $m \in\{C, Q\}$ on a primitive type indicates when a
type's value is expected to be known: $C$ indicates that the value is based
on a classical parameter of the oracle, and should be known at compile
time; $Q$ indicates that the value is a quantum input to the oracle, 
computed at runtime. 

\Cref{fig:sine-impl} shows the \vqimp implementation of the sine function,
which is used in quantum applications such as Hamiltonian
simulation~\cite{feynman1982simulating,Childs_2009}. 
Because $\tfixed$ types must be in the range $(-1,1)$, the function
takes $\frac{1}{8}$ times the input angle in variable $x_{/8}$ (the input 
angle $x$ is in $[0,2\pi)$). The result, stored in variable $x_r$, 
is computed by a Taylor expansion of $n$ terms.
The standard formula for the Taylor expansion is
$\sin{x}\approx
x-\frac{x^3}{3!}+\frac{x^5}{5!}-\frac{x^7}{7!}+...+(-1)^{n-1}\frac{x^{2n-1}}{(2n-1)!}$;
the loop in the algorithm computes an equivalent formula given input
$\frac{1}{8}x$, as shown at the bottom of the figure. 
% Performing this manual transformation is tedious, but allows for
% the significant reduction in qubits required to represent a
% $\tfixed$. We can use property testing to help assure the
% transformation is correct; automating the transformation would be
% interesting future work.

% The return value $8*x_r$ in \Cref{fig:sine-impl} is automatically computed by \vqimp.
% Programmers can set a flag $n$ for a $\tfixed$ variable $x$
% and \vqimp will automatically convert the value of $x$ to $\frac{x}{n}$, like $x_{/8}$ in \Cref{fig:sine-impl}, and multiply $n$ back to the return result, like $8*x_r$.
% \khh{But even if \vqimp adds the multiplication by 8, the user still has to write their algorithm with $x/8$ in mind, right? (E.g., ``pow(8,n2)'' in the code.) This doesn't seem very automatic.}

% Another feature of \vqimp is \emph{reversibility}; we provide two kinds.
% First, every \vqimp function call is reversible. For example, in a call to the \texttt{sin} function, excluding the computed result $x_r$, the side-effects on input variables, e.g., $x_{/8}$, are uncomputed once the call returns. 
% Second, \vqimp's \texttt{inv} operation can be used to manually uncompute a single assignment of a variable.
% For example, $\texttt{inv}(x_1)$ in \Cref{fig:sine-impl} uncomputes $x_1$ in every loop iteration.
% To implement this, we maintain a stack during compilation that tracks the nearest assignment of a variable.
% See \Cref{sec:appendix} for more details.

\myparagraph{Reversible Computation}
\label{sec:revcomp}

Since programs that run on quantum computers must be
\emph{reversible}, \vqimp compiles functions to reverse their
effects upon returning. In \Cref{fig:sine-impl}, after the
\texttt{main} function returns, only the return value is copied and
stored to a target variable. For other values, like $x_{/8}$, the 
compiler will insert an \emph{inverse circuit} to revert all side effects.

When variables are reused within a function, they must be
\emph{uncomputed} using \vqimp's $\sinv{x}$ operation. For
example, in \Cref{fig:sine-impl}, the second \texttt{inv} operation
returns $x_z$ to its state prior to the execution of
$\textcolor{red}{x_z\texttt{=}\texttt{pow}(x_{/8},n_4)}$ so that $x_z$ 
can be reassigned in the next iteration.
We plan to incorporate automatic uncomputation techniques to insert $\sinv{x}$ calls automatically, but doing so requires care to avoid blowup in the generated circuit \cite{unqomp}. 

The \pqasm compiler imposes three restrictions on the use of $\sinv{x}$, 
which aim to ensure that each use uncomputes just one assignment to $x$.
%
First, since the semantics of an \texttt{inv} operation reverses the
most recent assignment, we require that every \texttt{inv} operation
have a definite predecessor. Example \texttt{(1)} in \Cref{fig:inv-examples}
shows an \texttt{inv} operation on a variable that does not have a
predecessor; \texttt{(2)} shows a variable $z$ whose
predecessor is not always executed. Both are invalid in \vqimp.
Second, the statements between an \texttt{inv} operation and its
predecessor cannot write to any variables used in the body of the
predecessor. Example \texttt{(3)} presents an invalid case where $x$
is used in the predecessor of $z$, and is assigned between the
\texttt{inv} and the predecessor.  The third restriction is that, while
sequenced \texttt{inv} operations are allowed, the number of \texttt{inv}
operations must match the number of predecessors. Example \texttt{(4)}
is invalid, while \texttt{(5)} is valid, because the first
\texttt{inv} in \texttt{(5)} matches the multiplication assignment
and the second \texttt{inv} matches the addition assignment.

\begin{figure}[t]
\footnotesize
\[
\begin{array}{c}
\texttt{(1)}
\begin{array}{l}

a\texttt{=}x\texttt{ * }y;
\\
\texttt{inv}(z);
\textcolor{red}{\xmark}
\end{array}
\quad
\texttt{(2)}
\begin{array}{l}
\texttt{if}(x<y)\\
\;\;\;a\texttt{=}x\texttt{ * }y;\\
\texttt{else}\\
\;\;\;z\texttt{=}x\texttt{ * }y;
\\
\texttt{inv}(z);
\textcolor{red}{\xmark}
\end{array}

\quad
\texttt{(3)}
\begin{array}{l}

z\texttt{=}x\texttt{ * }y;
\\
x\texttt{=}x\texttt{ + }1;
\textcolor{red}{\xmark}
\\
\texttt{inv}(z);
\end{array}
\quad
\texttt{(4)}
\begin{array}{l}

z\texttt{=}x\texttt{ * }y;
\\
\texttt{inv}(z);
\\
\texttt{inv}(z);
\textcolor{red}{\xmark}
\end{array}
\quad
\texttt{(5)}
\begin{array}{l}

z\texttt{+=}x;
\\
z\texttt{=}x\texttt{ * }y;
\\
\texttt{inv}(z);
\\
\texttt{inv}(z);
\end{array}
\textcolor{green}{\cmark}
\end{array}
\]
\caption{Example (in)valid uses of \texttt{inv}}\label{fig:inv-examples}
\end{figure}

To implement these well-formedness checks, \pqasm's \vqimp compiler maintains a 
stack of assignment statements. Once the compiler hits an \texttt{inv}
operation, it pops statements from the stack to find a match
for the variable being uncomputed. It also checks that none of the popped
statements contain an assignment of variables
used in the predecessor statement.
}

%\input{sourcelang}
%\input{compilation}
\section{Evaluation: Applicativity via Case Studies}\label{sec:evaluation}

Here, we present an experimental evaluation of QSV on many programs to judge how QSV can be used to effectively build and validate useful quantum state preparation programs that have different patterns.
We compare QSV's efficiency and scalability with other frameworks in \Cref{sec:eval}.

%Here, we present an experimental evaluation of QSV on many programs to answer the questions:
\ignore{
\begin{itemize}
\item Q1: Applicability: Can QSV be used to build and validate useful quantum state preparation programs?

\end{itemize}
}
\ignore{
\begin{itemize}
% \item We state what is the evaluation metrics. How are we going to evaluate, and what questions the evaluation is trying to do?
% \item We aim to compare \pqasm’s capabilities to capabilities of existing quantum circuit validators.

\item We divide the state preparation cases into two groups and then develop validation strategies for different sets.

\item Showing the evaluation tables, we might need to find another framework for running the examples. Compared with Qiskit? In addition, it is not too impressive to only show the qubits; we might need to show the total qubits, and then we might need to show the number of gates.

\item We also need to have a way to demo how we can find bugs, like we show a bug program and show that our system can find the bugs.
\end{itemize}

\liyi{do things above, and maybe others.}
}

\myparagraph{Implementation} We implement QSV in Roqc and utilize QuickChick to create a quantum program validation framework.
We also provide a \pqasm circuit compiler to translate \pqasm programs to OpenQASM quantum circuits, via SQIR.

\myparagraph{Experimental Setup} We perform our evaluation of QuickChick tests and Qiskit on an Ubuntu computer, which has 8-core 13-gen i9 Intel processors and 16 GiB DDR5 memory.

%\liyi{maybe can be removed}
%\myparagraph{Statistics.} Our Roqc development is over 1,000 lines of code, ... A detailed view of the code statistics is
%given in the following table.

%\subsection{Q1: Applicaitivity via Case Studies}\label{sec:arith-oqasm}

\begin{figure}[h]
{\footnotesize
\begin{center}
\begin{tabular}{| l | c | c | c | c |}
\hline
 State Preparation Program  & 8B Gate \#  & 8B Qubit \# & 60B Gate \#  & 60B Qubit \# \\
 \hline
$n$ basis-ket & 766 & 9 & 5,732 & 61\\
Modular Exponentiation & 277K & 27 & 3.3M & 183\\
Amplitude Amplification & 65 & 9 & 481 & 61 \\
Hamming Weight & 9,088 & 16 & 170K & 120 \\
Distinct Element & 16,036  & 49 & 120K &  361 \\
\hline                           
\end{tabular}
\end{center}
}
\caption{Program statistics for single registers with 8 and 60 Qubits (8B/60B); $n=5$ for Distinct element. ‘K’ means thousand, ‘M’ means million’.}
\label{fig:qiskit-data}
\end{figure}

We show several case studies here to demonstrate the power of QSV for constructing and validating state preparation programs.
We classify \emph{two different program patterns} below and examine their validation strategies.
We list the qubit and gate counts for the case study programs in \Cref{fig:qiskit-data}.
The data are collected by compiling our programs to SQIR based on the elementary gateset $\{\cn{X}, \cn{H}, \cn{CX}, \cn{Rz}\}$.
To describe the programs, we define the following repeat operator for repeating a process $n$ times.
Here, $P$ is a function that takes a natural number and outputs a quantum program.

{\small
\begin{center}
$
Re(P,n)\triangleq\sifb{n=0}{\sskip}{\iseq{Re(P,n-1)}{P(n-1)}}
$
\end{center}
}

\subsubsection{Quantum Loop Programs.}\label{sec:quantumloop}
We first examine the class of programs only involving quantum unitary gates without measurement.
Such programs typically contain quantum loops, a repetition of subroutines formed via unitary gates.
These programs usually act as a large part of some quantum algorithms.
For example, the modular exponentiation state preparation program is the major part of Shor's algorithm, while the amplitude amplification state preparation program is the major part of an upgraded amplitude estimation algorithm \cite{10.1007/s11128-019-2565-2}.

\begin{figure}[t]
\vspace*{0.5em}
{\hspace*{-1em}
\begin{minipage}[b]{0.45\textwidth}
  {\tiny
  $
  \Qcircuit @C=0.5em @R=0.5em {
    &                     & & \gate{H} & \ctrl{6} & \qw & \qw & \qw & \qw & \qw & \qw & \\
    & \push{\overline{q_1}:\ket{0}\;\,} & & \gate{H} & \qw & \ctrl{5} & \qw & \qw & \qw & \qw & \qw &  \\
    &                     & &  & & & &  & & & & \\
    &                     & & \dots & & & & \dots & & & & \\
    &                     & & & & & & & & & & \\
    &                     & & \gate{H} & \qw & \qw & \qw & \qw & \qw & \ctrl{1} & \qw & \push{\;\;\varphi_1}\\
    &                     & & \qw & \multigate{4}{\texttt{$(c^{2^0} * \overline{q_2}) \,\cn{\%}\, n$}} & \multigate{4}{\texttt{$(c^{2^1} *  \overline{q_2}) \,\cn{\%}\, n$}} & \qw & \qw & \qw & \multigate{4}{\texttt{$(c^{2^{m-1}} *  \overline{q_2}) \,\cn{\%}\, n$}} & \qw & \\
    & \push{\overline{q_2}:\ket{1} \;\,} & & \qw & \ghost{\texttt{$(c^{2^0} * \overline{q_2}) \,\cn{\%}\, n$}} & \ghost{\texttt{$(c^{2^1} * \overline{q_2}) \,\cn{\%}\, n$}} & \qw & \qw & \qw & \ghost{\texttt{$(c^{2^{m-1}} * \overline{q_2}) \,\cn{\%}\, n$}} & \qw &  \\
    & & & \dots & & & & \dots & & & & \\
    & & & & & & & & & & & \\
    & & & \qw & \ghost{\texttt{$(c^{2^0} * \overline{q_2}) \,\cn{\%}\, n$}} & \ghost{\texttt{$(c^{2^1} * \overline{q_2}) \,\cn{\%}\, n$}} & \qw & \qw & \qw & \ghost{\texttt{$(c^{2^{m-1}} * \overline{q_2}) \,\cn{\%}\, n$}} & \qw
    \gategroup{1}{3}{5}{3}{1em}{\{}
    \gategroup{7}{3}{11}{3}{1em}{\{}
    \gategroup{1}{11}{11}{11}{1em}{\}}
    }
$
    }
    \caption{Modular Exponentiation Circuits}
\label{fig:mod-mult}
\end{minipage}
\hfill
\begin{minipage}[b]{0.5\textwidth}
{\tiny
$\qquad
  \Qcircuit @C=0.5em @R=0.5em {
    &                     & & \gate{H} & \ctrl{6} & \qw & \qw & \qw & \qw & \qw & \qw & \qw & \qw & \qw &  \\
    &                     & & \gate{H} & \qw & \ctrl{5} & \qw & \qw & \qw & \qw & \qw & \qw & \qw & \qw & \\
    & \push{\overline{q}:\ket{0}\;} & &  & & & &  & & & & & & & \\
    &                     & & \dots & & & & \dots & & &  & & & & \push{\;\;\varphi_2}\\
    &                     & & & & & & & & & & & & & \\
    &                     & & \gate{H} & \qw & \qw & \qw & \qw & \qw & \qw & \qw & \qw & \ctrl{1} & \qw & \\
    & \push{q':\ket{0}\;} & & \gate{Ry(\frac{r}{2^{n}})}  & \gate{Ry(\frac{r}{2^{n-1}})}      & \gate{Ry(\frac{r}{2^{n-2}})}  & \qw & & \dots  & & & \qw  & \gate{Ry(\frac{r}{2^{0}})} & \qw 
    \gategroup{1}{3}{6}{3}{1em}{\{}
    \gategroup{1}{14}{7}{14}{1em}{\}}
    }
$
}
\caption{Amplitude amplification state preparation.}
\label{fig:aacircuit}
\end{minipage}
}
\end{figure}

\myparagraph{Modular Exponentiation State Preparation}\label{sec:modmult}
%
In Shor's algorithm \cite{shors}, we prepare the the superposition state of modular exponentiation, as $\varphi_1 = \frac{1}{\sqrt{2^m}} \sum_j^{2^m}\aket{j}{m}\aket{\modexp{c}{j}{n}}{m}$, based on two natural numbers $c$ and $n$ with $\cn{gcd}(c,n)=1$.

{\footnotesize
\begin{center}
$
\begin{array}{l@{\;}c@{\;}l}
Q(\overline{q_1},\overline{q_2})(k) &\triangleq& \ictrl{(\overline{q_1}[k])}{\modmult{c^{2^k}}{\overline{q_2}}{n}}
\\[0.2em]
P(m) &\triangleq& \iseq{\inew{\overline{q_1}}}{\iseq{\iseq{\inew{\overline{q_2}}}{\ihad{\overline{q_1}}}}{{Re(Q(\overline{q_1},\overline{q_2}),m)}}}
\end{array}
$
\end{center}
}

We show the program to prepare the modular exponentiation state below, with the circuit diagram in \Cref{fig:mod-mult}.
The program starts with two new length $m$ qubit arrays $\overline{q_1}$ and $\overline{q_2}$, and turns $\overline{q_1}$ to a uniform superposition by applying $m$ Hadamard gates. We then repeat $m$ times a controlled modular multiplication ($\ictrl{(\overline{q_1}[k])}{\modmult{c^{2^k}}{\overline{q_2}}{n}}$) application, controlling on the qubit $\overline{q_1}[k]$ and applying modular multiplications to the qubit array $\overline{q_2}$.

As we mentioned in \Cref{sec:rand-testing}, to conduct the validation of the program correctness, we first cut off the first three operations ($\texttt{new}$ and $\cn{H}$ operations) in $P(m)$, resulting in a program piece $Re(Q(\overline{q_1},\overline{q_2}),m)$, where $Q(\overline{q_1},\overline{q_2})$ is a function taking in a number $k$ and outputting a program $\ictrl{(\overline{q_1}[k])}{\modmult{c^{2^k}}{\overline{q_2}}{n}}$.
The validation correctness specification is also transformed as the one without having the superposition state description below.

{\small
\begin{center}
$
\forall j \in [0,2^m)\,.\,\aket{j}{m}\aket{0}{m}\to\aket{j}{m}\aket{\modexp{c}{j}{n}}{m}
$
\end{center}
}

Validating the $m$-step loop program $Re(Q(\overline{q_1},\overline{q_2}),m)$ essentially executes the $Q$ program $m$ times.
In executing the $k$-th loop step, we have the following loop invariant.

{\small
\begin{center}
$
\aket{j}{k}\aket{\modexp{c}{j}{n}}{m}\to\aket{j}{k\splus 1}\aket{\modexp{c}{j}{n}}{m}
$
\end{center}
}

In the pre-state, $\aket{j}{k}$ is a length $k$ bitstring while the post-state has $\aket{j}{k\splus 1}$ being length $k\splus 1$.
To understand the behavior, notice that we have the most significant bit on the right. 
A $k\splus 1$ length bitstring $\aket{j}{k\splus 1}$ can be expressed as a composition over a length $k$ bitstring as $\aket{j}{k}\aket{0}{1}$ or $\aket{j}{k}\aket{1}{1}$, with the most sigificant bit being $0$ or $1$. 
For the former case, applying the controlled modular multiplication results in $\aket{j}{k}\aket{0}{1}\aket{\modexp{c}{j}{n}}{m}$,
i.e., the $\overline{q_2}$ part of the bitstring remains the same.
For the latter case ($\aket{j}{k}\aket{1}{1}$), the controlled modular multipliation results in the state $\aket{j}{k}\aket{1}{1}\aket{\modmult{c^{2^{k}}}{\modexp{c}{j}{n}}{n}}{m}=\aket{j}{k}\aket{1}{1}\aket{\modexp{c}{j+2^k}{n}}{m}$.
Both cases can be rewritten to the post-state in the loop invariant above.

To validate the program piece $Re(Q(\overline{q_1},\overline{q_2}),m)$ against the above specification,
we are given a length $2m$ bitstring with $\overline{q_1}$ and $\overline{q_2}$ both being length $m$,
and view $\overline{q_1}$ as a length $m$ array of random variables, and prepare a initial state $\aket{\overline{q_1}[0],...,\overline{q_1}[m\sminus 1]}{m}\aket{0}{m}$, with random generation of length $m$ binary bitstrings, as test instances, for the variables $\overline{q_1}[0],...,\overline{q_1}[m\sminus 1]$.
We then use the mechanism in \Cref{sec:rand-testing} to validate the program piece, performing a sort of unit testing.
As we mentioned in \Cref{sec:intro}, after we test enough samples of different input basis-ket states with different values for random variables, we have a high assurance that the modular exponentiation state preparation program correctly prepares a superposition state.

Note that the program is a deterministic quantum circuit program, so the probability of preparing the superposition state is $100\%$.  

\myparagraph{Amplitude Amplification State Preparation Through Ry Gates}\label{sec:aary}
%
In the amplitude amplification algorithm, one needs to prepare a special superposition state \cite{10.1007/s11128-019-2565-2}, with the circuits shown in \Cref{fig:aacircuit}.
The prepared state is $\varphi_2=\frac{1}{\sqrt{2^n}}\sum_j^{2^n}\aket{j}{n}\qket{\frac{(2j+1)r}{2^n}}$.
Then, the amplitude amplification algorithm utilizes the last qubits ($\qket{\frac{(2j+1)r}{2^n}}$) to amplify the amplitudes of the basis-kets having a particular property with respect to some $j$. The $r$ value is the upper limit of the possible amplitude value, i.e., we want to carefully select $r$ to ensure $\frac{(2j+1)r}{2^n}\in[0,\frac{\pi}{2})$.


{\footnotesize
\begin{center}
$
\begin{array}{l@{\;}c@{\;}l}
Q(\overline{q},q')(j) &\triangleq& \ictrl{(\overline{q}[j])}{\iry{\frac{r}{2^{n-j}}}{q'}}
\\[0.2em]
P(n) &\triangleq& \iseq{\inew{\overline{q}}}{\iseq{\iseq{\inew{q'}}{\iseq{\ihad{\overline{q}}}{\iry{\frac{r}{2^n}}{q'}}}}{{Re(Q(\overline{q},q'),n)}}}
\end{array}
$
\end{center}
}

We implement the program $P$ in \pqasm with the input of a qubit array $\overline{q}$ and a single qubit $q'$.
We then apply Hadamard operations on $\overline{q}$ and a $Y$-axis rotation on $q'$.
Eventually, we a series of controlled $Y$-axis rotation operations --- controlling on the $\overline{q}[j]$ qubit, for $j\in[0,n)$ and applying $\cn{Ry}$ on $q'$;
each single controlled $Y$ axis roation is handled by the $Q$ process.

Since there is no measurement in the above program, the success rate of preparing the amplitude amplification state is theoretically $100\%$.
We mainly test the correctness here.

{\small
\begin{center}
$
{\iseq{{{\iry{\frac{r}{2^n}}{q'}}}}{{Re(Q(\overline{q},q'),n)}}}
\qquad
\qquad
\forall j \in [0,2^n)\,.\,\aket{j}{n}\aket{0}{1}\to\aket{j}{n}\qket{\frac{(2j+1)r}{2^n}}
$
\end{center}
}

In doing so, we can adapt the same strategy in the modular exponentiation state preparation, where we assume the correctness of the portion, $\iseq{\inew{\overline{q}}}{\iseq{\inew{q'}}{\ihad{\overline{q}}}}$, can be easily judged, so we should mainly focus on validating the portion shown above on the left.
The correctness specification is described above on the right.
In validating this portion, we notice that the qubit array $\overline{q}$ is in $\thadt$ type, and we generate random variables, with binary values $0$ or $1$, for every qubit in the array. Through the validation procedure in validating the modular exponentiation program above, we can assure that the result of the program produces the superposition state $\varphi_2$.

\subsubsection{Repeat-until-success Programs.}\label{sec:repeat-success}

We then examine the class of repeat-until-success programs, using the strategy exactly demonstrated in \Cref{sec:rand-testing}.
We show the Hamming weight and the distinct element state preparation program validation below.

\begin{figure}[t]
{\scriptsize
$\begin{array}{c}
  \Qcircuit @C=0.5em @R=0.5em {
    &                     & & \gate{H} & \ctrl{6} & \qw & \qw & \qw & \qw & \qw & \qw & \qw & \\
    &                     & & \gate{H} & \qw & \ctrl{5} & \qw & \qw & \qw & \qw & \qw &  \qw & \\
    & \push{\overline{q_1}:\ket{0}\quad} & &  & & & &  & & & & & \push{\quad\varphi_3}\\
    &                     & & \dots & & & & \dots & & & & & \\
    &                     & & & & & & & & & & & \\
    &                     & & \gate{H} & \qw & \qw & \qw & \qw & \qw & \ctrl{1} & \qw & \qw & \\
    &                     & & \qw & \multigate{4}{\texttt{$\iadd{\overline{q_2}}{1}$}} & \multigate{4}{\texttt{$\iadd{\overline{q_2}}{1}$}} & \qw & \qw & \qw & \multigate{4}{\texttt{$\iadd{\overline{q_2}}{1}$}} & \qw & \meter & \\
    & \push{\overline{q_2}:\ket{0}\quad } & & \qw & \ghost{\texttt{$\iadd{\overline{q_2}}{1}$}} & \ghost{\texttt{$\iadd{\overline{q_2}}{1}$}} & \qw & \qw & \qw & \ghost{\texttt{$\iadd{\overline{q_2}}{1}$}} & \qw &  \meter & \push{\quad x}\\
    &                     & & \dots &                                     &                                     &     & \dots &     &                                         & & \dots & \\
    &                     & &       &                                     &                                     &     &       &     &                                         & & & \\
    &                     & & \qw   & \ghost{\texttt{$\iadd{\overline{q_2}}{1}$}} & \ghost{\texttt{$\iadd{\overline{q_2}}{1}$}} & \qw & \qw   & \qw & \ghost{\texttt{$\iadd{\overline{q_2}}{1}$}} & \qw & \meter 
    \gategroup{1}{3}{6}{3}{1em}{\{}
    \gategroup{7}{3}{11}{3}{1em}{\{}
    \gategroup{1}{12}{6}{12}{1em}{\}}
    }
    \end{array}
$
}
\caption{One step Hamming weight state preparation (repeat-until-success).}
\label{fig:hammingcircuit}
\end{figure}


\myparagraph{Hamming Weights State Preparation}\label{sec:hammingweight}
%
The quantum clique finding algorithm \cite{10.5555/2011430.2011431} requires the state preparation of a $k$-th Hamming weight superposition state,
i.e., we prepare a state $\varphi_3=\frac{1}{\sqrt{N}}\sum_j^N \aket{c_j}{n}$, with the number of $1$'s bit in $c_j$ is $k$.
Assuming that $\varphi_3$ is a length $n$ qubit state, $\varphi_3$ has $N$ different basis-kets, with $N=\begin{pmatrix}
n\\
k
\end{pmatrix}$.

{\footnotesize
\begin{center}
$
\begin{array}{l@{\;}c@{\;}l}
Q(\overline{q_1},\overline{q_2})(j) &\triangleq& \ictrl{(\overline{q_1}[j])}{\cn{add}(\overline{q_2},1)}
\\[0.2em]
P(n,k) &\triangleq& \iseq{\inew{\overline{q_1}}}{\iseq{\iseq{\inew{\overline{q_2}}}{\ihad{\overline{q_1}}}}{\iseq{Re(Q(\overline{q_1},\overline{q_2}),n)}{\smea{x}{\overline{q_2}}{\sifb{x=k}{\sskip}{P(n,k)}}}}}
\end{array}
$
\end{center}
}

The above is a repeat-until-success program of the Hamming weight program above, with the circuit in \Cref{fig:hammingcircuit} showing a single quantum step in $P(n,k)$.
The program starts with two new length $n$ qubit arrays $\overline{q_1}$ and $\overline{q_2}$, and turns $\overline{q_1}$ to a uniform superposition by applying $n$ Hadamard gates. We then repeat $n$ times of a controlled addition ($\ictrl{(\overline{q_1}[j])}{\cn{add}(\overline{q_2},1)}$) applications --- controlling on the qubit $\overline{q_1}[j]$ and applying additions to the qubit array $\overline{q_2}$.
The controlled additions count the number of $1$'s bits in $\overline{q_1}$ and store the result in $\overline{q_2}$.
If the measurement on the qubit array $\overline{q_2}$ results in $k$ (assigning to $x$), it means that the $\varphi_2$ state of the qubit array $\overline{q_1}$ is a superposition of basis-ket states with the vector having $k$ bits of $1$.
Otherwise, we repeat the process $P$ with two new qubit arrays $\overline{q_1}$ and $\overline{q_2}$ until the measurement result $k$ appears.
%The repeat-until-success program guarantees that a $k$-th Hamming weight superposition state is prepared correctly.
Note that $Q(\overline{q_1},\overline{q_2})$ in $Re$ is a function taking in a natural number argument $j$ and then performing a controlled addition.

{\small
\begin{center}
$
{\iseq{Re(Q(\overline{q_1},\overline{q_2}),n)}{\smea{x}{\overline{q_2}}{\sifb{x=k}{\sskip}{\sskip}}}}
$
\end{center}
}

{\small
\begin{center}
$
\forall j \in [0,2^n)\,.\,\aket{j}{n}\aket{0}{n}\to\aket{j}{n}\wedge \cn{sum}(\cn{n2b}(j))=k
$
\end{center}
}

To validate the correctness of the Hamming weight program, we shrink the program by removing the \cn{new} and \cn{H} operations.
The program piece and the transformed correctness specification are listed above.
We then utilize the procedure in \Cref{sec:rand-testing} to perform the validation.
Here, we assume that the $\thadt$ typed qubits $\overline{q_1}$ are already prepared, and we randomly generate a length $n$ bitstring for the random variables $\overline{q_1}[0],...,\overline{q_1}[n\sminus 1]$.
Each random variable, possibly being $0$ or $1$, represents the basis-bit of a single qubit superposition.
We set up the PBT to randomly sample values for the random variables and exclusively test the correctness of the transition behavior of basis-ket states.
The key correctness property ($\cn{sum}(\cn{n2b}(j))=k$) for the Hamming weight state is that each output basis-ket of $\overline{q_1}$ should have exactly $k$ bits of $1$.

The judgment of the efficiency of the program in successfully preparing the superposition state can easily be done by counting the number of basis-kets in a superposition quantum state.
Notice that every superposition state prepared by a simple Hadamard operation produces a uniform superposition, meaning that the likelihood of measuring out any basis-ket vector is equally likely.
Thus, we only need to compare the ratio between the number of basis-kets after the Hadamard operations are applied and the basis-ket number in $\overline{q_1}$ after the measurement is applied. The former contains $2^n$ different basis-kets for $n$ Hadamard operations, and the latter has $\begin{pmatrix}
n\\
k
\end{pmatrix}$ basis-kets in a $k$-th Hamming weight state.
So, the success rate of a single try in the program is $\begin{pmatrix}
n\\
k
\end{pmatrix} / 2^n$.


\begin{figure}[t]
\vspace*{-0.5em}
{\hspace*{-3em}
\begin{minipage}[t]{0.45\textwidth}
\subcaption{Subroutine $Q(j,k)$}
\label{fig:subroutine}
\vspace*{0.3em}
{\tiny
$\begin{array}{l}
\;\; Q(j,k) \;\;= \\[1em]
  \Qcircuit @C=0.5em @R=0.5em {
    & \push{\overline{q_{j+2}}}  & &  \multigate{2}{\texttt{$\qbool{\overline{q_{j+2}}}{=}{\overline{q_{k+2}}}{q_0}$}} & \qw &  \multigate{2}{\texttt{$\qbool{\overline{q_{j+2}}}{=}{\overline{q_{k+2}}}{q_0}$}} & \qw & \\
    &  \push{\overline{q_{k+2}}} & & \ghost{\texttt{$\qbool{\overline{q_{j+2}}}{=}{\overline{q_{k+2}}}{q_0}$}}         & \qw & \ghost{\texttt{$\qbool{\overline{q_{j+2}}}{=}{\overline{q_{k+2}}}{q_0}$}} & \qw &  \\
    & \push{q_0}             & & \ghost{\texttt{$\qbool{\overline{q_{j+2}}}{=}{\overline{q_{k+2}}}{q_0}$}}         & \ctrl{1} & \ghost{\texttt{$\qbool{\overline{q_{j+2}}}{=}{\overline{q_{k+2}}}{q_0}$}} & \qw &  \\
    & \push{\overline{q_1}}  & & \qw                                                                       & \gate{\texttt{$\iadd{\overline{q_1}}{1}$}} & \qw & \qw &  \\
    }
    \end{array}
$
}
\end{minipage}
%
  \begin{minipage}[t]{0.52\textwidth}
   \subcaption{Overall One Step Circuit}
\label{fig:diselems}
\vspace*{0.3em}
{\tiny
$\begin{array}{c}
  \Qcircuit @C=0.5em @R=0.5em {
    &\push{\overline{q_2}:\ket{0}}                     & & \gate{H} &  \multigate{7}{\texttt{$Q(2,3)$}} & \qw & \qw & \qw & \multigate{7}{\texttt{$Q(2,n+1)$}} & \multigate{7}{\texttt{$Q(3,4)$}}  & \qw & \qw & \qw     & \multigate{7}{\texttt{$Q(n,n+1)$}} & \qw & \\
    &\push{\overline{q_3}:\ket{0}} & & \gate{H} & \ghost{\texttt{$Q(2,3)$}} & \qw & \qw & \qw & \ghost{\texttt{$Q(2,n+1)$}} & \ghost{\texttt{$Q(3,4)$}}   & \qw & \qw & \qw     & \ghost{\texttt{$Q(n,n+1)$}} &  \qw & \\
    &  & &  & & &   & & & & & & & & & \push{\varphi_4}\\
    &                     & & \dots & & & \dots & & & &  & \dots & & & &\\
    &                     & & & & & & & & & & & & &\\
    &\push{\overline{q_{n+1}}:\ket{0}}                     & & \gate{H} & \ghost{\texttt{$Q(2,3)$}} & \qw & \qw & \qw & \ghost{\texttt{$Q(2,n+1)$}} & \ghost{\texttt{$Q(3,4)$}}  & \qw & \qw & \qw     & \ghost{\texttt{$Q(n,n+1)$}} & \qw & \\
    &\push{q_0:\ket{0}}    & & \qw & \ghost{\texttt{$Q(2,3)$}} & \qw & \qw & \qw & \ghost{\texttt{$Q(2,n+1)$}} & \ghost{\texttt{$Q(3,4)$}}  & \qw & \qw & \qw     & \ghost{\texttt{$Q(n,n+1)$}} & \qw & \\
    &\push{\overline{q_1}:\ket{0}}  & & \qw & \ghost{\texttt{$Q(2,3)$}} & \qw & \qw & \qw & \ghost{\texttt{$Q(2,n+1)$}} & \ghost{\texttt{$Q(3,4)$}}  & \qw & \qw & \qw     & \ghost{\texttt{$Q(n,n+1)$}} & \meter & \push{x}   
    %\gategroup{1}{3}{6}{3}{1em}{\{}
    \gategroup{1}{15}{7}{15}{1em}{\}}
    }
    \end{array}
$
}
\end{minipage}
\vspace*{-0.5em}
\caption{One step of the distinct element state preparation program.}
\label{fig:distinctelem}
}
\vspace*{-1em}
\end{figure}

\myparagraph{Distinct Element State Preparation}\label{sec:distinctness}
%
Another special superposition state is the one in the element distinctness algorithm.
Here, we assume that we are given a graph with $n$ different vertices, and the algorithm begins with a superposition of different combinations of vertices, as shown below.

{\small
\begin{center}
$
\varphi_4=\frac{1}{\sqrt{n!}}\sum_{j} \sigma_j(\ket{x_1}\ket{x_2}...\ket{x_n})
$
\end{center}
}

Here, $x_1$, $x_2$, ..., $x_n$ are different vertex keys in the graph, $\sigma_j$ is a permutation of the key list $\ket{x_1}\ket{x_2}...\ket{x_n}$.
There are $n!$ different kinds of permutations, so the uniform amplitude for each basis-ket is $\frac{1}{\sqrt{n!}}$.
Essentially, the superposition of different vertex combinations means we are preparing a superposition state containing all the permutations of different vertex keys.
Such superposition of permutation is widely used in many algorithms, such as the quantum fingerprinting algorithm \cite{Buhrman_2001}.

{\footnotesize
\begin{center}
$\hspace*{-0.5em}
\begin{array}{l@{\;}c@{\;}l}
Q(k)(j) &\triangleq& \iseq{\qbool{\overline{q_{j+2}}}{=}{\overline{q_{k+2}}}{q_0}}{\iseq{\ictrl{q_0}{(\overline{q_1}+1)}}{\qbool{\overline{q_{j+2}}}{=}{\overline{q_{k+2}}}{q_0}}}
\\[0.2em]
R(j)(n) &\triangleq& Re(Q(j),n)
\\[0.2em]
H(j) &\triangleq& \ihad{\overline{q_{j+2}}}
\\[0.2em]
T(j) &\triangleq& \inew{\overline{q_{j+1}}}
\\[0.2em]
P(n) &\triangleq& \iseq{\inew{q_0}}{\iseq{Re(T,n+1)}{\iseq{{Re(H,n)}}{\iseq{Re(R(n-1),n)}{\smea{x}{\overline{q_1}}{\sifb{x=0}{\sskip}{P(n)}}}}}}
\end{array}
$
\end{center}
}

For simplicity, we only implement the above program to prepare a superposition state of distinct elements, i.e., each basis-ket in the superposition state stores $n$ distinct elements (vertex key), each key having a qubit size $m$. Note that if $n=2^m$, i.e., we have $2^m$ different vertices having keys $u\in[0,2^m)$, then the superposition state represents a superposition of all the permutations.
We show a repeat-until-success program of preparing the distinct element superposition state above, with the circuit in \Cref{fig:distinctelem} showing a single quantum step in $P(n)$. We first initialize a single qubit $q_0$, and use $T(j)$ to initialize $n+1$ different qubit arrays, $\overline{q_{j+1}}$, with $j\in [0,n+1)$, and we assume that $\overline{q_{j+1}}$ is an $m$ length qubit array.
We then apply Hadamard operations to all qubit arrays $\overline{q_{j+2}}$, for $j \in [0,n)$.
Here, $q_0$ and $\overline{q_1}$ are ancillary qubits.

Essentially, we can view $\overline{q_{j+2}}$, for $j \in [0,n)$, as an $n$-length array of qubit arrays.
The program applies $O(n^2)$ times of $Q$ processes, each of which applies an equivalent check on two elements in the $n$-length array,
i.e., we compare the basis-ket data in $q_{j+2}$ and $q_{k+2}$ ($j,k\in[0,n)$), if $j+2 \neq k+2$ (same as $j\neq k$), and store the boolean result in $q_0$ bit, where $0$ represents the two basis-ket data are not equal and $1$ means they are equal.
Then, we also add the result to $\overline{q_1}$ and apply the comparison circuit again to clean up the ancillary qubit $q_0$, meaning that we restore $q_0$'s state to $\ket{0}$. This procedure describes the circuit in \Cref{fig:subroutine}.

After we apply the $Q$ function to any two different elements in the $n$-length array, we observe that $q_0$ is back to $\ket{0}$ state, and $\overline{q_1}$ stores the number of same basis-kets between any two distinct elements in the $n$-length array.
We then measure $\overline{q_1}$ and see if the measurement result is $0$. If so, a permutation superposition state is prepared because it means that in all the basis-kets in the prepared superposition, there are no two-qubit array elements $\overline{q_k}$ and $\overline{q_l}$ that have equal key.
If not, we repeat the process, and the repeat-until-success program guarantees the creation of the permutation superposition state.

To validate the correctness, we utilize the procedure in \Cref{sec:rand-testing}. We create the validating program piece by shrinking out the \cn{new} and \cn{H} operations in the original program. The program piece and the transformed specification are listed below.

{\small
\begin{center}
$
\iseq{Re(R(n-1),n)}{\smea{x}{\overline{q_1}}{\sifb{x=0}{\sskip}{\sskip}}}
$
\end{center}
}

{\small
\begin{center}
$
x=0\Rightarrow\forall j,j'\in [0,2^{n * m})\,.\,\aket{0}{1}\aket{0}{n}\aket{j}{n*m}\to \aket{0}{1}\aket{j'}{n*m} \wedge \cn{dis}(j',m)
$
\end{center}
}

In the specification, $j$ and $j'$ represent the values for two arrays of qubit arrays, i.e., $j$ represents the bitstring value for composing basis-ket values of all elements in the qubit array $\overline{q_{l+2}}$, with $l\in[0,n)$. The qubit array $\overline{q_{j+2}}$ has $n*m$ qubits, and we slice the basis-vector for the whole qubit array into $n$ different small segments for the qubit ranges $[l*m,l*(m+1)-1)$, each basis-vector segment representing a vertex key.
Since we apply Hadamard operations to all of them, it creates a uniform superposition state containing $2^{n * m}$ different basis-vector states.
In the post-state of the specification, the $q_0$ qubit is still $\ket{0}$.
For the qubit arrays $\overline{q_{2}},...,\overline{q_{m+2}}$, if the measurement result is $x=0$, we result in a superposition state of distinct elements, i.e., any two elements (each element $l$ is a segment of $[l*m,l*(m+1)-1)$) in the qubit array $\overline{q_{j+2}}$ have distinct basis-vectors.
We use the predicate $\cn{dis}(j',m)$ to indicate that all length $m$ segments in the bitstring $\ket{j}$ are pairwise distinct.
Via our PBT framework, we have a high assurance that the distinct element state preparation program correctly prepares such a superposition state.

The judgment of the efficiency of the program in successfully preparing the superposition state can easily be done by counting the number of basis-ket states in a superposition quantum state.
Notice that every superposition state prepared by a simple Hadamard operation produces a uniform superposition, meaning that the likelihood of measuring out any basis state vectors is equally likely.
Thus, we only need to compare the ratio between the number of basis-kets right after the Hadamard operations are applied and the basis-ket number in $\overline{q_{j+2}}$.
Here, we count the case for $n=2^m$ where a permutation superposition state is prepared.
For $j\in[0,n)$ with $n=2^m$, after the measurement is applied. The former contains $2^{n * m}$ different basis-kets for $n * m$ Hadamard operations, and the latter has $n!$ basis-kets in a $k$-th permutated superposition state. So, the success rate of a single try in the program is $\frac{n !}{2^{n * m}}$.

\section{Discussion: Efficiency, Scalability, and Utility, Compared to State-of-the-art}\label{sec:eval}

This section compares QSV's efficiency and scalability with the state-of-the-art platforms. We demonstrate that QSV can effectively validate program properties to provide confidence in program correctness. We also show that QSV \textit{scales} to validate and realize state preparation programs regardless of the program size (in terms of number of qubits required).
We thus justify the QSV's utility in successfully capturing bugs.

\begin{figure}[t]
{\footnotesize
\begin{center}
\begin{tabular}{| l | c | c | c | c |  c |}
\hline
 Program  & QSV  QCT 8B & QSV QCT 60B  & Qiskit Sim 8B & Qiskit Sim 60B & DDSim Sim 60B \\
 \hline
$n$ basis-ket & < 1.5  & 2.4 & < 1.5 & No & No \\
Modular Exponentiation & < 1.5  & 19.7 & No & No & No\\
Amplitude Amplification & < 1.5  & 2.1 & <1.5 & No & No  \\
Hamming Weight & < 1.5  & 24.9 & <1.5 & No & No \\
Distinct Element & < 5  & 336  & No & No & No \\
\hline                           
\end{tabular}
\end{center}
}
\caption{Evaluation on different state preparation programs for 8/60 qubit single registers (8B/60B). "QCT" is the time (in seconds) for QuickChick to run 10,000 tests. "Sim" records the time (in seconds) or whether or not Qiskit/DDSim can execute a single test. }
\label{fig:self-data}
\end{figure}

\myparagraph{Efficiency}\label{sec:testefficient}
We discuss the program development procedure in QSV, compared to state-of-the-art systems such as Qiskit.
Discussing the efficiency of a validation framework needs to be put in the context of human efforts for program development,
as users mainly care about how to effectively use QSV to develop state preparation programs.

The general procedure for developing state preparation programs in QSV is the traditional test-driven program development.
We are first given the program correctness properties, in the superposition state format, for different programs, such as the one in \Cref{sec:intro,sec:evaluation}.
We then start implementing the program via a possible program pattern and see if we can write the correct program based on the pattern, where the \pqasm high-level abstraction helps write programs.
For example, in dealing with all the programs in \Cref{fig:qiskit-data}, we first try to see if we can write all these programs via the quantum loop program pattern,
and rewrite the correctness properties based on the strategy presented in \Cref{sec:quantumloop}.

We then run our QSV validator to validate the implemented programs against the properties.
After several rounds of corrections, one can typically judge if the program is implementable or not.
For example, via our validator, we found that it might be hard to implement the Hamming weight and distinct element programs based on the quantum loop program pattern.
We then switch to other program patterns to implement these programs, e.g., we utilize the repeat-until-success program pattern to implement the two programs by rewriting the correctness properties for the two programs to the ones in \Cref{sec:repeat-success} and successfully find a solution.
Our validator can validate a program effectively via our PBT framework, which generates $10,000$ test cases every time.
As one can see in \Cref{fig:self-data}, executing a validation step involving running $10,000$ randomly generated test cases for all our example programs costs us less than $5$ seconds for small size programs ($8$ qubits) and less than 5.5 minutes for large size programs ($60$ qubits).
This indicates that a program developer can quickly correct minor bugs when developing their programs.

On the other hand, developing state preparation programs in the start-of-the-art system might be painful,
e.g., it is unlikely one can perform the test-driven development for implementing the programs in \Cref{fig:qiskit-data},
mainly because of a lack of proper validation facilities.
As we can see in \Cref{fig:self-data}, Qiskit might not execute a single test for some small-size ($8$ qubits) programs, such as distinct element programs.
The modular exponentiation program is executed for some small number settings, but is not executable in general because the Qiskit simulator adopts some special optimizations for components in Shor's algorithm.
Executing a large-size program ($60$ qubits) is completely impossible; either the program is too large for IBM's Aer simulator to run at all, and the simulator errors out, or the circuit construction may take hours and still not be done.
This indicates the difficulty of developing large-scale programs by conducting small-scale testing in Qiskit, not to mention the need to validate large datasets and coverage.
Another key issue is that the high-level abstraction support in the state-of-the-art systems is not well provided.
In Qiskit, we can only find the quantum addition operations, while the other arithmetic and comparison operations are missing.
In fact, we implement these operations in Qiskit based on our implementation in QSV.

\myparagraph{Scalability}
%We construct the previously mentioned programs in \pqasm and in Qiskit. 
%\label{sec:eval-oqasm}
To evaluate the QSV scalability, we not only compare the execution of small and large sizes against Qiskit but also another state-of-the-art quantum simulator, DDSim.
Here, we attempted to recreate (or find existing implementations of) the aforementioned programs on DDSim and Qiskit. We also performed PBT on our \pqasm implementations on systems of various sizes and verified the rigidity of our tests by mutating either the properties or the states and verifying that the tests failed. 

As shown in \Cref{fig:self-data}, we have fully validated the five examples in these papers via our PBT framework.
As far as we know, these constitute the first validated-correct implementations of the $n$ basis-ket, Hamming weight, and distinct elements programs. 
All other operations in the figure were validated with Quick\-Chick. To ensure these tests were efficacious, besides our program development procedures above, we also confirmed they could find hand-injected bugs; e.g., we changed the rotation angles in the \cn{Ry} gate in the amplitude amplification state preparation and confirmed that our PBT could catch the inserted bugs.
The tables in \Cref{fig:self-data} give the running times for our validator to validate programs---the times include the cost of extracting the Roqc code to OCaml, compiling it, and running it with $10,000$ randomly generated inputs via QuickChick.
We validated these programs on small (8-qubit) and large size (60-qubit) inputs (the number relevant to the reported qubit and gate sizes in \Cref{fig:qiskit-data}), with all the validation happening within 2.5 minutes (most of them are finished within seconds).
%Most tests are completed in a few seconds, while the test for the distinct element superposition preparation program finishes in a few minutes. 
For comparison, we translated our programs to \sqir, converted the \sqir programs to OpenQASM 2.0 \cite{Cross2017}, and then attempted to simulate the resulting circuits on a \textit{single test input} using DDSim~\cite{ddsim}, a state-of-the-art quantum simulator, and list the result in the fifth column. Unsurprisingly, the simulation of the 60-bit versions did not complete when running overnight.
The third and fourth columns in \Cref{fig:self-data} show the results for executing a \textit{single program run} in Qiskit, and Qiskit executes a few small-size programs (\Cref{sec:testefficient}) and none of the large-size programs.
%As we mentioned above, we implement these programs in Qiskit, which executes not a single run of a large size program.
The experiment provided a good degree of assurance of the scalability of QSV.
%The third column in \Cref{fig:self-data} shows whether or not the program is executable in DDSim, which accepts OpenQASM programs.
%As we mentioned above, none of the programs are classically simulatable.
%The fourth column in \Cref{fig:self-data} shows whether or not the program is executable using Qiskit. 
%It is almost impossible for these frameworks to test the quantum programs, having the level of circuit size as well as comprehension as our example programs.

\begin{wrapfigure}{r}{5.5cm}
{\footnotesize
\begin{center}
\begin{tabular}{| l | c |}
\hline
 Operation  & QCT  \\
 \hline
Addition & 2  \\
Comparison & 5  \\
Modular Multiplication & 794 \\
\hline                           
\end{tabular}
\end{center}
}
\caption{Arith operation QC time (60B).}
%\vspace*{-1em}
\label{fig:runningtime}
\end{wrapfigure}

There is a difference in the program execution between DDSim/Qiskit and QSV.
The latter abstracts the arithmetic operations and assumes that the operations can be dealt with in the previous VQO~\cite{oracleoopsla} framework, while DDSim executes the whole circuits generated from a QSV program.
To compare the effects, we list the QuickChick testing time (running 10,000 tests) of the operations used in our state preparation programs in \Cref{fig:runningtime};
such running time data was given in VQO.
The addition and comparison circuits do not greatly affect the execution of our \pqasm programs.
The validation of modular multiplication circuits might be costly, as our $60$-bit modular exponentiation contains $60$ modular multiplication operations.
However, a typical validation scheme might only validate the correctness of a costly subcomponent once and use its semantic property in validating other programs utilizing the subcomponents.
More importantly, the purpose of the experiment of DDSim/Qiskit executions is to show the state-of-the-art impossibility of executing quantum programs in a classical computer, while our QSV framework can validate quantum programs.

\myparagraph{Utility}
One of the utility of a program validation framework is to find bugs or faults in the existing algorithms.
During the quantum state preparation program development, we found several issues in two original algorithms \cite{Buhrman_2001,1366221} that utilize these special superposition states. The two algorithms both require the preparation of a superposition state of distinct elements (or the superposition state of permutations of distinct elements), but they do not specify how such a state can be effectively prepared. To the best of our knowledge, the state preparation program in \Cref{sec:distinctness} is the first program implementation of the state via the repeat-until-success scheme. As we can see in our probability analysis, the chance of preparing such a state is not very high.
This fact might indicate that the quantum algorithms advantage arguments over classical algorithms in these works might not be solid because of the unclear preparation for the initial states in these papers. Without our implementations for these state preparation programs, it is impossible to discover these delicate potential faults in these algorithms.

Indeed, in the algorithm \cite{10.5555/2011430.2011431} that uses the initial Hamming weight superposition state, the authors realized the potential low probability of preparing the initial state via the repeat-until-success scheme and pointed out the uses of a specialized gate instead of Hadamard gates, to start their repeat-until-success state preparation program. The special gates created a simple superposition state with a different probability distribution, rather than the uniform distribution in the case of using Hadamard gates. The analysis of these specialized superposition gates with different probability distributions will be included in our future work.

Another utility of using QSV is to judge the implemented programs' correctness and find a more optimized implementation.
In QSV, we can perform the tasks by using our PBT framework.
In other frameworks, such as OpenQASM, nothing fundamentally stopped the OpenQASM developers from making the same choices. Still, we note they did not have the benefit of the \pqasm type system and PBT framework.

\section{Related Work}
\label{sec:related}

This section gives related work beyond the discussion in \Cref{sec:implementation}.
%Essentially, \qafny is qualitatively different from the other works, so we mainly use the quantitative evaluation to explain the critical differences between \qafny and other frameworks.

\noindent\textbf{\textit{Quantum Circuit Languages.}}
Prior research has developed circuit-level compilers to compile quantum circuit languages to quantum computers, such as Qiskit \cite{Qiskit2019}, \tket \cite{tket}, Staq \cite{Amy2020}, PyZX \cite{Kissinger2019}, Nam \emph{et al.} \cite{Nam2018}, quilc~\cite{quilc}, Cirq~\cite{cirq}, ScaffCC \cite{JavadiAbhari2015}, and Project Q~\cite{Steiger_2018}. 
%None of the optimization or mapping code in these compilers is formally verified.
In addition, many quantum programming languages have been developed in recent years. 
Many of these languages (e.g. Quil~\cite{quilc}, OpenQASM ~\cite{Cross2017,10.1145/3505636}, \sqir~\cite{VOQC}) describe low-level circuit programs.
Higher-level languages may provide library functions for performing common oracle operations (e.g., Q\# \cite{qsharp}, Scaffold~\cite{scaffold,scaffCCnew}) or support compiling from classical programs to quantum circuits (e.g., Quipper~\cite{10.1145/2491956.2462177}), but still leave some important details 
%\myra{what is uncomputation of ancilla qubits?} \liyi{modified}
(like deallocating extra intermediate qubits) to the programmer.
There has been some work on type systems to enforce that deallocation happens correctly (e.g., Silq~\cite{sliqlanguage}) and on automated insertion of deallocation circuits (e.g., Quipper~\cite{10.1145/2491956.2462177}, Unqomp~\cite{unqomp}), but while these approaches provide useful automation, they may also lead to inefficiencies in compiled circuits.

\ignore{
Previously, formal verification has been applied to parts of the quantum compiler stack, but has not supported general quantum programs.
Amy \emph{et al.}~\cite{reverC}, and Rand \emph{et al.} \cite{Rand2018,Rand2017} developed certified compilers from source Boolean expressions to reversible circuits. 
% The problem of optimization verification has also been considered in the context of the ZX-calculus~\cite{Coecke2011}, which is characterized by a small set of rewrite rules that allow translation of a diagram to any other diagram representing the same computation~\cite{Jeandel2018}. 
Fagan and Duncan \cite{Fagan2018} verified an optimizer for ZX diagrams representing Clifford circuits (which use the non-universal gate set $\{{CX}, H, S\}$). 
Tao \emph{et al.} \cite{Tao2022} developed Giallar, a verification toolkit used to verify transformations in the Qiskit compiler. 

This research on compilers contains some degree of optimization, and nearly all emphasize satisfying architectural requirements, like mapping to a particular gate set or qubit topology. Recent work has looked at verified optimization of quantum circuits (e.g., \voqc~\cite{VOQC}, CertiQ~\cite{Shi2019}), but these works focus on optimizations towards reducing gate counts assuming gates have formal semantics rather than optimizing towards what happens in different machines.
Additionally, all the above languages describe operations regarding assembly gate structures instead of providing high-level algorithm specifications or program abstractions. \qafny \cite{li2024qafny} represents the first step in providing high-level program abstractions for quantum gate operations. 
%Compared to these tools, \name supports both a higher-level classical source language (\vqimp) and a more interesting quantum target language (\vqir).
}

\noindent\textbf{\textit{Quantum Software Testing and Validation.}}
%
There have been many approaches developed for validating quantum programs \cite{morphq_bugs,fuzz4all,10.1109/ASE51524.2021.9678798,fortunato,long:24,QDiff} including the use differential ~\cite{QDiff} and metamorphic testing ~\cite{10.1109/ICSE48619.2023.00202}, as well as mutation testing ~\cite{fortunato} and fuzzing ~\cite{fuzz4all}. Some key challenges exist for testing quantum programs. First, their input space explodes due to superposition. Second, their results are probabilistic (meaning we need to use statistical measures and/or other approaches to evaluate results).  Last, the expected result may be difficult or even impossible to determine. To date, the testing approaches have focused on validating small circuit subroutines (in the sense of having limited input qubit size) rather than testing comprehensive quantum programs, and they are all limited to performing tests in Qiskit, which might not capture all of the machine limitations.
%To perform gate-level equivalence checking, all the quantum circuit compilers and optimizers (\voqc, CertiQ, Quartz, ScaffCC, etc) contain sub-components for the tasks.

\noindent\textbf{\textit{Methodologies Possibly Used for Validating Quantum Programs.}}
%
SymQV \cite{10.1007/978-3-031-27481-7_12} proposed a method of encoding quantum states and gates as SMT-solvable predicates to perform automated verification.
Chen \emph{et al.} \cite{10.1145/3591270} and Abdulla \emph{et al.} \cite{abdulla2024verifyingquantumcircuitslevelsynchronized} used tree automata to symbolize quantum gates, instead of quantum states, and utilized tree automata to construct a tree structure for easing automated verification. These works can handle some large programs, but these programs have simple program structures, such as QFT.
Mei \emph{et al.} \cite{10.1007/978-3-031-65633-0_25} performed quantum stabilizer simulation based on the Gottesman–Knill theorem, which is a small subset of quantum programs and mainly used for error correction programs. Quasimodo \cite{10.1007/978-3-031-37709-9_11} is another symbolic execution based on a BDD-like structure to symbolize gates rather than states; their results are similar to the tree automata works \cite{abdulla2024verifyingquantumcircuitslevelsynchronized,10.1145/3591270}.
\qafny \cite{li2024} transformed quantum program verification to Dafny for automated verification.
These works tried to transform states and gates to perform automated verification, different from QSV, which tries to perform program testing and validation. The methodologies are also different from QSV where they try to symbolize quantum gates and states, while QSV only inserts special treatments in the standard quantum state representations. As a result, QSV can deal with large programs with comprehensive program structures.

\noindent\textbf{\textit{Verified Quantum Compilers.}}
%
%Recent work on formally verifying quantum programs includes \qwire~\cite{RandThesis}, \sqir~\cite{PQPC}, \qbricks~\cite{qbricks}, and \qafny \cite{li2024}. These tools have been used to verify a range of quantum algorithms, from Grover's search to quantum phase estimation. The program verification in these systems requires mannual labor.
%
%Like these tools, properties of \pqasm programs are expressed in a proof assistant.
%But, unlike these tools, we focus on a quantum sub-language that, while not able to express any quantum program, is automatically and efficiently testable for certain properties, such as correctness properties.
%This allows us to reuse existing infrastructure (like QuickChick~\cite{quickchick}) for testing Rocq properties.
% We design \vqir with both efficiency and verification in mind:  on one hand, \vqir allows users to build more efficient quantum circuit constructions by leveraging native quantum operations such as Hadamard and quantum Fourier transformation; on the other hand, 
% we identify a class of such circuit constructions whose semantics can be succinctly expressed and efficiently simulated, the specific form of which is enforced by a type system on \vqir. 
% The latter eases the verification of the compilation and enables
% random testing, for any well-formed \vqir program.
Recent work has looked at verified optimization of quantum circuits (e.g., \voqc~\cite{VOQC}, CertiQ~\cite{Shi2019}), but the problem of verified \emph{compilation} from high-level languages to quantum circuits has received less attention.
The only examples of verified compilers for quantum circuits are ReVerC~\cite{reverC} and ReQWIRE~\cite{Rand2018ReQWIRERA}.
Both of these tools support verified translation from a low-level Boolean expression language to circuits consisting of \texttt{X}, \texttt{CNOT}, and \texttt{CCNOT} gates.
%Compared to these tools, \pqasm supports the compilation of a generalized quantum state preparation program to low-level quantum circuits.
VQO \cite{oracleoopsla} is a certified compilation framework for verifying and compiling quantum arithmetic operations.
QSV utilizes VOQC \cite{VOQC} and VQO \cite{oracleoopsla} to compile \pqasm programs to quantum circuits.

\ignore{
\noindent\textbf{\textit{Oracles in Quantum Languages.}}
%
Many works provided library functions for performing common oracle operations (e.g., Q\# \cite{qsharp}, Scaffold~\cite{scaffold,scaffCCnew}) or support compiling from classical programs to quantum circuits (e.g., Quipper~\cite{Green2013}), but still leave some important details (like uncomputation of ancilla qubits) to the programmer.
%
\name \cite{oracleoopsla} is similar to Quipper but provides verification for many oracle operations based on different quantum techniques, such as QFT-based arithmetic operations. 
The key difference between QSV and any of the frameworks is that quantum arithemtic operations provide the ability to write solely quantum arithmetic operations,
such as the comparator in \Cref{fig:circuits} and the modular multiplication operations in \Cref{sec:modmult},
while QSV provides the ability to define state preparation programs as well as quickly validate them.
}

% VARIOUS OLD TEXT:

% Quipper's approach is efficacious, but it can be inefficient and risks
% bugs. It compiles to a circuit whose gates do not leverage a quantum
% computer's specific capabilities. For example, addition between
% integers compiles to a classical ripple-carry adder rather than one
% based on the \emph{quantum fourier transform} (QFT), which can be more
% space-efficient. Quipper's compilation strategy also blows up the use
% of ancillae. For example, implementing cosine as a Haskell function
% and then building a Quipper circuit from it uses $n^2$ ancilla qubits
% for an $n$ qubit-encoded number, and usage increases linearly with the
% number of steps of the Taylor expansion. Of course, programmers are
% not obligated to use the above recipe for constructing oracles---they
% can do it by hand for greater efficiency---but this risks
% mistakes. While writing this paper we found a bug in Quipper's adder:
% When adding numbers of two different precisions, the lower-precision
% number is shifted incorrectly.\footnote{The \texttt{k} on the last line of
%   \texttt{qdouble\_align} should be \texttt{h}; \url{https://www.mathstat.dal.ca/~selinger/quipper/doc/src/Quipper/Algorithms/QLS/QDouble.html\#line-413}}

% \item Quipper~\cite{10.1145/2491956.2462177} -- see section 4.6. It
%   uses Template Haskell to take a Haskell function $f$ of type
%   \emph{list of bool} $\rightarrow$ \emph{list of bool} (or just a
%   single \emph{bool}), and converts it to $U_f$ for a fixed number of
%   qubits. A subsequent step ``uncomputes'' any ancillae that are no
%   longer needed. Note that Quipper has implemented \texttt{sin},
%   \texttt{cos}, etc. But: This is low-level, since the programmer must
%   manage lists of (physical) qubits and ancillae. No
%   verification. Look at current code and see what it can do now?

% Quipper \cite{Green2013} is a Haskell-like functional quantum language. Many quantum oracles have been defined in Quipper. Users are able to generate quantum circuits by using the Quipper compiler. We have mentioned several Quipper limitations in Sec.~\ref{sec:evaluation}. The major limitations are two. First, the circuits generated from Quipper oracles are not effective in terms of qubits and gates, and most Quipper oracle definitions are not verified. Quipper has a new development of compiling the language to QPMC \cite{Anticoli2017}, which is a model checker that is capable of verifying algorithms defined in Quipper. However, the oracles defined in Quipper are largely not verified. 

% \myparagraph{Programming Experience}
% %
% \sourcelang provides programmability as well as performance
% benefits. It is a higher-level language based on C, which makes
% writing oracles easier than in circuit construction languages such as
% in SQIR, Cirq, Qiskit, and OpenQASM \cite{cross2021openqasm}. Quipper
% supports writing oracles using circuit combinators, too, but it also
% permits writing an oracle as a Haskell function directly, leveraging
% Template Haskell \cite{Green2013} to compile that function to a
% (Toffoli-based) circuit. This puts it on a similar level of usability
% as \sourcelang. One benefit of \name is its support for randomized
% testing, allowing programmers to troubleshoot their applications and
% debug their programs more quickly, especially when quantum techniques
% like QFT are used.  
% \ignore{
% The following is a real-life example.  Before we
% had \name, we were writing a modular exponentiation circuit in
% \sqir. Initially, the circuit's performance was encouraging, but while
% spending weeks trying to prove it correct, we discovered and fixed
% various bugs. By the time we finished, performance had degraded
% substantially. With a few hours' worth of random testing using \name,
% we could have found the flaws and fixed them in a way that preserved
% performance and maintained confidence in correctness \emph{before}
% attempting any time-consuming verification.
% } 
% \liyi{I now think we might not need that example.
%  Since the whole paper is talking about the benefits of random testing. so it should be clear enough at this point.
% So we just cut the example.}
% \mwh{I don't understand
%   this anecdote. This example doesn't have to do with OQIMP; it's
%   about writing in a circuit language (i.e., prior to \name). If you
%   wrote in OQIMP you could just simulate the circuit directly to test
%   it. So the random testing benefit is really just about OQASM
%   circuits. So what does that mean for overall claims about usability
%   benefits? Also, Quipper provides no verification, and this example
%   is about time taken to verify.}
% \mwh{Idea: move this paragraph to the related work section}

% ReVerC \cite{reverC} is a language for writing reversible Boolean expressions by using \texttt{X}, \texttt{CNOT}, and \texttt{CCX} gates, which is similar to RKQC. It has a compiler to compile a relatively high-level reversible language (Revs) \cite{parent2015reversible} to circuits, and it is verified. The limitation of ReverC is that the language it supports is a relatively low-level Boolean expression language. Even though the compiler is verified and contains several examples of defining arithmetic operations, the operations are not verified yet. Additionally, ReverC is a reversible language and it does not provide a connection between quantum algorithms and quantum arithmetic oracles, as we did in Sec.~\ref{sec:grover-search}. 

% % Some prior work has also applied formal methods to compilation of
% % oracles. ReVerC is a formally verified compiler for reversible
% % circuits, but the input language is only boolean expressions, not
% % integers or decimal numbers, and compilation is only to classical
% % gates. ReQWIRE has similar limitations.

% \myparagraph{Entanglement in Quantum Languages}

% Quantum entanglement is an important feature of quantum programs, and also a useful tool for reasoning about quantum programs \cite{quantumseparation,Yuan2022}.
% Unfortunately, entanglement detection is at least an NP-hard problem \cite{entanglementdetection}.
% By design, an \oqasm program can never introduce entanglement.

\section{Conclusion and Limitations}
\label{sec:conclusion}

We presented QSV, a framework for expressing and automatically validating quantum state preparation programs.
The core of QSV is a quantum language \pqasm, which can express a restricted class of quantum programs that are efficiently testable for certain properties and are useful for implementing state preparation programs. 
We have verified the translator from \pqasm to \sqir and have validated (or randomly tested) many programs written in \pqasm.
We have used \pqasm to implement state preparation programs useful in quantum computation, such as the ones in \Cref{fig:qiskit-data}.
We hope this work will be the basis for building a quantum validation framework for validating quantum programs on classical computers.

Our QSV is capable of defining most quantum program patterns with program validation. As mentioned in \Cref{sec:intro}, QSV targets validating state preparation programs. For almost all quantum programs, QSV is able to validate the most significant part of the program. For example, the validated modular multiplication program in \Cref{fig:mod-mult} is essentially 90\% of Shor's algorithm, except for the final inverse QFT gate and measurement.

Our type system specifically locates Hadamard operations, but there are no actual restrictions on Hadamard operations, as users can easily define similar behaviors via our \cn{Ry} and oracle operations. Via our type system, we identify the beginning Hadamard operations as a general quantum algorithm component to generate superposition sources so that QSV can locate the places to create random inputs for validating programs. We recognize the superposition state generation in many quantum algorithms as the major bottleneck for testing quantum programs; therefore, we utilize types to identify them, with special treatment to transform the superposition states to a simple and testable format.



%\input{semantics}

%%Acknowledgments
%\begin{acks}                            %% acks environment is optional
                                        %% contents suppressed with 'anonymous'
  %% Commands \grantsponsor{<sponsorID>}{<name>}{<url>} and
  %% \grantnum[<url>]{<sponsorID>}{<number>} should be used to
  %% acknowledge financial support and will be used by metadata
  %% extraction tools.
  % This material is based upon work supported by the
  % \grantsponsor{GS100000001}{National Science
  %   Foundation}{http://dx.doi.org/10.13039/100000001} under Grant
  % No.~\grantnum{GS100000001}{nnnnnnn} and Grant
  % No.~\grantnum{GS100000001}{mmmmmmm}.  Any opinions, findings, and
  % conclusions or recommendations expressed in this material are those
  % of the author and do not necessarily reflect the views of the
  % National Science Foundation.
 % We thank Finn Voichick for his helpful comments and contributions during the development of this work.
 % In memory of Rance Cleaveland.
%\end{acks}


%% Bibliography
\bibliography{reference}
%\newpage
%\appendix
%\section{OQASM: An Assembly Language for Quantum Oracles}
\label{sec:vqir}

The \oqasm expression $\mu$ used in \Cref{fig:pqasm} places an additional wrapper on top of the \oqasm expression $\iota$ given in \Cref{fig:vqir}. Here, we first provide a step-by-step explanation of \oqasm.

\oqasm is designed to express efficient quantum
oracles that can be easily tested and, if desired, proved
correct.
\oqasm operations leverage both the standard
computational basis and an alternative basis connected by the quantum
Fourier transform (QFT). 
\oqasm's type system tracks the bases of variables in
\oqasm programs, forbidding operations that would introduce
entanglement. \oqasm states are therefore efficiently
represented, so programs can be effectively tested and are simpler to
verify and analyze. In addition, \oqasm uses \emph{virtual qubits}
to support \emph{position shifting operations}, which support
arithmetic operations without introducing extra gates during
translation. All of these features are novel to quantum assembly
languages. 

This section presents \oqasm states and the language's syntax,
semantics, typing, and soundness results. As a running example, the QFT
adder~\cite{qft-adder} is shown in \Cref{fig:circuit-exampleb}. The Roqc
function \coqe{rz_adder} generates an \oqasm program that adds two
natural numbers \coqe{a} and \coqe{b}, each of length \coqe{n} qubits.

\begin{figure*}[t]
  \centering
  \begin{tabular}{c @{\qquad} c}

  \begin{minipage}[b]{.6\textwidth}
  % \includegraphics[width=1\textwidth]{qft-adder.png}
  {\scriptsize
    \Qcircuit @C=0.5em @R=0.75em {
      \lstick{\ket{a_{n-1}}} & \qw & \ctrl{5} & \qw & \qw & \qw & \qw & \qw & \qw & \qw & \rstick{\ket{a_{n-1}}} \\
      \lstick{\ket{a_{n-2}}} & \qw & \qw & \ctrl{4} & \qw & \qw & \qw & \qw & \qw & \qw & \rstick{\ket{a_{n-2}}}\\
      \lstick{\vdots} & & & & & & & & & & \rstick{\vdots} \\
      \lstick{} & & & & & & & & & & \\
      \lstick{\ket{a_0}} & \qw & \qw & \qw & \qw & \qw & \qw & \ctrl{1} & \qw & \qw & \rstick{\ket{a_0}} \\
      \lstick{\ket{b_{n-1}}} & \multigate{5}{\texttt{QFT}} & \gate{\texttt{SR 0}} & \multigate{3}{\texttt{SR 1}} & \qw & \qw & \qw & \multigate{5}{\texttt{SR (n-1)}} & \multigate{5}{\texttt{QFT}^{-1}} & \qw & \rstick{\ket{a_{n-1} + b_{n-1}}} \\
      \lstick{} & & & & & \dots & & & & \\
      \lstick{\ket{b_{n-2}}} & \ghost{\texttt{QFT}} & \qw  &  \ghost{\texttt{SR 1}} & \qw & \qw & \qw & \ghost{\texttt{SR (n-1)}} & \ghost{\texttt{QFT}^{-1}} & \qw & \rstick{\ket{a_{n-2} + b_{n-2}}} \\
      \lstick{\vdots} & & & & & & & & & & \rstick{\vdots} \\
      \lstick{} & & & & & & & & & & \\
      \lstick{\ket{b_0}} & \ghost{\texttt{QFT}} & \qw & \qw & \qw & \qw & \qw & \ghost{\texttt{SR (n-1)}} & \ghost{\texttt{QFT}^{-1}}  & \qw & \rstick{\ket{a_0 + b_0}} 
      }
      }
  \subcaption{Quantum circuit}
  \end{minipage}
  \hfill\hfill
  \begin{minipage}[b]{.38\textwidth}
  \begin{coq}
  Fixpoint rz_adder' (a b:var) (n:nat) 
    := match n with 
       | 0 => ID (a,0)
       | S m => CU (a,m) (SR m b); 
                rz_adder' a b m
       end.
  Definition rz_adder (a b:var) (n:nat) 
    := Rev a ; Rev b ; $\texttt{QFT}$ b ;
       rz_adder' a b n;
       $\texttt{QFT}^{-1}$ b; Rev b ; Rev a.
  \end{coq}
  \subcaption{\oqasm metaprogram (in Roqc)}
  \end{minipage}
  \end{tabular}
  %\vspace{-0.5em}
  \caption{Example \oqasm program: QFT-based adder}
  \label{fig:circuit-exampleb}
  \end{figure*}

\subsection{OQASM States} \label{sec:pqasm-states}

\begin{figure}[t]
  \small
  \[\hspace*{-0.5em}
\begin{array}{l>{$} p{1.2cm} <{$} c l}
      \text{Bit} & b & ::= & 0 \mid 1 \\
      \text{Natural number} & n & \in & \mathbb{N} \\
      \text{Real} & r & \in & \mathbb{R}\\
      \text{Phase} & \alpha(r) & ::= & e^{2\pi i r} \\
      \text{Basis} & \tau & ::= & \texttt{Nor} \mid \texttt{Phi}\;n \\
      \text{Unphased qubit} & \overline{q} & ::= & \ket{b} ~~\mid~~ \qket{r} \\
      \text{Qubit} & q & ::= &\alpha(r) \overline{q}\\
      \text{State (length $d$)} & \varphi & ::= & q_1 \otimes q_2 \otimes \cdots \otimes q_d
    \end{array}
  \]
  \caption{\oqasm state syntax}
  \label{fig:vqir-state}
\end{figure}

An \oqasm program state is represented according to the grammar in
\Cref{fig:vqir-state}. A state $\varphi$ of $d$ qubits is 
a length-$d$ tuple of qubit values $q$; the state models the tensor
product of those values. This means that the size of $\varphi$ is
$O(d)$ where $d$ is the number of qubits. A $d$-qubit state in a
language like \sqir is represented as a length $2^d$ vector of complex
numbers, $O(2^d)$ in the number of qubits. Our linear state
representation is possible because applying for any well-typed \oqasm
program on any well-formed \oqasm state never causes qubits to be
entangled.

A qubit value $q$ has one of two forms $\overline{q}$, scaled by a
global phase $\alpha(r)$. The two forms depend on the \emph{basis} $\tau$ that the qubit is in---it could be either \texttt{Nor} or \texttt{Phi}. A \texttt{Nor} qubit has form
$\ket{b}$ (where $b \in \{ 0, 1 \}$), which is a
computational basis value. 
A \texttt{Phi} qubit has the form $\qket{r} = \frac{1}{\sqrt{2}}(\ket{0}+\alpha(r)\ket{1})$, which is a value of the (A)QFT basis.
The number $n$ in \texttt{Phi}$\;n$ indicates the precision of the state $\varphi$.
As shown by~Beauregard \cite{qft-adder}, arithmetic on the computational basis can sometimes be more efficiently carried out on the QFT basis, which leads to the use of quantum operations (like QFT) when implementing circuits with classical input/output behavior.

\subsection{OQASM Syntax, Typing, and Semantics}\label{sec:oqasm-syn}

\begin{figure}[t]
\begin{minipage}[t]{0.5\textwidth}
{\small \centering

  $ \hspace*{-0.8em}
\begin{array}{llcl}
      \text{Position} & p & ::= & (x,n) \qquad   \text{Nat. Num}~n
                                  \qquad   \text{Variable}~x\\
      \text{Instruction} & \instr & ::= & \iskip{p} \mid \inot{p}
                                          \mid \irz[\lbrack -1 \rbrack]{n}{p} \mid \iseq{\instr}{\instr}\\
                & & \mid &  \isr[\lbrack -1 \rbrack]{n}{x} \mid \iqft[\lbrack -1 \rbrack]{n}{x} \mid \ictrl{p}{\instr}  \\
                      & & \mid & \ilshift{x} \mid \irshift{x} \mid \irev{x} 
    \end{array}
  $
}
  \caption{\oqasm syntax. For an operator \texttt{OP}, $\texttt{OP}^{\lbrack -1 \rbrack}$ indicates that the operator has a built-in inverse available.}
  \label{fig:vqir}
\end{minipage}
\hfill
\begin{minipage}[t]{0.38\textwidth}
{\scriptsize
\centering
\begin{tabular}{c@{$\quad=\quad$}c}
  \begin{minipage}{0.3\textwidth}

%\includegraphics[width=0.3\textwidth]{sr-meaning.png}
  \Qcircuit @C=0.5em @R=0.5em {
    \lstick{} & \qw     & \multigate{4}{\texttt{SR m}} & \qw & \qw \\
    \lstick{} & \qw     & \ghost{\texttt{SR m}}           & \qw & \qw \\
    \lstick{} & \vdots & & \vdots & \\
    \lstick{} & & & & \\
    \lstick{} & \qw     & \ghost{\texttt{SR m}}           & \qw  & \qw
    }
  \end{minipage} & 
  \begin{minipage}{0.3\textwidth}
  \small
  \Qcircuit @C=0.5em @R=0.5em {
    \lstick{} & \qw     & \gate{\texttt{RZ (m+1)}} & \qw & \qw \\
    \lstick{} & \qw     & \gate{\texttt{RZ m}}          & \qw & \qw \\
    \lstick{} & & \vdots & & \\
    \lstick{} & & & & \\
    \lstick{} & & & & \\
    \lstick{} & \qw     & \gate{\texttt{RZ 1}}           & \qw  & \qw
    }
 
  \end{minipage} 
\end{tabular}
}
\caption{\texttt{SR} unfolds to a series of \texttt{RZ} instructions}
\label{fig:sr-meaning}
\end{minipage}
\end{figure}

\Cref{fig:vqir} presents \oqasm's syntax. An \oqasm program consists of
a sequence of instructions $\instr$. Each instruction applies an
operator to either a variable $x$, representing a group of qubits, or a \emph{position} $p$, which identifies a particular offset into a variable $x$. 

The instructions in the first row correspond to simple single-qubit
quantum gates---$\iskip{p}$, $\inot{p}$, and $\irz[\lbrack -1 \rbrack]{n}{p}$
 ---and instruction sequencing.
The instructions in the next row apply to whole variables: $\iqft{n}{x}$
applies the AQFT to variable $x$ with $n$-bit precision and
$\iqft[-1]{n}{x}$ applies its inverse.
If $n$ equals the size of $x$, then the AQFT operation is exact.
$\isr[\lbrack -1 \rbrack]{n}{x}$
applies a series of \texttt{RZ} gates (\Cref{fig:sr-meaning}). 
Operation $\ictrl{p}{\instr}$
applies instruction $\instr$ \emph{controlled} on qubit position
$p$. All of the operations in this row---\texttt{SR}, \texttt{QFT}, and \texttt{CU}---will be translated to multiple \sqir
gates. The function \coqe{rz_adder} in \Cref{fig:circuit-exampleb}(b) uses
many of these instructions; e.g., it uses \texttt{QFT} and \texttt{QFT}$^{-1}$ and applies
\texttt{CU} to the $m$th position of variable \texttt{a} to control
instruction \texttt{SR m b}.

In the last row of \Cref{fig:vqir}, instructions $\ilshift{x}$,
$\irshift{x}$, and $\irev{x}$ are \emph{position shifting operations}.
Assuming that $x$ has $d$ qubits and $x_k$ represents the $k$-th qubit
state in $x$, $\texttt{Lshift}\;x$ changes the $k$-th qubit state to
$x_{(k + 1)\mmod d}$, $\texttt{Rshift}\;x$ changes it to
$x_{(k + d - 1)\mmod d}$, and \texttt{Rev} changes it to $x_{d-1-k}$. In
our implementation, shifting is \emph{virtual}, not physical. The \oqasm
translator maintains a logical map of variables/positions to concrete
qubits and ensures that shifting operations are no-ops, introducing no extra gates.

Other quantum operations could be added to \oqasm to
allow reasoning about a larger class of quantum programs while still
guaranteeing a lack of entanglement. 

\begin{figure}[t]
\begin{minipage}[b]{0.57\textwidth}
{\scriptsize
  \begin{mathpar}
    \inferrule[X]{\Omegaty(x)=\texttt{Nor} \\ n < \Omegasz(x)}{\Sigma;\Omega \vdash \inot{(x,n)}\triangleright \Omega}
  
    \inferrule[RZ]{\Omegaty(x)=\texttt{Nor} \\ n < \Omegasz(x)}{\Sigma;\Omega \vdash \irz{q}{(x,n)} \triangleright \Omega}

    \inferrule[SR]{\Omegaty(x)=\tphi{n} \\ m < n}{\Sigma;\Omega \vdash \texttt{SR}\;m\;x\triangleright \Omega}   

    \inferrule[QFT]{\Omegaty(x)=\texttt{Nor}\\n \le \Omegasz(x)}{\Sigma; \Omega \vdash \iqft{n}{x}\triangleright \Omega[x\mapsto \tphi{n}]}    
     
    \inferrule[RQFT]{\Omegaty(x)=\tphi{n}\\n \le \Omegasz(x)}{\Sigma; \Omega \vdash \iqft[-1]{n}{x}\triangleright \Omega[x\mapsto \texttt{Nor}]}             
    
    \inferrule[CU]{\Omegaty(x)=\texttt{Nor} \\ \texttt{fresh}~(x,n)~\instr \\\\ \Sigma; \Omega\vdash \instr\triangleright \Omega \\ \texttt{neutral}(\instr)}{\Sigma; \Omega \vdash \texttt{CU}\;(x,n)\;\instr \triangleright \Omega} 
     
    \inferrule[LSH]{\Omegaty(x)=\texttt{Nor}}{\Sigma; \Omega \vdash \texttt{Lshift}\;x\triangleright \Omega}

     \inferrule[SEQ]{\Sigma; \Omega\vdash \instr_1\triangleright \Omega' \\ \Sigma; \Omega'\vdash \instr_2\triangleright \Omega''}{\Sigma; \Omega \vdash \instr_1\;;\;\instr_2\triangleright \Omega''} 
    
  \end{mathpar}
}
  \caption{Select \oqasm typing rules}
  \label{fig:exp-well-typeda}
\end{minipage}
\hfill
\hfill
\begin{minipage}[b]{0.4\textwidth}
{\footnotesize
\begin{center}\hspace*{-1em}
\begin{tikzpicture}[->,>=stealth',shorten >=1pt,auto,node distance=3.2cm,
                    semithick]
  \tikzstyle{every state}=[fill=black,draw=none,text=white]

  \node[state] (A)              {$\texttt{Nor}$};
  \node[state]         (C) [left of=A] {$\tphi{n}$};

  \path (A) edge [loop above]            node {$\Big\{\begin{array}{l}\texttt{ID},~\texttt{X},~\texttt{RZ}^{\lbrack -1 \rbrack},~\texttt{CU},\\
              \texttt{Rev},\texttt{Lshift},\texttt{Rshift}\end{array}\Big\}$} (A)
            edge   node [above] {\{$\texttt{QFT}\;n$\}} (C);
  \path (C) edge [loop above]            node {$\{\texttt{ID},~\texttt{SR}^{\lbrack -1 \rbrack}\}$} (C)
            edge  [bend right]             node {$\{\texttt{QFT}^{-1}\;n\}$} (A);
\end{tikzpicture}
\end{center}
}
\caption{Type rules' state machine}
\label{fig:state-machine}
\end{minipage}
\end{figure}

\myparagraph{Typing}
\label{sec:vqir-typing}

In \oqasm, typing is concerning a \emph{type environment}
$\Omega$ and a predefined \emph{size
  environment} $\Sigma$, which map \oqasm
variables to their basis and size (number of qubits), respectively.
The typing judgment is written $\Sigma; \Omega\vdash \instr \triangleright \Omega'$ which
states that $\instr$ is well-typed under $\Omega$ and $\Sigma$, and
transforms the variables' bases to be as in $\Omega'$ ($\Sigma$ is unchanged). 
$\Sigma$ is fixed because the number of qubits in execution is always fixed.
It is generated in the high-level language compiler, such as \sourcelang in \cite{oracleoopsla}.
The algorithm generates $\Sigma$ by taking an \sourcelang program and scanning through
all the variable initialization statements.
Select type rules are given in \Cref{fig:exp-well-typeda}; 
the rules not shown (for \texttt{ID}, \texttt{Rshift}, \texttt{Rev}, \texttt{RZ}$^{-1}$, and \texttt{SR}$^{-1}$) are similar.

The type system enforces three invariants. First, it enforces that
instructions are well-formed, meaning that gates are applied to valid
qubit positions (the second premise in \rulelab{X}) and that any control qubit is distinct from the
target(s) (the \texttt{fresh} premise in
\rulelab{CU}).  This latter property enforces the quantum
\emph{no-cloning rule}.
For example, applying the \texttt{CU} in \code{rz\_adder'} (\Cref{fig:circuit-exampleb}) is valid
because position \code{a,m} is distinct from variable \code{b}.

Second, the type system enforces that instructions leave affected
qubits on a proper basis (thereby avoiding entanglement). The
rules implement the state machine shown in
\Cref{fig:state-machine}. For example, $\texttt{QFT}\;n$ transforms a variable from \texttt{Nor} to
$\tphi{n}$ (rule \rulelab{QFT}), while $\texttt{QFT}^{-1}\;n$
transforms it from $\tphi{n}$ back to \texttt{Nor} (rule
\rulelab{RQFT}). Position shifting operations 
are disallowed on variables $x$ in
the \texttt{Phi} basis because the qubits that makeup $x$ are
internally related (see \Cref{appx:well-formed}) and cannot be rearranged. Indeed, applying a
\texttt{Lshift} and then a $\texttt{QFT}^{-1}$ on $x$ in \texttt{Phi}
would entangle $x$'s qubits.

% \begin{figure}[t]
% {\footnotesize
% \begin{center}
% \begin{tikzpicture}[->,>=stealth',shorten >=1pt,auto,node distance=3.2cm,
%                     semithick]
%   \tikzstyle{every state}=[fill=white,draw=black,text=black]
% 
%   \node[initial,accepting,state] (A)              {$\texttt{OK}$};
%   \node[state]         (B) [right of=A] {$ $};
% 
%   \path (A) edge [loop above]            node {$b,\epsilon / \epsilon$} (A)
%             edge  [above] node {$a,\emptyset / a$} (B);
%   \path (B) edge [loop right]            node [right] {$\begin{array}{l}b,\epsilon / \epsilon\\
%                                                                 a,a' / a a'\\
%                                                                 a,\overline{a} / \epsilon\\
%                                                  \end{array}$} (B)
%             edge  [bend left]             node [above] {$\epsilon,\emptyset / \emptyset$} (A);
% \end{tikzpicture}
% \end{center}
% }
% {
% \footnotesize
% $
% \begin{array}{l}
% a,a'\in \{\ilshift{x},\irshift{x},\irev{x} \} \wedge a' \neq \overline{a}
% \\
% \overline{\ilshift{x}}=\irshift{x}
% \quad
% \overline{\irshift{x}}=\ilshift{x}
% \quad
% \overline{\irev{x}}=\irev{x}
% \\
% b\not\in\{\ilshift{x},\irshift{x},\irev{x}, \instr;\instr \}
% \\
% \emptyset=\text{ no element in stack}
% \end{array}
% $
% }
% 
% \caption{Pushdown automata for \texttt{neutral}}
% \label{fig:pushdown-neu}
% \end{figure}

Third, the type system enforces that the effect of position-shifting
operations can be statically tracked. The \texttt{neutral} condition of
\rulelab{CU} requires that any shifting within $\instr$ is restored by the time it
completes. 
For example, $\sseq{\ictrl{p}{(\ilshift{x})}}{\inot{(x,0)}}$ is not well-typed because knowing the final physical position of qubit $(x,0)$ would require statically knowing the value of $p$. 
On the other hand, the program $\sseq{\ictrl{c}{(\sseq{\ilshift{x}}{\sseq{\inot{(x,0)}}{\irshift{x}}})}}{\inot{(x,0)}}$ is well-typed 
because the effect of the \texttt{Lshift} is ``undone'' by an \texttt{Rshift} inside the body of the \texttt{CU}.

% \texttt{neutral}'s definition in \Cref{fig:pushdown-neu}
% views $\instr$ as a string concatenated
% by the sequence operation ($;$) and requires $\instr$ to be
% accepted according to a family of pushdown automatas $\{G\}_{x}$ for every $x$ presented in $\instr$. 
% A program $\instr$ is \texttt{neutral}, iff, $\instr$ as a string is
% accepted by all the automatas in $\{G\}_{x}$.

\myparagraph{Semantics}\label{sec:pqasm-dsem}

\begin{figure}[t]
{\scriptsize
\[
\begin{array}{lll}
\llbracket \iskip{p} \rrbracket\varphi &= \varphi\\[0.2em]

\llbracket \inot{(x, i)} \rrbracket\varphi &= \app{\uparrow\xsem(\downarrow\varphi(x,i))}{\varphi}{(x,i)}
& \texttt{where  }\xsem(\ket{0})=\ket{1} \qquad\, \xsem(\ket{1})=\ket{0}
\\[0.5em]

\llbracket \ictrl{(x,i)}{\instr} \rrbracket\varphi &=  \csem(\downarrow\varphi(x,i),\instr,\varphi)
&
\texttt{where  }
\csem({\ket{0}},{\instr},\varphi)=\varphi\quad\;\,
\csem({\ket{1}},{\instr},\varphi)=\llbracket \instr \rrbracket\varphi
\\[0.4em]

\llbracket \irz{m}{(x,i)} \rrbracket\varphi &= \app{\uparrow {\rsem}({m},\downarrow\varphi(x,i))}{\varphi}{(x,i)}
&\texttt{where  }{\rsem}(m,\ket{0})=\ket{0} \; \quad{\rsem}(m,\ket{1})=\alpha(\frac{1}{2^m})\ket{1}
\\[0.5em]

\llbracket \irz[-1]{m}{(x,i)} \rrbracket\varphi &= \app{\uparrow {\rrsem}({m},\downarrow\varphi(x,i))}{\varphi}{(x,i)}
 &\texttt{where  }{\rrsem}(m,\ket{0})=\ket{0}
\quad{\rrsem}(m,\ket{1})=\alpha(-\frac{1}{2^m})\ket{1}
\\[0.5em]

\llbracket \isr{m}{x} \rrbracket\varphi &
                                            \multicolumn{2}{l}{= \app{\uparrow \qket{r_i+\frac{1}{2^{m-i+1}}}}{\varphi}{\forall i \le m.\;(x,i)}
\qquad \texttt{when  }
\downarrow\varphi(x,i) = \qket{r_i}}\\[0.5em]

\llbracket \isr[-1]{m}{x} \rrbracket\varphi&\multicolumn{2}{l}{= \app{\uparrow \qket{r_i-\frac{1}{2^{m-i+1}}}}{\varphi}{\forall i \le m.\;(x,i)}
\qquad \texttt{when  }
\downarrow\varphi(x,i) = \qket{r_i}}\\[0.5em]

\llbracket \iqft{n}{x} \rrbracket\varphi &= \app{\uparrow\qsem(\Sigma(x),\downarrow\varphi(x),n)}{\varphi}{x}
& \texttt{where  }\qsem(i,\ket{y},n)=\bigotimes_{k=0}^{i-1}(\qket{\frac{y}{2^{n-k}}})
\\[0.5em]

\llbracket \iqft[-1]{n}{x} \rrbracket\varphi &=  \app{\uparrow\qsem^{-1}(\Sigma(x),\downarrow\varphi(x),n)}{\varphi}{x}
\\[0.5em]

\llbracket \ilshift{x} \rrbracket\varphi &= \app{{\psem}_{l}(\varphi(x))}{\varphi}{x}
&
\texttt{where  }{\psem}_{l}(q_0\otimes q_1\otimes \cdots \otimes q_{n-1})=q_{n-1}\otimes q_0\otimes q_1 \otimes \cdots
\\[0.5em]

\llbracket \irshift{x} \rrbracket\varphi &= \app{{\psem}_{r}(\varphi(x))}{\varphi}{x}
&
\texttt{where  }{\psem}_{r}(q_0\otimes q_1\otimes \cdots \otimes q_{n-1})=q_1\otimes \cdots \otimes q_{n-1} \otimes q_0
\\[0.5em]

\llbracket \irev{x} \rrbracket\varphi &= \app{{\psem}_{a}(\varphi(x))}{\varphi}{x}
&
\texttt{where  }{\psem}_{a}(q_0\otimes \cdots \otimes q_{n-1})=q_{n-1}\otimes \cdots \otimes q_0
\\[0.5em]

\llbracket \iota_1; \iota_2 \rrbracket\varphi &= \llbracket \iota_2 \rrbracket (\llbracket \iota_1 \rrbracket\varphi)
\end{array}
\]
}
{\scriptsize
$
\begin{array}{l}
\\[0.2em]
\downarrow \alpha(b)\overline{q}=\overline{q}
\qquad
\downarrow (q_1\otimes \cdots \otimes q_n) = \downarrow q_1\otimes \cdots \otimes \downarrow q_n
\\[0.2em]
\app{\uparrow \overline{q}}{\varphi}{(x,i)}=\app{\alpha(b)\overline{q}}{\varphi}{(x,i)}
\qquad \texttt{where  }\varphi(x,i)=\alpha(b)\overline{q_i}
\\[0.2em]
\app{\uparrow \alpha(b_1)\overline{q}}{\varphi}{(x,i)}=\app{\alpha(b_1+b_2)\overline{q}}{\varphi}{(x,i)}
\qquad \texttt{where  }\varphi(x,i)=\alpha(b_2)\overline{q_i}
\\[0.2em]
\app{q_x}{\varphi}{x}=\app{q_{(x,i)}}{\varphi}{\forall i < \Sigma(x).\;(x,i)}
\\[0.2em]
\app{\uparrow q_x}{\varphi}{x}=\app{\uparrow q_{(x,i)}}{\varphi}{\forall i < \Sigma(x).\;(x,i)}
\end{array}
$
}
%\vspace*{-0.5em}
\caption{\oqasm semantics}
  \label{fig:deno-sema}
\end{figure}

The semantics of an \oqasm program is a partial function
$\llbracket\rrbracket$ from
an instruction $\instr$ and input state $\varphi$ to an output state
$\varphi'$, written 
$\llbracket \instr \rrbracket\varphi=\varphi'$, shown in \Cref{fig:deno-sema}.
% The definition for $\llbracket\rrbracket$ is syntax-driven, meaning that it is defined in terms of the state syntax presented in \Cref{fig:vqir-state}.

% defines the denotational semantics of \oqasm, which maps a \oqasm instruction $\instr \in \{\instr\}$ to its unitary operator on $\varphi \in \hsp{S}^d$.

% The key takeaway of the \oqasm denotational semantics is that given an input $\varphi \in \hsp{S}^d$, a well-typed instruction affects only one qubit (notation: $\varphi{(x,n)}$ or $q_{(x,n)}$) or qubit array (notation: $\varphi{(x)}$ or $q_x$), which means it \emph{does not create entanglement}.
% The benefit is that we can completely describe the state $\varphi$ using $d$ terms instead of considering a length $2^d$ vector, as would generally be required to analyze an $d$-qubit system.

Recall that a state $\varphi$ is a tuple of $d$ qubit values,
modeling the tensor product $q_1\otimes \cdots \otimes q_d$. 
The rules implicitly map each variable $x$ to a
range of qubits in the state, e.g., 
$\varphi(x)$ corresponds to some sub-state $q_k\otimes \cdots \otimes q_{k+n-1}$
where $\Omegasz(x)=n$.
%
Many of the rules in \Cref{fig:deno-sema} update a \emph{portion} of a
state. $\app{q_{(x,i)}}{\varphi}{(x,i)}$ updates the $i$-th
qubit of variable $x$ to be the (single-qubit) state $q_{(x,i)}$, and
$\app{q_{x}}{\varphi}{x}$ to update variable $x$ according to
the qubit \emph{tuple} $q_x$.
$\app{\uparrow q_{(x,i)}}{\varphi}{(x,i)}$ and $\app{\uparrow q_{x}}{\varphi}{x}$ are similar, except that they also accumulate the previous global phase of $\varphi(x,i)$ (or $\varphi(x)$).
$\downarrow$ is to convert a qubit $\alpha(b)\overline{q}$ to an unphased qubit $\overline{q}$.
%Thus, we have $\downarrow \alpha(b)\overline{q}=\overline{q}$ 
%and $\downarrow (q_1\otimes...\otimes q_n) = \downarrow q_1\otimes...\otimes \downarrow q_n$. 
%$\app{\uparrow q_{(x,i))}}{\varphi}{(x,i)}$ means to put back the global phase to the result qubit assigning to $(x,i)$. 
%%If $\varphi(x,i)=e^{2\pi i b}\overline{q}$ 
%and the result $q_{(x,i)}=\overline{q_{(x,i)}}$, 
%then we assign $e^{2\pi i b}\overline{q_{(x,i)}}$ to $(x,i)$;
%if the result $q_{(x,i)}=e^{2\pi i b_1}\overline{q_{(x,i)}}$, then we assign $e^{2\pi i (b+b_1)}\overline{q_{(x,i)}}$ to $(x,i)$. $\app{\uparrow q_{x}}{\varphi}{x}$ applies the above scenario to a list of qubits $q_k\otimes ... \otimes q_{k+n-1}$
%where $\Omegasz(x)=n$.

Function $\xsem$ updates the state of a single
qubit according to the rules for the standard quantum gate $X$.  
\texttt{cu} is a conditional operation
depending on the \texttt{Nor}-basis qubit $(x,i)$. 
\texttt{RZ} (or $\texttt{RZ}^{-1}$) is an z-axis phase rotation operation.
Since it applies to \texttt{Nor}-basis, it applies a global phase.
By \Cref{thm:sem-same}, when it is compiled to \sqir,
the global phase might be turned into a local one.
For example, to prepare the state $\sum_{j=0}^{2^n\sminus 1}(-i)^x\ket{x}$ \cite{ChildsNAND}, 
a series of Hadamard gates are applied, followed by several controlled-\texttt{RZ} gates on $x$,
where the controlled-\texttt{RZ} gates are definable by \oqasm.
\texttt{SR} (or
$\texttt{SR}^{-1}$) applies an $m+1$ series of \texttt{RZ} (or
$\texttt{RZ}^{-1}$) rotations where the $i$-th rotation
applies a phase of $\alpha({\frac{1}{2^{m-i+1}}})$
(or $\alpha({-\frac{1}{2^{m-i+1}}})$).
$\qsem$ applies an approximate quantum Fourier transform; $\ket{y}$ is an abbreviation of
$\ket{b_1}\otimes \cdots \otimes \ket{b_i}$ (assuming $\Omegasz(y)=i$) and $n$ is the degree of approximation.
If $n = i$, then the operation is the standard QFT\@.
Otherwise, each qubit in the state is mapped to $\qket{\frac{y}{2^{n-k}}}$, which is equal to $\frac{1}{\sqrt{2}}(\ket{0} + \alpha(\frac{y}{2^{n-k}})\ket{1})$ when $k < n$ and $\frac{1}{\sqrt{2}}(\ket{0} + \ket{1}) = \ket{+}$ when $n \leq k$ (since $\alpha(n) = 1$ for any natural number $n$).
$\qsem^{-1}$ is the inverse function of $\qsem$. 
Note that the input state to $\qsem^{-1}$ is guaranteed to have the form $\bigotimes_{k=0}^{i-1}(\qket{\frac{y}{2^{n-k}}})$ because it has type $\tphi{n}$.
$\psem_l$, $\psem_r$, and
$\psem_a$ are the semantics for \itext{Lshift}, 
\itext{Rshift}, and \itext{Rev}, respectively.   
% Several takeaways about \oqasm denotational semantics.
% For any operation application within the space domain $\hsp{S}^d$, the semantic application $U$ only affects the specific qubit ($\varphi_{(x,n)}$) / qubit array ($\varphi_{x}$) that it targets at, which does not create entanglement with other subsystems.
% This clear separation only works for the domain $\hsp{S}^d$.
% When we compile these operations to \sqir and see their effects on a general Hilbert space $\hsp{H}$, they might have entanglement effects.
% \yxp{Even if we turn it into unitary over the Hilbert space, it still does not generate entanglement with other subsystems.}
% \liyi{Can you have CNOT x y when x is Had and y is in Nor, then you will have entanglement. }
% However, the clear separation in $\hsp{S}^d$ provides a decompositional and analytical way of verifying and validating quantum oracles; thus, each sub-oracle component can be analyzed individually. The potential entanglements in a general Hilbert space become the naturally extended (additive) superposition effects.
% In addition, all semantic functions in Fig.~\ref{fig:deno-sema} are carefully engineered to only target qubits in a register $\varphi$ and do not target individual vectors in the vector space $\varphi$ represents.
% For example, $\xsem$ is defined for a basis phase space case $\ket{c}$, and we also define the case for superposition $\frac{1}{\sqrt{2}}(\ket{0}+(-1)^c\ket{1})$. We do not assume that the semantics of the basis phase space is automatically extended to dealing with individual elements in the superposition case.
% By using the semantics to prove quantum oracle properties, we only need to consider $O(n)$ qubits instead of the possible $2^n$ expanded vector elements.
% The semantics of a universal quantum assembly language like \sqir, by contrast, represents a quantum state as a unitary matrix whose size is \emph{exponential} in the number of vectors by expanding qubits to vectors in a register. \sqir's semantics also relies on using concrete qubits; a unitary matrix and virtual positions would inject a virtual-to-physical mapping into the semantic definition, which can severely complicate proofs~\cite{PQPC}. This leads to the successful correctness proof of the QFT-adder for the first time (Sec.~\ref{sec:op-verification}).
% We only define semantic functions for qubit forms when it is possible to apply them. For example, we do not define $\xsem$ for the form $\frac{1}{\sqrt{2}}(\ket{0}+e^{2\pi{i} b}\ket{1})$, because the \oqasm type system does not allow it. 

\subsection{OQASM Metatheory}\label{sec:metatheory}

\myparagraph{Soundness}
The following statement is proved: well-typed \oqasm programs are well-defined; i.e., the type system is sound concerning the semantics. 
Below is the well-formedness of an \oqasm state.

\begin{definition}[Well-formed \oqasm state]\label{appx:well-formed}\rm 
  A state $\varphi$ is \emph{well-formed}, written
  $\Sigma;\Omega \vdash \varphi$, iff:
\begin{itemize}
\item For every $x \in \Omega$ such that $\Omegaty(x) = \texttt{Nor}$,
  for every $k <\Omegasz(x)$, $\varphi(x,k)$ has the form
  $\alpha(r)\ket{b}$.

\item For every $x \in \Omega$ such that $\Omegaty(x) = \tphi{n}$ and $n \le \Omegasz(x)$,
  there exists a value $\upsilon$ such that for
  every $k < \Omegasz(x)$, $\varphi(x,k)$ has the form
  $\alpha(r)\qket{\frac{\upsilon}{ 2^{n- k}}}$.\footnote{Note that $\Phi(x) = \Phi(x + n)$, where the integer $n$ refers to phase $2 \pi n$; so multiple choices of $\upsilon$ are possible.}
\end{itemize}
\end{definition}

\noindent
Type soundness is stated as follows; the proof is by induction on $\instr$ and is mechanized in Roqc.

\begin{theorem}\label{thm:type-sound-oqasm}\rm[\oqasm type soundness]
If $\Sigma; \Omega \vdash \instr \triangleright \Omega'$ and $\Sigma;\Omega \vdash \varphi$ then there exists $\varphi'$ such that $\llbracket \instr \rrbracket\varphi=\varphi'$ and $\Sigma;\Omega' \vdash \varphi'$.
\end{theorem}

\myparagraph{Algebra}
Mathematically, the set of well-formed $d$-qubit \oqasm states for a given $\Omega$ can be interpreted as a subset $\hsp{S}^d$ of a $2^d$-dimensional Hilbert space $\hsp{H}^d$ \footnote{A \emph{Hilbert space} is a vector space with an inner product that is complete with respect to the norm defined by the inner product. $\hsp{S}^d$ is a sub\emph{set}, not a sub\emph{space} of $\hsp{H}^d$ because $\hsp{S}^d$ is not closed under addition: Adding two well-formed states can produce a state that is not well-formed.}. The semantics function $\llbracket \rrbracket$ can be interpreted as a $2^d \times 2^d$ unitary matrix, as is standard when representing the semantics of programs without measurement~\cite{PQPC}.
Because \oqasm's semantics can be viewed as a unitary matrix, correctness properties extend by linearity from $\hsp{S}^d$ to $\hsp{H}^d$---an oracle that performs addition for classical \texttt{Nor} inputs will also perform addition over a superposition of \texttt{Nor} inputs. The following statement is proved: $\hsp{S}^d$ is closed under well-typed \oqasm programs.

Given a qubit size map $\Sigma$ and type environment $\Omega$, the set of \oqasm programs that are well-typed concerning $\Sigma$ and $\Omega$ (i.e., $\Sigma;\Omega \vdash \instr \triangleright \Omega'$) form an algebraic structure $(\{\instr\},\Sigma, \Omega,\hsp{S}^d)$, where $\{\instr\}$ defines the set of valid program syntax, such that there exists $\Omega'$, $\Sigma;\Omega \vdash \instr \triangleright \Omega'$ for all $\instr$ in $\{\instr\}$; $\hsp{S}^d$ is the set of $d$-qubit states on which programs $\instr\in \{\instr\}$ are run, and are well-formed ($\Sigma;\Omega \vdash \varphi$) according to \Cref{appx:well-formed}.
From the \oqasm semantics and the type soundness theorem, for all $\instr \in \{\instr\}$ and $\varphi \in \hsp{S}^d$, such that $\Sigma;\Omega \vdash \instr \triangleright \Omega'$ and $\Sigma;\Omega \vdash \varphi$, and $\llbracket \instr \rrbracket\varphi=\varphi'$, $\Sigma;\Omega' \vdash \varphi'$, and $\varphi' \in \hsp{S}^d$. Thus, $(\{\instr\},\Sigma, \Omega,\hsp{S}^d)$, where $\{\instr\}$ defines a groupoid.

The groupoid can be certainly extended to another algebraic structure $(\{\instr'\},\Sigma,\hsp{H}^d)$, where $\hsp{H}^d$ is a general $2^d$ dimensional Hilbert space $\hsp{H}^d$ and $\{\instr'\}$ is a universal set of quantum gate operations.
Clearly, the following is true: $\hsp{S}^d \subseteq \hsp{H}^d$ and $\{\instr\} \subseteq \{\instr'\}$, because sets $\hsp{H}^d$ and $\{\instr'\}$ can be acquired by removing the well-formed ($\Sigma;\Omega \vdash \varphi$) and well-typed ($\Sigma;\Omega \vdash \instr \triangleright \Omega'$) definitions for $\hsp{S}^d$ and $\{\instr\}$, respectively.
$(\{\instr'\},\Sigma,\hsp{H}^d)$ is a groupoid because every \oqasm operation is valid in a traditional quantum language like \sqir. The following two theorems are to connect \oqasm operations with operations in the general Hilbert space: 

 \begin{theorem}\label{thm:subgroupoid}\rm
   $(\{\instr\},\Sigma, \Omega,\hsp{S}^d) \subseteq (\{\instr\},\Sigma,\hsp{H}^d)$ is a subgroupoid.
 \end{theorem}

\begin{theorem}\label{thm:sem-same}\rm
Let $\ket{y}$ be an abbreviation of $\bigotimes_{m=0}^{d-1} \alpha(r_m) \ket{b_m}$ for $b_m \in \{0,1\}$.
If for every $i\in [0,2^d)$, $\llbracket \instr \rrbracket\ket{y_i}=\ket{y'_i}$, then $\llbracket \instr \rrbracket (\sum_{i=0}^{2^d-1} \ket{y_i})=\sum_{i=0}^{2^d-1} \ket{y'_i}$.
\end{theorem}

The following theorems are proved as corollaries of the compilation correctness theorem from \oqasm to \sqir (\cite{oracleoopsla}). 
\Cref{thm:subgroupoid} suggests that the space $\hsp{S}^d$ is closed under the application of any well-typed \oqasm operation.
\Cref{thm:sem-same} says that \oqasm oracles can be safely applied to superpositions over classical states.\footnote{Note that a superposition over classical states can describe \emph{any} quantum state, including entangled states.}

\begin{figure}[t]
  {\scriptsize
    \begin{mathpar}
      \inferrule[ ]{}{\inot{(x,n)}\xrightarrow{\text{inv}} \inot{(x,n)}}
    
      \inferrule[  ]{}{\texttt{SR}\;m\;x\xrightarrow{\text{inv}} \texttt{SR}^{-1}\;m\;x}
  
      \inferrule[ ]{}{\iqft{n}{x} \xrightarrow{\text{inv}}  \iqft[-1]{n}{x}}   
  
      \inferrule[ ]{}{\texttt{Lshift}\;x\xrightarrow{\text{inv}} \texttt{Rshift}\;x} 
       
      \inferrule[ ]{\instr \xrightarrow{\text{inv}} \instr'}{\texttt{CU}\;(x,n)\;\instr \xrightarrow{\text{inv}} \texttt{CU}\;(x,n)\;\instr'} 
  
      \inferrule[ ]{\instr_1 \xrightarrow{\text{inv}} \instr'_1 \\ \instr_2 \xrightarrow{\text{inv}} \instr'_2}{\instr_1\;;\;\instr_2\xrightarrow{\text{inv}} \instr'_2\;;\;\instr'_1} 
      
    \end{mathpar}
  }
  \caption{Select \oqasm inversion rules}
  \label{fig:exp-reversed-fun}
\end{figure}

\begin{figure}[t]
{\tiny
\hspace*{-2em}
\centering
\begin{tabular}{c@{$\quad=\quad$}c@{\qquad}c@{$\quad=\quad$}c}
  \begin{minipage}{0.25\textwidth}
  \footnotesize
  \Qcircuit @C=0.25em @R=0.35em {
    & \qw & \multigate{3}{(x+a)_n} & \qw \\
    & \vdots & & \\
    & & & \\
    & \qw & \ghost{(x+a)_n} & \qw \\
    }
  \end{minipage}
&
\begin{minipage}{.45\textwidth}
% \includegraphics[width=1\textwidth]{qft-adder.png}
  \footnotesize
  \Qcircuit @C=0.35em @R=0.55em {
     & \qw & \gate{\texttt{SR}\;0} & \multigate{3}{\texttt{SR}\;1} & \qw & \qw & \qw & \multigate{5}{\texttt{SR}\;(n-1)} & \qw  \\
      & & & & & \dots & & &  \\
      & \qw & \qw  &  \ghost{\texttt{SR}\; 1} & \qw & \qw & \qw & \ghost{\texttt{SR}\;(n-1)} & \qw \\
      & & & & & & & &  \\
     & & & & & & & &  \\
   & \qw & \qw & \qw & \qw & \qw & \qw & \ghost{\texttt{SR}\;(n-1)}  & \qw 
    }
\end{minipage}
&  
\begin{minipage}{0.25\textwidth}
  \footnotesize
  \Qcircuit @C=0.25em @R=0.35em {
    & \qw & \multigate{3}{(x-a)_n} & \qw \\
    & \vdots & & \\
    & & & \\
    & \qw & \ghost{(x+a)_n} & \qw \\
    }
  \end{minipage}
&
\begin{minipage}{.45\textwidth}
% \includegraphics[width=1\textwidth]{qft-adder.png}
  \footnotesize
  \Qcircuit @C=0.35em @R=0.55em {
    & \qw & \multigate{5}{\texttt{SR}^{-1} (n-1)} & \qw & \qw & \qw & \multigate{3}{\texttt{SR}^{-1} 1} & \gate{\texttt{SR}^{-1} 0} & \qw \\
    &     &                                  &     & \dots &   &                              &                      &   \\
    & \qw & \ghost{\texttt{SR}^{-1} (n-1)}        & \qw & \qw   & \qw & \ghost{\texttt{SR}^{-1} 1} & \qw & \qw  \\
      & & & & & & & &  \\
     & & & & & & & &  \\
    & \qw & \ghost{\texttt{SR}^{-1} (n-1)} & \qw & \qw & \qw & \qw & \qw & \qw 
    }
\end{minipage}
\end{tabular}
}
\caption{Addition/subtraction circuits are inverses}
\label{fig:circuit-add-sub}
\end{figure}

\oqasm programs are easily invertible, as shown by the rules in \Cref{fig:exp-reversed-fun}.
This inversion operation is useful for constructing quantum oracles; for example, the core logic in the QFT-based subtraction circuit is just the inverse of the core logic in the addition circuit (\Cref{fig:exp-reversed-fun}).
This allows us to reuse the proof of addition in the proof of subtraction.
The inversion function satisfies the following properties:

 \begin{theorem}\label{thm:reversibility}\rm[Type reversibility]
    For any well-typed program $\instr$, such that $\Sigma; \Omega \vdash \instr \triangleright \Omega'$, its inverse $\instr'$, where $\instr \xrightarrow{\text{inv}} \instr'$, is also well-typed and $\Sigma;\Omega' \vdash \instr' \triangleright \Omega$. Moreover, $\llbracket \instr ; \instr' \rrbracket \varphi=\varphi$.
 \end{theorem}

\end{document}
%\vspace*{-0.5em}
\section{Background}
\label{sec:background}

Here, we provide background information on Quantum Computing. 

\noindent\textbf{\textit{Quantum Data}.} A quantum datum
%\footnote{Most literature describes quantum data as \emph{quantum states}. Here, we refer to them as quantum data to avoid confusion between program and quantum states.  } %
consists of one or more quantum bits (\emph{qubits}), which can be expressed as a two-dimensional vector $\begin{psmallmatrix} \alpha \\ \beta \end{psmallmatrix}$ where the \emph{amplitudes} $\alpha$ and $\beta$ are complex numbers and $|\alpha|^2 + |\beta|^2 = 1$.
%
We frequently write the qubit vector as $\alpha\aket{0}{1} + \beta\aket{1}{1}$ (the Dirac notation \cite{Dirac1939ANN}), where $\aket{0}{1} = \begin{psmallmatrix} 1 \\ 0 \end{psmallmatrix}$ and $\aket{1}{1} = \begin{psmallmatrix} 0 \\ 1 \end{psmallmatrix}$ are \emph{computational basis-vectors} and $\alpha\aket{0}{1}$ and $\beta\aket{1}{1}$ are basis-kets. The subscripts indicate the number of qubits for the basis-ket. When no necessity of mentioning qubit numbers, we denote the basis-ket as $\ket{0}$ or $\ket{1}$ by omitting them.
When both $\alpha$ and $\beta$ are non-zero, we can think of the qubit being ``both 0 and 1 at once,'' a.k.a. in a \emph{superposition} \cite{mike-and-ike}, e.g., $\frac{1}{\sqrt{2}}(\aket{0}{1} + \aket{1}{1})$ represents a superposition of $\aket{0}{1}$ and $\aket{1}{1}$.
Larger quantum data can be formed by composing smaller ones with the \emph{tensor product} ($\otimes$) from linear algebra, e.g., the two-qubit datum $\aket{0}{1} \otimes \aket{1}{1}$ (also written as $\aket{01}{2}$) corresponds to vector $[~0~1~0~0~]^T$.
% 
However, many multi-qubit data cannot be \emph{separated} and expressed as the tensor product of smaller data; such inseparable datum states are called \emph{entangled}, e.g.\/ $\frac{1}{\sqrt{2}}(\aket{00}{2} + \aket{11}{2})$, known as a \emph{Bell pair}, which can be rewritten to $\Msum_{d=0}^{1}{\frac{1}{\sqrt{2}}}{\aket{dd}{2}}$, where $dd$ is a bit string consisting of two bits, both being the same (i.e., $d=0$ or $d=1$). Each term $\frac{1}{\sqrt{2}}\aket{dd}{2}$ is named a \emph{basis-ket}, consisting an amplitude $\frac{1}{\sqrt{2}}$ and a basis vector $\aket{dd}{2}$.

%
%It is computationally hard to determine if an arbitrary quantum value is
%separable, therefore we say a general quantum value is \emph{possibly
%  entangled}.

%   
% An $n$-qubit quantum value state is typically represented as a $2^n$ dimensional
% vector. Alternatively, the values can be represented in different forms. For
% example, an initialized qubit typically has a value $\ket{0}$ or $\ket{1}$,
% called a \textit{normal typed value} ($\tnort$) in \qafny. A collection of $n$
% qubits that are in superposition but not entangled, i.e.,
% $\frac{1}{\sqrt{2}}(\ket{0} + \alpha(r_0)\ket{1})\otimes ... \otimes
% \frac{1}{\sqrt{2}}(\ket{0} + \alpha(r_{n-1})\ket{1})$, can be encoded as
% $\shad{2^n}{n\sminus 1}{\alpha(r_j)}$, where $\alpha(r_j)=e^{2\pi i r_j}$ and
% $r_j \in \mathbb{R}$, which is called a \textit{Hadamard typed value} ($\thadt$)
% in \qafny.

%Value $\alpha(r_j)$ is the \emph{local phase} of the state, which is a unique quantum amplitude whose norm is $1$, i.e., $\slen{\alpha(r_j)}=1$. 
%In the state $\frac{1}{\sqrt{2}}(\ket{0} + \ket{1})$, we can view the local phase $1$ as $e^{0}$, and $\frac{1}{\sqrt{2^n}}e^{0}$ is the amplitude for every basis-ket.
%This is not a standard form for all unentangled multi-qubit value states but rather a convenient way of representing a particular class of states common in many quantum algorithms, which can be utilized for proof automation.

% The most general representation for $n$-qubit values is as a linear combination of basis-kets \cite{mike-and-ike}, or in
% Dirac notation in \cite{Dirac1939ANN}, as $\sch{m}{z_j}{\ket{c_j}}$, where $z_j\in \mathbb{C}$ is an amplitude, $c_j$ is an $n$-length bitstring called a \emph{basis} (computational basis only), and $m < 2^n$. The above notation is the same as ${z_0}\ket{c_0}+...+{z_{m}}\ket{c_{m}}$; each $j$-th element ${z_j}\ket{c_j}$ represents a \emph{basis-ket} in the superposition value state. 
% This is called an \textit{entanglement typed value} ($\tcht$) in \qafny.
% For example, the bell pair can be represented as $\sch{1}{\frac{1}{\sqrt{2}}}{\ket{c_j}}$ with $c_0=00$ and $c_1=11$.
% Notice that the basis-kets' bases ($c_j$) are all distinct in a value state. 
% \qafny identifies these three different representations and uses a type system to transform quantum-state representations properly.

%%Notice that the amplitudes satisfy the relation $\sum_{0}^{m}\slen{z_j}^2 = 1$. However, in some intermediate program evaluation in QNP, we lose the restriction to be $\sum_{0}^{m}\slen{z_j}^2 \le 1$, because a state $\sch{m}{z_j}{c_j}$ can be split into two parts as $\sch{m}{z_j}{c_j}=\schii{m_1}{z_i}{c_i}+\schk{m_2}{z_k}{c_k}$, and we might only want to reason about a portion of the state $\schii{m_1}{z_i}{c_i}$ locally so that $\sum_{0}^{m_1}\slen{z_i}^2 < 1$. 
%
%%In QNP, each quantum state is associated with a \emph{session}, referring to a cluster of quantum array pieces possibly entangled. We can view a session as a virtual quantum array that manages quantum physical qubit array pieces living in different locations but is locally connected through entanglement. See \Cref{sec:quantum-state}.

          \begin{wrapfigure}{r}{3cm}
          %  \vspace*{-0.2em}
            {\qquad
              \footnotesize
              \Qcircuit @C=0.5em @R=0.5em {
                \lstick{\ket{0}} & \gate{H} & \ctrl{1} & \qw &\qw & & \dots & \\
                \lstick{\ket{0}} & \qw & \targ & \ctrl{1} & \qw & &  \dots &  \\
                \lstick{\ket{0}} & \qw & \qw   & \targ & \qw & &  \dots &  \\
                & \vdots &   &  &  & & & \\
                & \vdots &  & \dots & & & \ctrl{1} & \qw  \\
                \lstick{\ket{0}} & \qw & \qw & \qw &\qw &\qw & \targ & \qw
              }
            }
            \caption{GHZ Circuit}
            \label{fig:background-circuit-examplea}
          \end{wrapfigure}
          
\noindent\textbf{\textit{Quantum Computation and Measurement.}} 
%Quantum programming languages are essentially hybrid, containing both quantum and classical components, so that they can collaboratively finish a task (the \emph{QRAM model}~\cite{Knill1996}).
%
Computation on a quantum datum consists of a series of \emph{quantum operations}, each acting on a subset of qubits in the quantum datum. In the standard form, quantum computations are expressed as \emph{circuits}, as in \Cref{fig:background-circuit-examplea}, which depicts a circuit that prepares the Greenberger-Horne-Zeilinger (GHZ) state \cite{Greenberger1989} --- an $n$-qubit entangled datum of the form: $\ket{\text{GHZ}^n} = \frac{1}{\sqrt{2}}(\bigotimes^n \ket{0}+\bigotimes^n\ket{1})$.
In these circuits, each horizontal wire represents a qubit, and boxes on these wires indicate quantum operations, or \emph{gates}. The circuit in \Cref{fig:background-circuit-examplea} uses $n$ qubits and applies $n$ gates: a \emph{Hadamard} (\coqe{H}) gate and $n-1$ \emph{controlled-not} (\coqe{CNOT}) gates. Applying a gate to a quantum datum \emph{evolves} it.
% 
Its traditional semantics is expressed by multiplying the datum's vector form by the gate's corresponding matrix representation: $n$-qubit gates are $2^n$-by-$2^n$ matrices.
% 
Except for measurement gates, a gate's matrix must be \emph{unitary} and thus preserve appropriate invariants of quantum data's amplitudes. 
%
A \emph{measurement} operation extracts classical information from a quantum datum. It collapses the datum to a basis-ket with a probability related to the datum's amplitudes (\emph{measurement probability}), e.g., measuring $\frac{1}{\sqrt{2}}(\aket{0}{1} + \aket{1}{1})$ collapses the datum to $\aket{0}{1}$ or $\aket{1}{1}$, each with probability $\frac{1}{2}$. The ket values correspond to classical values $0$ or $1$, respectively. 
A more general form of quantum measurement is \emph{partial measurement}, which measures a subset of qubits in a qubit array;
% 
such operations often have simultaneity effects due to entanglement, \ie{} in a Bell pair $\frac{1}{\sqrt{2}}(\aket{00}{2} + \aket{11}{2})$, measuring one qubit guarantees the same outcome for the other --- if the first bit is measured as $0$, the second bit will be measured as $0$.

\noindent\textbf{\textit{Quantum Oracles.}} Quantum algorithms manipulate input information encoded in ``oracles'', which are callable black-box circuits. 
%For example, Grover's algorithm for unstructured quantum search \cite{grover1996,grover1997} is a general approach for searching a quantum ``database,'' which is encoded in an oracle for a function $f : \{0, 1\}^n \to \{0, 1\}$. Grover's algorithm finds an element $x \in \{0, 1\}^n$ such that $f(x) = 1$ using $O(2^{n/2})$ queries, a quadratic speedup over the best possible classical algorithm, which requires $\Omega(2^n)$ queries. 
Quantum oracles are usually quantum-reversible implementations of classical operations, especially arithmetic operations. Their behavior is defined in terms of transitions between single basis-kets.
We can infer the global state behavior based on the single basis-ket behavior through the quantum summation formula below. This resembles an array map operation in \Cref{fig:intros2}.
\oqasm in VQO~\cite{oracleoopsla} is a language that permits the definitions of quantum oracles with efficient verification and testing facilities using the following summation formula:
%The main approach in VQO is to utilize the summation formula to disallow the appearance of quantum entanglement state in analyzing programs,
%whereas \pqasm disregards if a state is entangled or a simple superposition without entanglement and utilizes the summation formula to reduce a quantum superposition state to a singleton basis-ket state to perform analysis.

{\footnotesize
  \begin{mathpar}
 \inferrule[]{ \forall j\,.\, x_j \longrightarrow f(x_j)  }{ \Sigma_j \alpha_j \ket{x_j} \longrightarrow \Sigma_j \alpha_j f(\ket{x_j})}
 % \inferrule[]{ \forall j\,.\, (x_j,y) \longrightarrow f(x_j,y) }{ \Sigma_j \alpha_j \ket{x_j}\ket{y} \longrightarrow \Sigma_j \alpha_j f(\ket{x_j}\ket{y})}
    \end{mathpar}
}

\noindent\textbf{\textit{Repeat-Until-Success Quantum Programs.}} A repeat-until-success program utilizes the probabilistic feature of partial measurement operations. We first set up a one-step repeat-until-success by linking the desired quantum state with the success measurement of a certain classical value. If such a value is observed after measurement, we know that the desired state is successfully prepared; otherwise, we repeat the one-step procedure.
One example of a one-step repeat-until-success procedure is in \Cref{fig:intros-example} to repeat the $n$ basis-ket superposition state.
If we measure out $v=1$, the desired state $\varphi$ is prepared; otherwise, we repeat the procedure.

\noindent\textbf{\textit{No Cloning.}} 
The \emph{no-cloning theorem} indicates no general way of exactly copying a quantum datum. In quantum circuits, this is related to ensuring the reversible property of unitary gate applications.
% 
For example, the controlled node and controlled body of a quantum control gate cannot refer to the same qubits, e.g., $\ictrl{q}{\iota}$ violates the property if $q$ is mentioned in $\iota$.
% 
\pqasm{} enforces no cloning through our type system.
 
\section{OQASM: An Assembly Language for Quantum Oracles}
\label{sec:vqir}

The \oqasm expression $\mu$ used in \Cref{fig:pqasm} places an additional wrapper on top of the \oqasm expression $\iota$ given in \Cref{fig:vqir}. Here, we first provide a step-by-step explanation of \oqasm.

\oqasm is designed to express efficient quantum
oracles that can be easily tested and, if desired, proved
correct.
\oqasm operations leverage both the standard
computational basis and an alternative basis connected by the quantum
Fourier transform (QFT). 
\oqasm's type system tracks the bases of variables in
\oqasm programs, forbidding operations that would introduce
entanglement. \oqasm states are therefore efficiently
represented, so programs can be effectively tested and are simpler to
verify and analyze. In addition, \oqasm uses \emph{virtual qubits}
to support \emph{position shifting operations}, which support
arithmetic operations without introducing extra gates during
translation. All of these features are novel to quantum assembly
languages. 

This section presents \oqasm states and the language's syntax,
semantics, typing, and soundness results. As a running example, the QFT
adder~\cite{qft-adder} is shown in \Cref{fig:circuit-exampleb}. The Roqc
function \coqe{rz_adder} generates an \oqasm program that adds two
natural numbers \coqe{a} and \coqe{b}, each of length \coqe{n} qubits.

\begin{figure*}[t]
  \centering
  \begin{tabular}{c @{\qquad} c}

  \begin{minipage}[b]{.6\textwidth}
  % \includegraphics[width=1\textwidth]{qft-adder.png}
  {\scriptsize
    \Qcircuit @C=0.5em @R=0.75em {
      \lstick{\ket{a_{n-1}}} & \qw & \ctrl{5} & \qw & \qw & \qw & \qw & \qw & \qw & \qw & \rstick{\ket{a_{n-1}}} \\
      \lstick{\ket{a_{n-2}}} & \qw & \qw & \ctrl{4} & \qw & \qw & \qw & \qw & \qw & \qw & \rstick{\ket{a_{n-2}}}\\
      \lstick{\vdots} & & & & & & & & & & \rstick{\vdots} \\
      \lstick{} & & & & & & & & & & \\
      \lstick{\ket{a_0}} & \qw & \qw & \qw & \qw & \qw & \qw & \ctrl{1} & \qw & \qw & \rstick{\ket{a_0}} \\
      \lstick{\ket{b_{n-1}}} & \multigate{5}{\texttt{QFT}} & \gate{\texttt{SR 0}} & \multigate{3}{\texttt{SR 1}} & \qw & \qw & \qw & \multigate{5}{\texttt{SR (n-1)}} & \multigate{5}{\texttt{QFT}^{-1}} & \qw & \rstick{\ket{a_{n-1} + b_{n-1}}} \\
      \lstick{} & & & & & \dots & & & & \\
      \lstick{\ket{b_{n-2}}} & \ghost{\texttt{QFT}} & \qw  &  \ghost{\texttt{SR 1}} & \qw & \qw & \qw & \ghost{\texttt{SR (n-1)}} & \ghost{\texttt{QFT}^{-1}} & \qw & \rstick{\ket{a_{n-2} + b_{n-2}}} \\
      \lstick{\vdots} & & & & & & & & & & \rstick{\vdots} \\
      \lstick{} & & & & & & & & & & \\
      \lstick{\ket{b_0}} & \ghost{\texttt{QFT}} & \qw & \qw & \qw & \qw & \qw & \ghost{\texttt{SR (n-1)}} & \ghost{\texttt{QFT}^{-1}}  & \qw & \rstick{\ket{a_0 + b_0}} 
      }
      }
  \subcaption{Quantum circuit}
  \end{minipage}
  \hfill\hfill
  \begin{minipage}[b]{.38\textwidth}
  \begin{coq}
  Fixpoint rz_adder' (a b:var) (n:nat) 
    := match n with 
       | 0 => ID (a,0)
       | S m => CU (a,m) (SR m b); 
                rz_adder' a b m
       end.
  Definition rz_adder (a b:var) (n:nat) 
    := Rev a ; Rev b ; $\texttt{QFT}$ b ;
       rz_adder' a b n;
       $\texttt{QFT}^{-1}$ b; Rev b ; Rev a.
  \end{coq}
  \subcaption{\oqasm metaprogram (in Roqc)}
  \end{minipage}
  \end{tabular}
  %\vspace{-0.5em}
  \caption{Example \oqasm program: QFT-based adder}
  \label{fig:circuit-exampleb}
  \end{figure*}

\subsection{OQASM States} \label{sec:pqasm-states}

\begin{figure}[t]
  \small
  \[\hspace*{-0.5em}
\begin{array}{l>{$} p{1.2cm} <{$} c l}
      \text{Bit} & b & ::= & 0 \mid 1 \\
      \text{Natural number} & n & \in & \mathbb{N} \\
      \text{Real} & r & \in & \mathbb{R}\\
      \text{Phase} & \alpha(r) & ::= & e^{2\pi i r} \\
      \text{Basis} & \tau & ::= & \texttt{Nor} \mid \texttt{Phi}\;n \\
      \text{Unphased qubit} & \overline{q} & ::= & \ket{b} ~~\mid~~ \qket{r} \\
      \text{Qubit} & q & ::= &\alpha(r) \overline{q}\\
      \text{State (length $d$)} & \varphi & ::= & q_1 \otimes q_2 \otimes \cdots \otimes q_d
    \end{array}
  \]
  \caption{\oqasm state syntax}
  \label{fig:vqir-state}
\end{figure}

An \oqasm program state is represented according to the grammar in
\Cref{fig:vqir-state}. A state $\varphi$ of $d$ qubits is 
a length-$d$ tuple of qubit values $q$; the state models the tensor
product of those values. This means that the size of $\varphi$ is
$O(d)$ where $d$ is the number of qubits. A $d$-qubit state in a
language like \sqir is represented as a length $2^d$ vector of complex
numbers, $O(2^d)$ in the number of qubits. Our linear state
representation is possible because applying for any well-typed \oqasm
program on any well-formed \oqasm state never causes qubits to be
entangled.

A qubit value $q$ has one of two forms $\overline{q}$, scaled by a
global phase $\alpha(r)$. The two forms depend on the \emph{basis} $\tau$ that the qubit is in---it could be either \texttt{Nor} or \texttt{Phi}. A \texttt{Nor} qubit has form
$\ket{b}$ (where $b \in \{ 0, 1 \}$), which is a
computational basis value. 
A \texttt{Phi} qubit has the form $\qket{r} = \frac{1}{\sqrt{2}}(\ket{0}+\alpha(r)\ket{1})$, which is a value of the (A)QFT basis.
The number $n$ in \texttt{Phi}$\;n$ indicates the precision of the state $\varphi$.
As shown by~Beauregard \cite{qft-adder}, arithmetic on the computational basis can sometimes be more efficiently carried out on the QFT basis, which leads to the use of quantum operations (like QFT) when implementing circuits with classical input/output behavior.

\subsection{OQASM Syntax, Typing, and Semantics}\label{sec:oqasm-syn}

\begin{figure}[t]
\begin{minipage}[t]{0.5\textwidth}
{\small \centering

  $ \hspace*{-0.8em}
\begin{array}{llcl}
      \text{Position} & p & ::= & (x,n) \qquad   \text{Nat. Num}~n
                                  \qquad   \text{Variable}~x\\
      \text{Instruction} & \instr & ::= & \iskip{p} \mid \inot{p}
                                          \mid \irz[\lbrack -1 \rbrack]{n}{p} \mid \iseq{\instr}{\instr}\\
                & & \mid &  \isr[\lbrack -1 \rbrack]{n}{x} \mid \iqft[\lbrack -1 \rbrack]{n}{x} \mid \ictrl{p}{\instr}  \\
                      & & \mid & \ilshift{x} \mid \irshift{x} \mid \irev{x} 
    \end{array}
  $
}
  \caption{\oqasm syntax. For an operator \texttt{OP}, $\texttt{OP}^{\lbrack -1 \rbrack}$ indicates that the operator has a built-in inverse available.}
  \label{fig:vqir}
\end{minipage}
\hfill
\begin{minipage}[t]{0.38\textwidth}
{\scriptsize
\centering
\begin{tabular}{c@{$\quad=\quad$}c}
  \begin{minipage}{0.3\textwidth}

%\includegraphics[width=0.3\textwidth]{sr-meaning.png}
  \Qcircuit @C=0.5em @R=0.5em {
    \lstick{} & \qw     & \multigate{4}{\texttt{SR m}} & \qw & \qw \\
    \lstick{} & \qw     & \ghost{\texttt{SR m}}           & \qw & \qw \\
    \lstick{} & \vdots & & \vdots & \\
    \lstick{} & & & & \\
    \lstick{} & \qw     & \ghost{\texttt{SR m}}           & \qw  & \qw
    }
  \end{minipage} & 
  \begin{minipage}{0.3\textwidth}
  \small
  \Qcircuit @C=0.5em @R=0.5em {
    \lstick{} & \qw     & \gate{\texttt{RZ (m+1)}} & \qw & \qw \\
    \lstick{} & \qw     & \gate{\texttt{RZ m}}          & \qw & \qw \\
    \lstick{} & & \vdots & & \\
    \lstick{} & & & & \\
    \lstick{} & & & & \\
    \lstick{} & \qw     & \gate{\texttt{RZ 1}}           & \qw  & \qw
    }
 
  \end{minipage} 
\end{tabular}
}
\caption{\texttt{SR} unfolds to a series of \texttt{RZ} instructions}
\label{fig:sr-meaning}
\end{minipage}
\end{figure}

\Cref{fig:vqir} presents \oqasm's syntax. An \oqasm program consists of
a sequence of instructions $\instr$. Each instruction applies an
operator to either a variable $x$, representing a group of qubits, or a \emph{position} $p$, which identifies a particular offset into a variable $x$. 

The instructions in the first row correspond to simple single-qubit
quantum gates---$\iskip{p}$, $\inot{p}$, and $\irz[\lbrack -1 \rbrack]{n}{p}$
 ---and instruction sequencing.
The instructions in the next row apply to whole variables: $\iqft{n}{x}$
applies the AQFT to variable $x$ with $n$-bit precision and
$\iqft[-1]{n}{x}$ applies its inverse.
If $n$ equals the size of $x$, then the AQFT operation is exact.
$\isr[\lbrack -1 \rbrack]{n}{x}$
applies a series of \texttt{RZ} gates (\Cref{fig:sr-meaning}). 
Operation $\ictrl{p}{\instr}$
applies instruction $\instr$ \emph{controlled} on qubit position
$p$. All of the operations in this row---\texttt{SR}, \texttt{QFT}, and \texttt{CU}---will be translated to multiple \sqir
gates. The function \coqe{rz_adder} in \Cref{fig:circuit-exampleb}(b) uses
many of these instructions; e.g., it uses \texttt{QFT} and \texttt{QFT}$^{-1}$ and applies
\texttt{CU} to the $m$th position of variable \texttt{a} to control
instruction \texttt{SR m b}.

In the last row of \Cref{fig:vqir}, instructions $\ilshift{x}$,
$\irshift{x}$, and $\irev{x}$ are \emph{position shifting operations}.
Assuming that $x$ has $d$ qubits and $x_k$ represents the $k$-th qubit
state in $x$, $\texttt{Lshift}\;x$ changes the $k$-th qubit state to
$x_{(k + 1)\mmod d}$, $\texttt{Rshift}\;x$ changes it to
$x_{(k + d - 1)\mmod d}$, and \texttt{Rev} changes it to $x_{d-1-k}$. In
our implementation, shifting is \emph{virtual}, not physical. The \oqasm
translator maintains a logical map of variables/positions to concrete
qubits and ensures that shifting operations are no-ops, introducing no extra gates.

Other quantum operations could be added to \oqasm to
allow reasoning about a larger class of quantum programs while still
guaranteeing a lack of entanglement. 

\begin{figure}[t]
\begin{minipage}[b]{0.57\textwidth}
{\scriptsize
  \begin{mathpar}
    \inferrule[X]{\Omegaty(x)=\texttt{Nor} \\ n < \Omegasz(x)}{\Sigma;\Omega \vdash \inot{(x,n)}\triangleright \Omega}
  
    \inferrule[RZ]{\Omegaty(x)=\texttt{Nor} \\ n < \Omegasz(x)}{\Sigma;\Omega \vdash \irz{q}{(x,n)} \triangleright \Omega}

    \inferrule[SR]{\Omegaty(x)=\tphi{n} \\ m < n}{\Sigma;\Omega \vdash \texttt{SR}\;m\;x\triangleright \Omega}   

    \inferrule[QFT]{\Omegaty(x)=\texttt{Nor}\\n \le \Omegasz(x)}{\Sigma; \Omega \vdash \iqft{n}{x}\triangleright \Omega[x\mapsto \tphi{n}]}    
     
    \inferrule[RQFT]{\Omegaty(x)=\tphi{n}\\n \le \Omegasz(x)}{\Sigma; \Omega \vdash \iqft[-1]{n}{x}\triangleright \Omega[x\mapsto \texttt{Nor}]}             
    
    \inferrule[CU]{\Omegaty(x)=\texttt{Nor} \\ \texttt{fresh}~(x,n)~\instr \\\\ \Sigma; \Omega\vdash \instr\triangleright \Omega \\ \texttt{neutral}(\instr)}{\Sigma; \Omega \vdash \texttt{CU}\;(x,n)\;\instr \triangleright \Omega} 
     
    \inferrule[LSH]{\Omegaty(x)=\texttt{Nor}}{\Sigma; \Omega \vdash \texttt{Lshift}\;x\triangleright \Omega}

     \inferrule[SEQ]{\Sigma; \Omega\vdash \instr_1\triangleright \Omega' \\ \Sigma; \Omega'\vdash \instr_2\triangleright \Omega''}{\Sigma; \Omega \vdash \instr_1\;;\;\instr_2\triangleright \Omega''} 
    
  \end{mathpar}
}
  \caption{Select \oqasm typing rules}
  \label{fig:exp-well-typeda}
\end{minipage}
\hfill
\hfill
\begin{minipage}[b]{0.4\textwidth}
{\footnotesize
\begin{center}\hspace*{-1em}
\begin{tikzpicture}[->,>=stealth',shorten >=1pt,auto,node distance=3.2cm,
                    semithick]
  \tikzstyle{every state}=[fill=black,draw=none,text=white]

  \node[state] (A)              {$\texttt{Nor}$};
  \node[state]         (C) [left of=A] {$\tphi{n}$};

  \path (A) edge [loop above]            node {$\Big\{\begin{array}{l}\texttt{ID},~\texttt{X},~\texttt{RZ}^{\lbrack -1 \rbrack},~\texttt{CU},\\
              \texttt{Rev},\texttt{Lshift},\texttt{Rshift}\end{array}\Big\}$} (A)
            edge   node [above] {\{$\texttt{QFT}\;n$\}} (C);
  \path (C) edge [loop above]            node {$\{\texttt{ID},~\texttt{SR}^{\lbrack -1 \rbrack}\}$} (C)
            edge  [bend right]             node {$\{\texttt{QFT}^{-1}\;n\}$} (A);
\end{tikzpicture}
\end{center}
}
\caption{Type rules' state machine}
\label{fig:state-machine}
\end{minipage}
\end{figure}

\myparagraph{Typing}
\label{sec:vqir-typing}

In \oqasm, typing is concerning a \emph{type environment}
$\Omega$ and a predefined \emph{size
  environment} $\Sigma$, which map \oqasm
variables to their basis and size (number of qubits), respectively.
The typing judgment is written $\Sigma; \Omega\vdash \instr \triangleright \Omega'$ which
states that $\instr$ is well-typed under $\Omega$ and $\Sigma$, and
transforms the variables' bases to be as in $\Omega'$ ($\Sigma$ is unchanged). 
$\Sigma$ is fixed because the number of qubits in execution is always fixed.
It is generated in the high-level language compiler, such as \sourcelang in \cite{oracleoopsla}.
The algorithm generates $\Sigma$ by taking an \sourcelang program and scanning through
all the variable initialization statements.
Select type rules are given in \Cref{fig:exp-well-typeda}; 
the rules not shown (for \texttt{ID}, \texttt{Rshift}, \texttt{Rev}, \texttt{RZ}$^{-1}$, and \texttt{SR}$^{-1}$) are similar.

The type system enforces three invariants. First, it enforces that
instructions are well-formed, meaning that gates are applied to valid
qubit positions (the second premise in \rulelab{X}) and that any control qubit is distinct from the
target(s) (the \texttt{fresh} premise in
\rulelab{CU}).  This latter property enforces the quantum
\emph{no-cloning rule}.
For example, applying the \texttt{CU} in \code{rz\_adder'} (\Cref{fig:circuit-exampleb}) is valid
because position \code{a,m} is distinct from variable \code{b}.

Second, the type system enforces that instructions leave affected
qubits on a proper basis (thereby avoiding entanglement). The
rules implement the state machine shown in
\Cref{fig:state-machine}. For example, $\texttt{QFT}\;n$ transforms a variable from \texttt{Nor} to
$\tphi{n}$ (rule \rulelab{QFT}), while $\texttt{QFT}^{-1}\;n$
transforms it from $\tphi{n}$ back to \texttt{Nor} (rule
\rulelab{RQFT}). Position shifting operations 
are disallowed on variables $x$ in
the \texttt{Phi} basis because the qubits that makeup $x$ are
internally related (see \Cref{appx:well-formed}) and cannot be rearranged. Indeed, applying a
\texttt{Lshift} and then a $\texttt{QFT}^{-1}$ on $x$ in \texttt{Phi}
would entangle $x$'s qubits.

% \begin{figure}[t]
% {\footnotesize
% \begin{center}
% \begin{tikzpicture}[->,>=stealth',shorten >=1pt,auto,node distance=3.2cm,
%                     semithick]
%   \tikzstyle{every state}=[fill=white,draw=black,text=black]
% 
%   \node[initial,accepting,state] (A)              {$\texttt{OK}$};
%   \node[state]         (B) [right of=A] {$ $};
% 
%   \path (A) edge [loop above]            node {$b,\epsilon / \epsilon$} (A)
%             edge  [above] node {$a,\emptyset / a$} (B);
%   \path (B) edge [loop right]            node [right] {$\begin{array}{l}b,\epsilon / \epsilon\\
%                                                                 a,a' / a a'\\
%                                                                 a,\overline{a} / \epsilon\\
%                                                  \end{array}$} (B)
%             edge  [bend left]             node [above] {$\epsilon,\emptyset / \emptyset$} (A);
% \end{tikzpicture}
% \end{center}
% }
% {
% \footnotesize
% $
% \begin{array}{l}
% a,a'\in \{\ilshift{x},\irshift{x},\irev{x} \} \wedge a' \neq \overline{a}
% \\
% \overline{\ilshift{x}}=\irshift{x}
% \quad
% \overline{\irshift{x}}=\ilshift{x}
% \quad
% \overline{\irev{x}}=\irev{x}
% \\
% b\not\in\{\ilshift{x},\irshift{x},\irev{x}, \instr;\instr \}
% \\
% \emptyset=\text{ no element in stack}
% \end{array}
% $
% }
% 
% \caption{Pushdown automata for \texttt{neutral}}
% \label{fig:pushdown-neu}
% \end{figure}

Third, the type system enforces that the effect of position-shifting
operations can be statically tracked. The \texttt{neutral} condition of
\rulelab{CU} requires that any shifting within $\instr$ is restored by the time it
completes. 
For example, $\sseq{\ictrl{p}{(\ilshift{x})}}{\inot{(x,0)}}$ is not well-typed because knowing the final physical position of qubit $(x,0)$ would require statically knowing the value of $p$. 
On the other hand, the program $\sseq{\ictrl{c}{(\sseq{\ilshift{x}}{\sseq{\inot{(x,0)}}{\irshift{x}}})}}{\inot{(x,0)}}$ is well-typed 
because the effect of the \texttt{Lshift} is ``undone'' by an \texttt{Rshift} inside the body of the \texttt{CU}.

% \texttt{neutral}'s definition in \Cref{fig:pushdown-neu}
% views $\instr$ as a string concatenated
% by the sequence operation ($;$) and requires $\instr$ to be
% accepted according to a family of pushdown automatas $\{G\}_{x}$ for every $x$ presented in $\instr$. 
% A program $\instr$ is \texttt{neutral}, iff, $\instr$ as a string is
% accepted by all the automatas in $\{G\}_{x}$.

\myparagraph{Semantics}\label{sec:pqasm-dsem}

\begin{figure}[t]
{\scriptsize
\[
\begin{array}{lll}
\llbracket \iskip{p} \rrbracket\varphi &= \varphi\\[0.2em]

\llbracket \inot{(x, i)} \rrbracket\varphi &= \app{\uparrow\xsem(\downarrow\varphi(x,i))}{\varphi}{(x,i)}
& \texttt{where  }\xsem(\ket{0})=\ket{1} \qquad\, \xsem(\ket{1})=\ket{0}
\\[0.5em]

\llbracket \ictrl{(x,i)}{\instr} \rrbracket\varphi &=  \csem(\downarrow\varphi(x,i),\instr,\varphi)
&
\texttt{where  }
\csem({\ket{0}},{\instr},\varphi)=\varphi\quad\;\,
\csem({\ket{1}},{\instr},\varphi)=\llbracket \instr \rrbracket\varphi
\\[0.4em]

\llbracket \irz{m}{(x,i)} \rrbracket\varphi &= \app{\uparrow {\rsem}({m},\downarrow\varphi(x,i))}{\varphi}{(x,i)}
&\texttt{where  }{\rsem}(m,\ket{0})=\ket{0} \; \quad{\rsem}(m,\ket{1})=\alpha(\frac{1}{2^m})\ket{1}
\\[0.5em]

\llbracket \irz[-1]{m}{(x,i)} \rrbracket\varphi &= \app{\uparrow {\rrsem}({m},\downarrow\varphi(x,i))}{\varphi}{(x,i)}
 &\texttt{where  }{\rrsem}(m,\ket{0})=\ket{0}
\quad{\rrsem}(m,\ket{1})=\alpha(-\frac{1}{2^m})\ket{1}
\\[0.5em]

\llbracket \isr{m}{x} \rrbracket\varphi &
                                            \multicolumn{2}{l}{= \app{\uparrow \qket{r_i+\frac{1}{2^{m-i+1}}}}{\varphi}{\forall i \le m.\;(x,i)}
\qquad \texttt{when  }
\downarrow\varphi(x,i) = \qket{r_i}}\\[0.5em]

\llbracket \isr[-1]{m}{x} \rrbracket\varphi&\multicolumn{2}{l}{= \app{\uparrow \qket{r_i-\frac{1}{2^{m-i+1}}}}{\varphi}{\forall i \le m.\;(x,i)}
\qquad \texttt{when  }
\downarrow\varphi(x,i) = \qket{r_i}}\\[0.5em]

\llbracket \iqft{n}{x} \rrbracket\varphi &= \app{\uparrow\qsem(\Sigma(x),\downarrow\varphi(x),n)}{\varphi}{x}
& \texttt{where  }\qsem(i,\ket{y},n)=\bigotimes_{k=0}^{i-1}(\qket{\frac{y}{2^{n-k}}})
\\[0.5em]

\llbracket \iqft[-1]{n}{x} \rrbracket\varphi &=  \app{\uparrow\qsem^{-1}(\Sigma(x),\downarrow\varphi(x),n)}{\varphi}{x}
\\[0.5em]

\llbracket \ilshift{x} \rrbracket\varphi &= \app{{\psem}_{l}(\varphi(x))}{\varphi}{x}
&
\texttt{where  }{\psem}_{l}(q_0\otimes q_1\otimes \cdots \otimes q_{n-1})=q_{n-1}\otimes q_0\otimes q_1 \otimes \cdots
\\[0.5em]

\llbracket \irshift{x} \rrbracket\varphi &= \app{{\psem}_{r}(\varphi(x))}{\varphi}{x}
&
\texttt{where  }{\psem}_{r}(q_0\otimes q_1\otimes \cdots \otimes q_{n-1})=q_1\otimes \cdots \otimes q_{n-1} \otimes q_0
\\[0.5em]

\llbracket \irev{x} \rrbracket\varphi &= \app{{\psem}_{a}(\varphi(x))}{\varphi}{x}
&
\texttt{where  }{\psem}_{a}(q_0\otimes \cdots \otimes q_{n-1})=q_{n-1}\otimes \cdots \otimes q_0
\\[0.5em]

\llbracket \iota_1; \iota_2 \rrbracket\varphi &= \llbracket \iota_2 \rrbracket (\llbracket \iota_1 \rrbracket\varphi)
\end{array}
\]
}
{\scriptsize
$
\begin{array}{l}
\\[0.2em]
\downarrow \alpha(b)\overline{q}=\overline{q}
\qquad
\downarrow (q_1\otimes \cdots \otimes q_n) = \downarrow q_1\otimes \cdots \otimes \downarrow q_n
\\[0.2em]
\app{\uparrow \overline{q}}{\varphi}{(x,i)}=\app{\alpha(b)\overline{q}}{\varphi}{(x,i)}
\qquad \texttt{where  }\varphi(x,i)=\alpha(b)\overline{q_i}
\\[0.2em]
\app{\uparrow \alpha(b_1)\overline{q}}{\varphi}{(x,i)}=\app{\alpha(b_1+b_2)\overline{q}}{\varphi}{(x,i)}
\qquad \texttt{where  }\varphi(x,i)=\alpha(b_2)\overline{q_i}
\\[0.2em]
\app{q_x}{\varphi}{x}=\app{q_{(x,i)}}{\varphi}{\forall i < \Sigma(x).\;(x,i)}
\\[0.2em]
\app{\uparrow q_x}{\varphi}{x}=\app{\uparrow q_{(x,i)}}{\varphi}{\forall i < \Sigma(x).\;(x,i)}
\end{array}
$
}
%\vspace*{-0.5em}
\caption{\oqasm semantics}
  \label{fig:deno-sema}
\end{figure}

The semantics of an \oqasm program is a partial function
$\llbracket\rrbracket$ from
an instruction $\instr$ and input state $\varphi$ to an output state
$\varphi'$, written 
$\llbracket \instr \rrbracket\varphi=\varphi'$, shown in \Cref{fig:deno-sema}.
% The definition for $\llbracket\rrbracket$ is syntax-driven, meaning that it is defined in terms of the state syntax presented in \Cref{fig:vqir-state}.

% defines the denotational semantics of \oqasm, which maps a \oqasm instruction $\instr \in \{\instr\}$ to its unitary operator on $\varphi \in \hsp{S}^d$.

% The key takeaway of the \oqasm denotational semantics is that given an input $\varphi \in \hsp{S}^d$, a well-typed instruction affects only one qubit (notation: $\varphi{(x,n)}$ or $q_{(x,n)}$) or qubit array (notation: $\varphi{(x)}$ or $q_x$), which means it \emph{does not create entanglement}.
% The benefit is that we can completely describe the state $\varphi$ using $d$ terms instead of considering a length $2^d$ vector, as would generally be required to analyze an $d$-qubit system.

Recall that a state $\varphi$ is a tuple of $d$ qubit values,
modeling the tensor product $q_1\otimes \cdots \otimes q_d$. 
The rules implicitly map each variable $x$ to a
range of qubits in the state, e.g., 
$\varphi(x)$ corresponds to some sub-state $q_k\otimes \cdots \otimes q_{k+n-1}$
where $\Omegasz(x)=n$.
%
Many of the rules in \Cref{fig:deno-sema} update a \emph{portion} of a
state. $\app{q_{(x,i)}}{\varphi}{(x,i)}$ updates the $i$-th
qubit of variable $x$ to be the (single-qubit) state $q_{(x,i)}$, and
$\app{q_{x}}{\varphi}{x}$ to update variable $x$ according to
the qubit \emph{tuple} $q_x$.
$\app{\uparrow q_{(x,i)}}{\varphi}{(x,i)}$ and $\app{\uparrow q_{x}}{\varphi}{x}$ are similar, except that they also accumulate the previous global phase of $\varphi(x,i)$ (or $\varphi(x)$).
$\downarrow$ is to convert a qubit $\alpha(b)\overline{q}$ to an unphased qubit $\overline{q}$.
%Thus, we have $\downarrow \alpha(b)\overline{q}=\overline{q}$ 
%and $\downarrow (q_1\otimes...\otimes q_n) = \downarrow q_1\otimes...\otimes \downarrow q_n$. 
%$\app{\uparrow q_{(x,i))}}{\varphi}{(x,i)}$ means to put back the global phase to the result qubit assigning to $(x,i)$. 
%%If $\varphi(x,i)=e^{2\pi i b}\overline{q}$ 
%and the result $q_{(x,i)}=\overline{q_{(x,i)}}$, 
%then we assign $e^{2\pi i b}\overline{q_{(x,i)}}$ to $(x,i)$;
%if the result $q_{(x,i)}=e^{2\pi i b_1}\overline{q_{(x,i)}}$, then we assign $e^{2\pi i (b+b_1)}\overline{q_{(x,i)}}$ to $(x,i)$. $\app{\uparrow q_{x}}{\varphi}{x}$ applies the above scenario to a list of qubits $q_k\otimes ... \otimes q_{k+n-1}$
%where $\Omegasz(x)=n$.

Function $\xsem$ updates the state of a single
qubit according to the rules for the standard quantum gate $X$.  
\texttt{cu} is a conditional operation
depending on the \texttt{Nor}-basis qubit $(x,i)$. 
\texttt{RZ} (or $\texttt{RZ}^{-1}$) is an z-axis phase rotation operation.
Since it applies to \texttt{Nor}-basis, it applies a global phase.
By \Cref{thm:sem-same}, when it is compiled to \sqir,
the global phase might be turned into a local one.
For example, to prepare the state $\sum_{j=0}^{2^n\sminus 1}(-i)^x\ket{x}$ \cite{ChildsNAND}, 
a series of Hadamard gates are applied, followed by several controlled-\texttt{RZ} gates on $x$,
where the controlled-\texttt{RZ} gates are definable by \oqasm.
\texttt{SR} (or
$\texttt{SR}^{-1}$) applies an $m+1$ series of \texttt{RZ} (or
$\texttt{RZ}^{-1}$) rotations where the $i$-th rotation
applies a phase of $\alpha({\frac{1}{2^{m-i+1}}})$
(or $\alpha({-\frac{1}{2^{m-i+1}}})$).
$\qsem$ applies an approximate quantum Fourier transform; $\ket{y}$ is an abbreviation of
$\ket{b_1}\otimes \cdots \otimes \ket{b_i}$ (assuming $\Omegasz(y)=i$) and $n$ is the degree of approximation.
If $n = i$, then the operation is the standard QFT\@.
Otherwise, each qubit in the state is mapped to $\qket{\frac{y}{2^{n-k}}}$, which is equal to $\frac{1}{\sqrt{2}}(\ket{0} + \alpha(\frac{y}{2^{n-k}})\ket{1})$ when $k < n$ and $\frac{1}{\sqrt{2}}(\ket{0} + \ket{1}) = \ket{+}$ when $n \leq k$ (since $\alpha(n) = 1$ for any natural number $n$).
$\qsem^{-1}$ is the inverse function of $\qsem$. 
Note that the input state to $\qsem^{-1}$ is guaranteed to have the form $\bigotimes_{k=0}^{i-1}(\qket{\frac{y}{2^{n-k}}})$ because it has type $\tphi{n}$.
$\psem_l$, $\psem_r$, and
$\psem_a$ are the semantics for \itext{Lshift}, 
\itext{Rshift}, and \itext{Rev}, respectively.   
% Several takeaways about \oqasm denotational semantics.
% For any operation application within the space domain $\hsp{S}^d$, the semantic application $U$ only affects the specific qubit ($\varphi_{(x,n)}$) / qubit array ($\varphi_{x}$) that it targets at, which does not create entanglement with other subsystems.
% This clear separation only works for the domain $\hsp{S}^d$.
% When we compile these operations to \sqir and see their effects on a general Hilbert space $\hsp{H}$, they might have entanglement effects.
% \yxp{Even if we turn it into unitary over the Hilbert space, it still does not generate entanglement with other subsystems.}
% \liyi{Can you have CNOT x y when x is Had and y is in Nor, then you will have entanglement. }
% However, the clear separation in $\hsp{S}^d$ provides a decompositional and analytical way of verifying and validating quantum oracles; thus, each sub-oracle component can be analyzed individually. The potential entanglements in a general Hilbert space become the naturally extended (additive) superposition effects.
% In addition, all semantic functions in Fig.~\ref{fig:deno-sema} are carefully engineered to only target qubits in a register $\varphi$ and do not target individual vectors in the vector space $\varphi$ represents.
% For example, $\xsem$ is defined for a basis phase space case $\ket{c}$, and we also define the case for superposition $\frac{1}{\sqrt{2}}(\ket{0}+(-1)^c\ket{1})$. We do not assume that the semantics of the basis phase space is automatically extended to dealing with individual elements in the superposition case.
% By using the semantics to prove quantum oracle properties, we only need to consider $O(n)$ qubits instead of the possible $2^n$ expanded vector elements.
% The semantics of a universal quantum assembly language like \sqir, by contrast, represents a quantum state as a unitary matrix whose size is \emph{exponential} in the number of vectors by expanding qubits to vectors in a register. \sqir's semantics also relies on using concrete qubits; a unitary matrix and virtual positions would inject a virtual-to-physical mapping into the semantic definition, which can severely complicate proofs~\cite{PQPC}. This leads to the successful correctness proof of the QFT-adder for the first time (Sec.~\ref{sec:op-verification}).
% We only define semantic functions for qubit forms when it is possible to apply them. For example, we do not define $\xsem$ for the form $\frac{1}{\sqrt{2}}(\ket{0}+e^{2\pi{i} b}\ket{1})$, because the \oqasm type system does not allow it. 

\subsection{OQASM Metatheory}\label{sec:metatheory}

\myparagraph{Soundness}
The following statement is proved: well-typed \oqasm programs are well-defined; i.e., the type system is sound concerning the semantics. 
Below is the well-formedness of an \oqasm state.

\begin{definition}[Well-formed \oqasm state]\label{appx:well-formed}\rm 
  A state $\varphi$ is \emph{well-formed}, written
  $\Sigma;\Omega \vdash \varphi$, iff:
\begin{itemize}
\item For every $x \in \Omega$ such that $\Omegaty(x) = \texttt{Nor}$,
  for every $k <\Omegasz(x)$, $\varphi(x,k)$ has the form
  $\alpha(r)\ket{b}$.

\item For every $x \in \Omega$ such that $\Omegaty(x) = \tphi{n}$ and $n \le \Omegasz(x)$,
  there exists a value $\upsilon$ such that for
  every $k < \Omegasz(x)$, $\varphi(x,k)$ has the form
  $\alpha(r)\qket{\frac{\upsilon}{ 2^{n- k}}}$.\footnote{Note that $\Phi(x) = \Phi(x + n)$, where the integer $n$ refers to phase $2 \pi n$; so multiple choices of $\upsilon$ are possible.}
\end{itemize}
\end{definition}

\noindent
Type soundness is stated as follows; the proof is by induction on $\instr$ and is mechanized in Roqc.

\begin{theorem}\label{thm:type-sound-oqasm}\rm[\oqasm type soundness]
If $\Sigma; \Omega \vdash \instr \triangleright \Omega'$ and $\Sigma;\Omega \vdash \varphi$ then there exists $\varphi'$ such that $\llbracket \instr \rrbracket\varphi=\varphi'$ and $\Sigma;\Omega' \vdash \varphi'$.
\end{theorem}

\myparagraph{Algebra}
Mathematically, the set of well-formed $d$-qubit \oqasm states for a given $\Omega$ can be interpreted as a subset $\hsp{S}^d$ of a $2^d$-dimensional Hilbert space $\hsp{H}^d$ \footnote{A \emph{Hilbert space} is a vector space with an inner product that is complete with respect to the norm defined by the inner product. $\hsp{S}^d$ is a sub\emph{set}, not a sub\emph{space} of $\hsp{H}^d$ because $\hsp{S}^d$ is not closed under addition: Adding two well-formed states can produce a state that is not well-formed.}. The semantics function $\llbracket \rrbracket$ can be interpreted as a $2^d \times 2^d$ unitary matrix, as is standard when representing the semantics of programs without measurement~\cite{PQPC}.
Because \oqasm's semantics can be viewed as a unitary matrix, correctness properties extend by linearity from $\hsp{S}^d$ to $\hsp{H}^d$---an oracle that performs addition for classical \texttt{Nor} inputs will also perform addition over a superposition of \texttt{Nor} inputs. The following statement is proved: $\hsp{S}^d$ is closed under well-typed \oqasm programs.

Given a qubit size map $\Sigma$ and type environment $\Omega$, the set of \oqasm programs that are well-typed concerning $\Sigma$ and $\Omega$ (i.e., $\Sigma;\Omega \vdash \instr \triangleright \Omega'$) form an algebraic structure $(\{\instr\},\Sigma, \Omega,\hsp{S}^d)$, where $\{\instr\}$ defines the set of valid program syntax, such that there exists $\Omega'$, $\Sigma;\Omega \vdash \instr \triangleright \Omega'$ for all $\instr$ in $\{\instr\}$; $\hsp{S}^d$ is the set of $d$-qubit states on which programs $\instr\in \{\instr\}$ are run, and are well-formed ($\Sigma;\Omega \vdash \varphi$) according to \Cref{appx:well-formed}.
From the \oqasm semantics and the type soundness theorem, for all $\instr \in \{\instr\}$ and $\varphi \in \hsp{S}^d$, such that $\Sigma;\Omega \vdash \instr \triangleright \Omega'$ and $\Sigma;\Omega \vdash \varphi$, and $\llbracket \instr \rrbracket\varphi=\varphi'$, $\Sigma;\Omega' \vdash \varphi'$, and $\varphi' \in \hsp{S}^d$. Thus, $(\{\instr\},\Sigma, \Omega,\hsp{S}^d)$, where $\{\instr\}$ defines a groupoid.

The groupoid can be certainly extended to another algebraic structure $(\{\instr'\},\Sigma,\hsp{H}^d)$, where $\hsp{H}^d$ is a general $2^d$ dimensional Hilbert space $\hsp{H}^d$ and $\{\instr'\}$ is a universal set of quantum gate operations.
Clearly, the following is true: $\hsp{S}^d \subseteq \hsp{H}^d$ and $\{\instr\} \subseteq \{\instr'\}$, because sets $\hsp{H}^d$ and $\{\instr'\}$ can be acquired by removing the well-formed ($\Sigma;\Omega \vdash \varphi$) and well-typed ($\Sigma;\Omega \vdash \instr \triangleright \Omega'$) definitions for $\hsp{S}^d$ and $\{\instr\}$, respectively.
$(\{\instr'\},\Sigma,\hsp{H}^d)$ is a groupoid because every \oqasm operation is valid in a traditional quantum language like \sqir. The following two theorems are to connect \oqasm operations with operations in the general Hilbert space: 

 \begin{theorem}\label{thm:subgroupoid}\rm
   $(\{\instr\},\Sigma, \Omega,\hsp{S}^d) \subseteq (\{\instr\},\Sigma,\hsp{H}^d)$ is a subgroupoid.
 \end{theorem}

\begin{theorem}\label{thm:sem-same}\rm
Let $\ket{y}$ be an abbreviation of $\bigotimes_{m=0}^{d-1} \alpha(r_m) \ket{b_m}$ for $b_m \in \{0,1\}$.
If for every $i\in [0,2^d)$, $\llbracket \instr \rrbracket\ket{y_i}=\ket{y'_i}$, then $\llbracket \instr \rrbracket (\sum_{i=0}^{2^d-1} \ket{y_i})=\sum_{i=0}^{2^d-1} \ket{y'_i}$.
\end{theorem}

The following theorems are proved as corollaries of the compilation correctness theorem from \oqasm to \sqir (\cite{oracleoopsla}). 
\Cref{thm:subgroupoid} suggests that the space $\hsp{S}^d$ is closed under the application of any well-typed \oqasm operation.
\Cref{thm:sem-same} says that \oqasm oracles can be safely applied to superpositions over classical states.\footnote{Note that a superposition over classical states can describe \emph{any} quantum state, including entangled states.}

\begin{figure}[t]
  {\scriptsize
    \begin{mathpar}
      \inferrule[ ]{}{\inot{(x,n)}\xrightarrow{\text{inv}} \inot{(x,n)}}
    
      \inferrule[  ]{}{\texttt{SR}\;m\;x\xrightarrow{\text{inv}} \texttt{SR}^{-1}\;m\;x}
  
      \inferrule[ ]{}{\iqft{n}{x} \xrightarrow{\text{inv}}  \iqft[-1]{n}{x}}   
  
      \inferrule[ ]{}{\texttt{Lshift}\;x\xrightarrow{\text{inv}} \texttt{Rshift}\;x} 
       
      \inferrule[ ]{\instr \xrightarrow{\text{inv}} \instr'}{\texttt{CU}\;(x,n)\;\instr \xrightarrow{\text{inv}} \texttt{CU}\;(x,n)\;\instr'} 
  
      \inferrule[ ]{\instr_1 \xrightarrow{\text{inv}} \instr'_1 \\ \instr_2 \xrightarrow{\text{inv}} \instr'_2}{\instr_1\;;\;\instr_2\xrightarrow{\text{inv}} \instr'_2\;;\;\instr'_1} 
      
    \end{mathpar}
  }
  \caption{Select \oqasm inversion rules}
  \label{fig:exp-reversed-fun}
\end{figure}

\begin{figure}[t]
{\tiny
\hspace*{-2em}
\centering
\begin{tabular}{c@{$\quad=\quad$}c@{\qquad}c@{$\quad=\quad$}c}
  \begin{minipage}{0.25\textwidth}
  \footnotesize
  \Qcircuit @C=0.25em @R=0.35em {
    & \qw & \multigate{3}{(x+a)_n} & \qw \\
    & \vdots & & \\
    & & & \\
    & \qw & \ghost{(x+a)_n} & \qw \\
    }
  \end{minipage}
&
\begin{minipage}{.45\textwidth}
% \includegraphics[width=1\textwidth]{qft-adder.png}
  \footnotesize
  \Qcircuit @C=0.35em @R=0.55em {
     & \qw & \gate{\texttt{SR}\;0} & \multigate{3}{\texttt{SR}\;1} & \qw & \qw & \qw & \multigate{5}{\texttt{SR}\;(n-1)} & \qw  \\
      & & & & & \dots & & &  \\
      & \qw & \qw  &  \ghost{\texttt{SR}\; 1} & \qw & \qw & \qw & \ghost{\texttt{SR}\;(n-1)} & \qw \\
      & & & & & & & &  \\
     & & & & & & & &  \\
   & \qw & \qw & \qw & \qw & \qw & \qw & \ghost{\texttt{SR}\;(n-1)}  & \qw 
    }
\end{minipage}
&  
\begin{minipage}{0.25\textwidth}
  \footnotesize
  \Qcircuit @C=0.25em @R=0.35em {
    & \qw & \multigate{3}{(x-a)_n} & \qw \\
    & \vdots & & \\
    & & & \\
    & \qw & \ghost{(x+a)_n} & \qw \\
    }
  \end{minipage}
&
\begin{minipage}{.45\textwidth}
% \includegraphics[width=1\textwidth]{qft-adder.png}
  \footnotesize
  \Qcircuit @C=0.35em @R=0.55em {
    & \qw & \multigate{5}{\texttt{SR}^{-1} (n-1)} & \qw & \qw & \qw & \multigate{3}{\texttt{SR}^{-1} 1} & \gate{\texttt{SR}^{-1} 0} & \qw \\
    &     &                                  &     & \dots &   &                              &                      &   \\
    & \qw & \ghost{\texttt{SR}^{-1} (n-1)}        & \qw & \qw   & \qw & \ghost{\texttt{SR}^{-1} 1} & \qw & \qw  \\
      & & & & & & & &  \\
     & & & & & & & &  \\
    & \qw & \ghost{\texttt{SR}^{-1} (n-1)} & \qw & \qw & \qw & \qw & \qw & \qw 
    }
\end{minipage}
\end{tabular}
}
\caption{Addition/subtraction circuits are inverses}
\label{fig:circuit-add-sub}
\end{figure}

\oqasm programs are easily invertible, as shown by the rules in \Cref{fig:exp-reversed-fun}.
This inversion operation is useful for constructing quantum oracles; for example, the core logic in the QFT-based subtraction circuit is just the inverse of the core logic in the addition circuit (\Cref{fig:exp-reversed-fun}).
This allows us to reuse the proof of addition in the proof of subtraction.
The inversion function satisfies the following properties:

 \begin{theorem}\label{thm:reversibility}\rm[Type reversibility]
    For any well-typed program $\instr$, such that $\Sigma; \Omega \vdash \instr \triangleright \Omega'$, its inverse $\instr'$, where $\instr \xrightarrow{\text{inv}} \instr'$, is also well-typed and $\Sigma;\Omega' \vdash \instr' \triangleright \Omega$. Moreover, $\llbracket \instr ; \instr' \rrbracket \varphi=\varphi$.
 \end{theorem}




